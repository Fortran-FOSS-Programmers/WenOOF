\section{Introduction}\label{sec:introduction}
\subsection{Background}

Partial Differential Equation is a differential equation that contains unknow multivariable functions and theis partial derivatives. PDEs are used to formulate problems involving functions of several variables, and can be used to describe a wide variety of phenomena such as sound, heat, electrostatics, electrodynamics, fluid dynamics, elasticity, or quantum mechanics.

constitutes a class of mathematical models of paramount relevance, it being applied to the modelling of a wide range of problems. Briefly, an IVP is an Ordinary Differential Equations system (ODE) coupled with specified initial values of the unknown state variables, the solution of which are searched at a given time after the initial time considered.

The prototype of IVP can be expressed as:
\begin{equation}
  \begin{matrix}
  U_t = R(t,U) \\
  U_0 = U(t_0)
  \end{matrix}
\label{eq:IVP}
\end{equation}
where $U(t)$ is the vector of state variables being a function of the time-like independent variable $t$, $U_t = \frac{dU}{dt} = R(t,U)$ is the (vectorial) residuals function and $U(t_0)$ is the (vectorial) initial conditions, namely the state variables function evaluated at the initial time $t_0$. In general, the residuals function $R$ is a function of the state variable $U$ through which it is a function of time, but it can be also a direct function of time, thus in general $R=R(t,U(t))$ holds.

The problem prototype \ref{eq:IVP} is ubiquitous in the mathematical modelling of physical problems: essentially whenever an \emph{evolutionary} phenomenon is considered the prevision (simulation) of the future solutions involves the solution of an IVP. As a matter of facts, many physical problems (fluid dynamics, chemistry, biology, evolutionary-anthropology, \dots) are described by means of an IVP.

It is worth noting that the state vector variables $U$ and its corresponding residuals function $U_t = \frac{dU}{dt} = R(t,U)$ are \emph{problem dependent}: the number and the meaning of the state variables as well as the equations governing their evolution (that are embedded into the residuals function) are different for the Navier-Stokes conservation laws with respect the Burgers one, as an example. Nevertheless, the \emph{solvers} used for the prediction of the Navier-Stokes equations evolution (hereafter the \emph{solution}) are the same that are used for Burgers equations time-integration. As a consequence, the solution of the IVP model prototype can be generalized, allowing the application of the same solver to many different problems, thus eliminating the necessity to re-implement the same solver for each different problem.

FOODIE library is designed for solving the generalized IVP \ref{eq:IVP}, it being completely unaware of the actual problem's definition. FOODIE library provides a high-level, well-documented, simple Application Program Interface (API) for many well-known ODE integration solvers, its aims being twofold:

\begin{itemize}
  \item provide a robust set of ODE solvers ready to be applied to a wide range of different problems;
  \item provide a simple framework for the rapid development of new ODE solvers.
\end{itemize}

{\color{red} Add citations to IVP, Cauchy, ODE references.}

\subsection{Related Works}

{\color{red} To be written.}

\subsection{Motivations and Aims}

FOODIE library must:

\begin{itemize}
  \item be written in modern Fortran (standard 2008 or newer);
  \item be written by means of Object Oriented Programming (OOP) paradigm;
  \item be well documented;
  \item be Tests Driven Developed (TDD);
  \item be collaboratively developed;
  \item be free.
\end{itemize}

FOODIE, meaning Fortran Object oriented Ordinary Differential Equations integration library, has been developed with the aim to satisfy the above specifications. The present paper is its first comprehensive presentation.

The Fortran (Formula Translator) programming language is the \emph{de facto} standard into computer science field: it strongly facilitates the effective and efficient translation of (even complex) mathematical and numerical models into an operative software without compromise on computations speed and accuracy. Moreover, its simple syntax is suitable for scientific researchers that are interested (and skilled) in the physical aspects of the numerical computations rather than computer technicians. Consequently, we develop FOODIE using Fortran language: FOODIE is written by research scientists for research scientists.

One key-point of the FOODIE development is the \emph{problem generalization}: the problem solved must be the IVP \ref{eq:IVP} rather than any of its actual definitions. Consequently, we must rely on a \emph{generic} implementation of the solvers. To this aim, OOP is very useful: it allows to express IVP \ref{eq:IVP} in a very concise and clear formulation that is really generic. In particular, our implementation is based on \emph{Abstract Calculus Pattern} (ACP) concept.

\paragraph{The Abstract Calculus Pattern}
The abstract calculus pattern provides a simple solution for the connection between the very high-level expression of IVP \ref{eq:IVP} and the eventual concrete (low-level) implementation of the ODE problem being solved. ACP essentially constitutes a \emph{contract} based on an Abstract Data Type (ADT): we specify an ADT supporting a certain set of mathematical operators (differential and integral ones) and implement FOODIE solvers only on the basis of this ADT. FOODIE clients must formulate the ODE problem under integration defining their own concrete extensions of our ADT (implementing all the deferred operators). Such an approach defines the abstract calculus pattern: FOODIE solvers are aware of only the ADT, while FOODIE clients extend the ADT for defining the concrete ODE problem.

Is is worth noting that this ACP emancipates the solvers implementations from any low-level problem-dependent details: the ODE solvers developed with this pattern are extremely concise, clear, maintainable and less errors-prone with respect a low-level (non abstract) pattern. Moreover, the FOODIE clients can use solvers being extremely robust: as a matter of facts, FOODIE solvers are expressed in a formulation very close to the mathematical one and are tested on an extremely varying family of problems. As shown in the following, such a great flexibility does not compromise the computational efficiency.

The present paper is organized as following: in section \ref{sec:MNmodels} a brief description of the mathematical and numerical methods currently implemented into FOODIE is presented; in section \ref{sec:API} a detailed discussion on the implementation specifications is provided by means of an analytical code-listings review; in section \ref{sec:Tests} a verification analysis on the results of FOODIE applications is presented; section \ref{sec:parallel} provides an analysis of FOODIE performances under parallel frameworks scenario like the OpenMP and MPI paradigms; finally, in section \ref{sec:conclusions} concluding remarks and perspectives are depicted.

{\color{red} Add citations to Fortran standards, OOP, TDD, free software, ACP, ADT, Rouson's book references.}

