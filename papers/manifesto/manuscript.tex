%&pdflatex
\documentclass[pdftex,preprint,3p,times,numbers]{elsarticle}

\journal{Computer Physics Communications}

%\usepackage{moreverb}

\usepackage{hyperref}
\hypersetup{pdfborder={0 0 0}}

\usepackage{float}
\usepackage{wrapfig}
\usepackage{caption}
\usepackage{subcaption}
\usepackage{multirow}
\graphicspath{{images/}}

\usepackage[pdftex,usenames]{xcolor}

\usepackage{booktabs}
%\usepackage{colortbl}
\usepackage{multirow}

\usepackage{amsmath}
\usepackage{amssymb}
\usepackage[utf8x]{inputenc}
\usepackage[T1]{fontenc}

\usepackage{xspace}

\usepackage{xcolor}
\definecolor{Maroon}{cmyk}{0,0.87,0.68,0.32}
\definecolor{RoyalBlue}{cmyk}{1,0.50,0,0}
\definecolor{gray}{cmyk}{0.01,0.01,0.01,0.01}
\usepackage{listings}
\lstdefinelanguage{MyFortran}[08]{Fortran}{morecomment=[l]{\#},morestring=[d]',morekeywords={procedure,pass,deferred,non_overridable,generic,class,is}}
\lstdefinestyle{code}{%
  basicstyle=\footnotesize,%
  backgroundcolor=\color{gray},%
  language=MyFortran,%
  captionpos=b,%
  columns=fixed,%
  keepspaces=true,%
  xleftmargin=10pt,%
  numbers=none,%
  numberstyle={\tiny},%
  keywordstyle=\color{RoyalBlue},%
  % stringstyle={\sffamily},%
  texcl=true,%
  upquote=true,%
  commentstyle=\color{Maroon}%
}

\lstdefinestyle{codesimple}{%
  basicstyle=\footnotesize,%
  backgroundcolor=\color{gray},%
  captionpos=b,%
  columns=fixed,%
  keepspaces=true,%
  xleftmargin=10pt,%
  numbers=none,%
}

\DeclareSymbolFont{extraup}{U}{zavm}{m}{n}
\DeclareMathSymbol{\vardiamond}{\mathalpha}{extraup}{87}
\definecolor{OMP}{RGB}{255,127,0}
\definecolor{MPI}{RGB}{0,127,255}
\definecolor{HYB}{RGB}{127,0,255}

\DeclareGraphicsExtensions{.pdf,.png,.jpg,.bmp,.mps}
\newcommand{\citeh}[1]{\citeauthor{#1} \citenum{#1}}

\renewcommand{\thesubfigure}{\Alph{subfigure}}

\begin{document}

\begin{frontmatter}

\title{WenOOF, WENO interpolation Object Oriented Fortran library based on Abstract Calculus Pattern}

\author[insean]{Zaghi S.\corref{cor1}\fnref{sz}}
\ead{stefano.zaghi@cnr.it}
\fntext[sz]{Ph. D., Aerospace Engineer, Research Scientist, Dept. of Computational Hydrodynamics at CNR-INSEAN.}
\address[insean]{CNR-INSEAN, Istituto Nazionale per Studi ed Esperienze di Architettura Navale, Via di Vallerano 139, Rome, Italy, 00128}
\cortext[cor1]{Corresponding author}

\author[dima]{Rossi G.\fnref{gr}}
\ead{giacomo.rossi@uniroma1.it}
\fntext[gr]{Ph.D., Space Engineer, Research Fellow, Dept. of Mechanical and Aerospace Engineering at Universit\'{a} di Roma ``Sapienza''}
\address[dima]{Dipartimento di Ingegneria Meccanica e Aerospaziale, Universit\'{a} di Roma ``Sapienza'', Via Eudossiana 18, Rome, Italy, 00184}

\begin{abstract}
  The (numerical) solution of partial differential equations (PDEs) can lead to discontinuous solutions. Weighted Essentially Non-Oscillatory shock capturing schemes can handle such solutions and show desiderable properties as Total Variation Diminishing and high order of accuracy in smooth regions. The present paper is the first \emph{manifesto} of WenOOF, a library aimed to implement WENO reconstruction and interpolation schemes by means of a clear, concise and efficient \emph{abstract} interface. WenOOF, meaning WENO interpolation Object Oriented Fortran library, has manifolds aims: to provide a set to built-in numerical schemes that are accurate, robust, validated and efficient and to allow easy application of these schemes to (almost) all PDEs by means of an effective Abstract Calculus Pattern. The key idea is to allow the same scheme-implementation to be applied to all reconstruction/interpolation problems thus avoiding the re-implementation of the reconstruction scheme for each different conservation problem: code re-usability is consequently maximized, WenOOF being a general robust framework. Besides, the same framework also allows rapid development of new WENO schemes due to the high abstraction level of the library itself.

WenOOF is a modern Fortran library which main features are:
\begin{description}
  \item[Free] WenOOF is a free software;
  \item[OOP] WenOOF is based on Object Oriented Programming paradigm;
  \item[TDD] the WenOOF development follows the Test Driven Development software process;
  \item[Accurately documented] the WenOOF documentation is based on high quality, first class solutions embedding detailed (mathematical) descriptions directly inside code sources; moreover, comprehensive hyper-linked documentations is also provided;
  \item[Collaborative] the development of WenOOF takes advantage of web communications, the main project being hosted on GitHub.
\end{description}

The present paper is the first announcement of WenOOF project: the current implementation is extensively discussed and its capabilities are proved by means of tests and examples.
\end{abstract}

\begin{keyword}
  Weighted Essentially Non-Oscillatory (WENO) \sep
  Partial Differential Equations (PDEs) \sep
  Object Oriented Programming (OOP) \sep
  Abstract Calculus Pattern (ACP) \sep
  Fortran
\end{keyword}

\end{frontmatter}

{\bf PROGRAM SUMMARY}

\begin{small}
\noindent
\emph{Manuscript Title:} WenOOF, WENO interpolation Object Oriented Fortran library based on Abstract Calculus Pattern \\
\emph{Authors:} Zaghi, S., Rossi, G. \\
\emph{Program title:} WenOOF \\
\emph{Journal Reference:} \\
\emph{Catalogue identifier:} \\
\emph{Licensing provisions:} GNU General Public License (GPL) v3 \\
\emph{Programming language:} Fortran (standard 2008 or newer); developed and tested with GNU gfortran 6.2 or newer \\
\emph{Computer(s) for which the program has been designed:} designed for shared-memory multi-cores workstations and for hybrid distributed/shared-memory supercomputers, but any computer system with a Fortran (2008+) compiler is suited \\
\emph{Operating system(s) for which the program has been designed:} designed for POSIX architecture and tested on GNU/Linux one \\
\emph{RAM required to execute with typical data: bytes:} $[1MB,1GB]\times core$, simulation-dependent \\
\emph{Has the code been vectorised or parallelized?:} the library is not aware of the parallel back-end, it providing a high-level models, but the provided tests suite shows parallel usage by means of MPI library and OpenMP paradigm \\
\emph{Number of processors used:} tested up to 256 \\
\emph{Supplementary material:}    \\
\emph{Keywords:} WENO, OOP, ACP, Fortran \\
\emph{CPC Library Classification:} 4.3 Differential Equations, 4.10 Interpolation, 12 Gases and Fluids \\
\emph{External routines/libraries used:} \\
\emph{CPC Program Library subprograms used:} \\
\emph{Nature of problem:} \\
Numerical integration of (general) Partial Differential Equations system \\
\emph{Solution method:} \\
\emph{Restrictions:} \\
\emph{Unusual features:} \\
\emph{Additional comments:} \\
\emph{Running time:} \\
\emph{References:} \\
% \begin{thebibliography}{0}
% \end{thebibliography}
\end{small}

\section{Introduction}\label{sec:introduction}

Interpolation is the process of deriving a simple function from a set of discrete data points so that the function passes through all the given data points (i.e. reproduces the data points exactly) and can be used to estimate data points in-between the given ones.

Interpolation is also used to simplify complicated functions by sampling data points and then interpolating them using a simpler function. Polynomials are commonly used for interpolation because they are easier to evaluate, differentiate, and integrate. Unfortunately, interpolation of order greater than one can suffer of the Gibbs' phenomenon~\cite{gibbs-b-1906} next to discontinuities.

The original idea of WENO schemes~\cite{liu-1994} is to use a convex combination of all candidate stencils (instead of using only the smoothest one as in ENO schemes~\cite{harten-1987}) to obtain high order reconstruction: this approach can obviously be extended to interpolation process, leading to an high order oscillatory free interpolation.

{\color{red} Add interpolation background and citation to interpolation related works.}


\clearpage

\section{Mathematical and Numerical Models}\label{sec:MNmodels}

Assume we have a uniform mesh $x_1, x_2, \dots x_n$ with $\Delta x = x_{n+1} - x_n$ and that we know the values of a function $u$ at all the grid points, that is $u_i = u(x_i)$ for all $i$. We would like to find an approximation of the function $u(x)$ at the point $x^*$ other than the nodes $x_i$, with $x_{i-\frac{1}{2}} < x^* < x_{i+\frac{1}{2}}$, where $x_{i-\frac{1}{2}}$ and $x_{i+\frac{1}{2}}$ are the cell interfaces.

For a $r^{th}$ order accurate interpolation, there are $r$ candidate stencils next to the target point $x^*$: we denote these stencil as $S_k$, where $k=0, \dots, r-1$ labels the stencils from the leftmost stencil to the rightmost stencil in that order. Using  the Lagrange form of the interpolation polynomial, the polynom $p_k(x)$ over the stencil $S_k$ can be written as:

\begin{equation}
  \label{eq:Lagrange}
  p_k(x^*) = \sum_{j=0}^{r-1} u_{i-r+k+j+1} \sum_{\substack{l=0 \\ l \neq j}}^{r-1} \frac{x^* - x_{i-r+k+l+1}}{x_{i-r+k+j+1} - x_{i-r+k+l+1}} = \sum_{j=0}^{r-1} a_{k,i-r+j+1} u_{i-r+k+j+1}
\end{equation}

where $a_{k,i-r+j+1}$ are the Lagrange coefficients of the stencil $S_k$.

In table~\ref{tab:polynomial_coefficients} are reported the polynomial coefficients from $r=2$ to $r=9$ for all the interpolating stencils, for $x^* = x_{i+\frac{1}{2}}$; polynomial coefficients for $x^*=x_{i-\frac{1}{2}}$ can be obtained by table~\ref{tab:polynomial_coefficients} by symmetry.

\begin{table}
  \begin{center}
    \caption{Polynomial coefficients from $r=2$ to $r=9$ for $x^*=x_{i+\frac{1}{2}}$}
    \label{tab:polynomial_coefficients}
    \begin{tabular}{ccccccccccc}
      \toprule
      $r$  &  $k$  &  $j=0$  &  $j=1$  &  $j=2$  &  $j=3$  &  $j=4$  &  $j=5$  &  $j=6$  &  $j=7$  &  $j=8$  \\
      \midrule
      9  &  0  &  $ \frac{6435}{32768}$  &  $-\frac{7293}{4096}$  &  $ \frac{58905}{8192}$  &  $-\frac{69615}{4096}$  &  $ \frac{425425}{16384}$  &  $-\frac{109395}{4096}$  &  $ \frac{153153}{8192}$  &  $-\frac{36465}{4096}$  &  $ \frac{109395}{32768}$ \\ \addlinespace
         &  1  &  $-\frac{ 429}{32768}$  &  $ \frac{ 495}{4096}$  &  $-\frac{ 4095}{8192}$  &  $ \frac{ 5005}{4096}$  &  $-\frac{ 32175}{16384}$  &  $ \frac{  9009}{4096}$  &  $-\frac{ 15015}{8192}$  &  $ \frac{ 6435}{4096}$  &  $ \frac{  6435}{32768}$ \\ \addlinespace
         &  2  &  $ \frac{  99}{32768}$  &  $-\frac{ 117}{4096}$  &  $ \frac{ 1001}{8192}$  &  $-\frac{ 1287}{4096}$  &  $ \frac{  9009}{16384}$  &  $-\frac{  3003}{4096}$  &  $ \frac{  9009}{8192}$  &  $ \frac{ 1287}{4096}$  &  $-\frac{   429}{32768}$ \\ \addlinespace
         &  3  &  $-\frac{  45}{32768}$  &  $ \frac{  55}{4096}$  &  $-\frac{  495}{8192}$  &  $ \frac{  693}{4096}$  &  $-\frac{  5775}{16384}$  &  $ \frac{  3465}{4096}$  &  $ \frac{  3465}{8192}$  &  $-\frac{  165}{4096}$  &  $ \frac{    99}{32768}$ \\ \addlinespace
         &  4  &  $ \frac{  35}{32768}$  &  $-\frac{  45}{4096}$  &  $ \frac{  441}{8192}$  &  $-\frac{  735}{4096}$  &  $ \frac{ 11025}{16384}$  &  $ \frac{  2205}{4096}$  &  $-\frac{   735}{8192}$  &  $ \frac{   63}{4096}$  &  $-\frac{    45}{32768}$ \\ \addlinespace
         &  5  &  $-\frac{  45}{32768}$  &  $ \frac{  63}{4096}$  &  $-\frac{  735}{8192}$  &  $ \frac{ 2205}{4096}$  &  $ \frac{ 11025}{16384}$  &  $-\frac{   735}{4096}$  &  $ \frac{   441}{8192}$  &  $-\frac{   45}{4096}$  &  $ \frac{    35}{32768}$ \\ \addlinespace
         &  6  &  $ \frac{  99}{32768}$  &  $-\frac{ 165}{4096}$  &  $ \frac{ 3465}{8192}$  &  $ \frac{ 3465}{4096}$  &  $-\frac{  5775}{16384}$  &  $ \frac{   693}{4096}$  &  $-\frac{   495}{8192}$  &  $ \frac{   55}{4096}$  &  $-\frac{    45}{32768}$ \\ \addlinespace
         &  7  &  $-\frac{ 429}{32768}$  &  $ \frac{1287}{4096}$  &  $ \frac{ 9009}{8192}$  &  $-\frac{ 3003}{4096}$  &  $ \frac{  9009}{16384}$  &  $-\frac{  1287}{4096}$  &  $ \frac{  1001}{8192}$  &  $-\frac{  117}{4096}$  &  $ \frac{    99}{32768}$ \\ \addlinespace
         &  8  &  $ \frac{6435}{32768}$  &  $ \frac{6435}{4096}$  &  $-\frac{15015}{8192}$  &  $ \frac{ 9009}{4096}$  &  $-\frac{ 32175}{16384}$  &  $ \frac{  5005}{4096}$  &  $-\frac{  4095}{8192}$  &  $ \frac{  495}{4096}$  &  $-\frac{   429}{32768}$ \\ \addlinespace

      8  &  0  &  $-\frac{ 429}{ 2048}$  &  $ \frac{3465}{2048}$  &  $-\frac{12285}{2048}$  &  $ \frac{25025}{2048}$  &  $-\frac{ 32175}{ 2048}$  &  $ \frac{ 27027}{2048}$  &  $-\frac{ 15015}{2048}$  &  $ \frac{ 6435}{2048}$  \\ \addlinespace
         &  1  &  $ \frac{  33}{ 2048}$  &  $-\frac{ 273}{2048}$  &  $ \frac{ 1001}{2048}$  &  $-\frac{ 2145}{2048}$  &  $ \frac{  3003}{ 2048}$  &  $-\frac{  3003}{2048}$  &  $ \frac{  3003}{2048}$  &  $ \frac{  429}{2048}$  \\ \addlinespace
         &  2  &  $-\frac{   9}{ 2048}$  &  $ \frac{  77}{2048}$  &  $-\frac{  297}{2048}$  &  $ \frac{  693}{2048}$  &  $-\frac{  1155}{ 2048}$  &  $ \frac{  2079}{2048}$  &  $ \frac{   693}{2048}$  &  $-\frac{   33}{2048}$  \\ \addlinespace
         &  3  &  $ \frac{   5}{ 2048}$  &  $-\frac{  45}{2048}$  &  $ \frac{  189}{2048}$  &  $-\frac{  525}{2048}$  &  $ \frac{  1575}{ 2048}$  &  $ \frac{   945}{2048}$  &  $-\frac{   105}{2048}$  &  $ \frac{    9}{2048}$  \\ \addlinespace
         &  4  &  $-\frac{   5}{ 2048}$  &  $ \frac{  49}{2048}$  &  $-\frac{  245}{2048}$  &  $ \frac{ 1225}{2048}$  &  $ \frac{  1225}{ 2048}$  &  $-\frac{   245}{2048}$  &  $ \frac{    49}{2048}$  &  $-\frac{    5}{2048}$  \\ \addlinespace
         &  5  &  $ \frac{   9}{ 2048}$  &  $-\frac{ 105}{2048}$  &  $ \frac{  945}{2048}$  &  $ \frac{ 1575}{2048}$  &  $-\frac{   525}{ 2048}$  &  $ \frac{   189}{2048}$  &  $-\frac{    45}{2048}$  &  $ \frac{    5}{2048}$  \\ \addlinespace
         &  6  &  $-\frac{  33}{ 2048}$  &  $ \frac{ 693}{2048}$  &  $ \frac{ 2079}{2048}$  &  $-\frac{ 1155}{2048}$  &  $ \frac{   693}{ 2048}$  &  $-\frac{   297}{2048}$  &  $ \frac{    77}{2048}$  &  $-\frac{    9}{2048}$  \\ \addlinespace
         &  7  &  $ \frac{ 429}{ 2048}$  &  $ \frac{3003}{2048}$  &  $-\frac{ 3003}{2048}$  &  $ \frac{ 3003}{2048}$  &  $-\frac{  2145}{ 2048}$  &  $ \frac{  1001}{2048}$  &  $-\frac{   273}{2048}$  &  $ \frac{   33}{2048}$  \\ \addlinespace

      7  &  0  &  $ \frac{ 231}{ 1024}$  &  $-\frac{ 819}{ 512}$  &  $ \frac{ 5005}{1024}$  &  $-\frac{ 2145}{ 256}$  &  $ \frac{  9009}{ 1024}$  &  $-\frac{  3003}{ 512}$  &  $ \frac{  3003}{1024}$  \\ \addlinespace
         &  1  &  $-\frac{  21}{ 1024}$  &  $ \frac{  77}{ 512}$  &  $-\frac{  495}{1024}$  &  $ \frac{  231}{ 256}$  &  $-\frac{  1155}{ 1024}$  &  $ \frac{   693}{ 512}$  &  $ \frac{   231}{1024}$  \\ \addlinespace
         &  2  &  $ \frac{   7}{ 1024}$  &  $-\frac{  27}{ 512}$  &  $ \frac{  189}{1024}$  &  $-\frac{  105}{ 256}$  &  $ \frac{   945}{ 1024}$  &  $ \frac{   189}{ 512}$  &  $-\frac{    21}{1024}$  \\ \addlinespace
         &  3  &  $-\frac{   5}{ 1024}$  &  $ \frac{  21}{ 512}$  &  $-\frac{  175}{1024}$  &  $ \frac{  175}{ 256}$  &  $ \frac{   525}{ 1024}$  &  $-\frac{    35}{ 512}$  &  $ \frac{     7}{1024}$  \\ \addlinespace
         &  4  &  $ \frac{   7}{ 1024}$  &  $-\frac{  35}{ 512}$  &  $ \frac{  525}{1024}$  &  $ \frac{  175}{ 256}$  &  $-\frac{   175}{ 1024}$  &  $ \frac{    21}{ 512}$  &  $-\frac{     5}{1024}$  \\ \addlinespace
         &  5  &  $-\frac{  21}{ 1024}$  &  $ \frac{ 189}{ 512}$  &  $ \frac{  945}{1024}$  &  $-\frac{  105}{ 256}$  &  $ \frac{   189}{ 1024}$  &  $-\frac{    27}{ 512}$  &  $ \frac{     7}{1024}$  \\ \addlinespace
         &  6  &  $ \frac{ 231}{ 1024}$  &  $ \frac{ 693}{ 512}$  &  $-\frac{ 1155}{1024}$  &  $ \frac{  231}{ 256}$  &  $-\frac{   495}{ 1024}$  &  $ \frac{    77}{ 512}$  &  $-\frac{    21}{1024}$  \\ \addlinespace

      6  &  0  &  $-\frac{  63}{  256}$  &  $ \frac{ 385}{ 256}$  &  $-\frac{  495}{ 128}$  &  $ \frac{  693}{ 128}$  &  $-\frac{  1155}{  256}$  &  $ \frac{   693}{ 256}$  \\ \addlinespace
         &  1  &  $ \frac{   7}{  256}$  &  $-\frac{  45}{ 256}$  &  $ \frac{   63}{ 128}$  &  $-\frac{  105}{ 128}$  &  $ \frac{   315}{  256}$  &  $ \frac{    63}{ 256}$  \\ \addlinespace
         &  2  &  $-\frac{   3}{  256}$  &  $ \frac{  21}{ 256}$  &  $-\frac{   35}{ 128}$  &  $ \frac{  105}{ 128}$  &  $ \frac{   105}{  256}$  &  $-\frac{     7}{ 256}$  \\ \addlinespace
         &  3  &  $ \frac{   3}{  256}$  &  $-\frac{  25}{ 256}$  &  $ \frac{   75}{ 128}$  &  $ \frac{   75}{ 128}$  &  $-\frac{    25}{  256}$  &  $ \frac{     3}{ 256}$  \\ \addlinespace
         &  4  &  $-\frac{   7}{  256}$  &  $ \frac{ 105}{ 256}$  &  $ \frac{  105}{ 128}$  &  $-\frac{   35}{ 128}$  &  $ \frac{    21}{  256}$  &  $-\frac{     3}{ 256}$  \\ \addlinespace
         &  5  &  $ \frac{  63}{  256}$  &  $ \frac{ 315}{ 256}$  &  $-\frac{  105}{ 128}$  &  $ \frac{   63}{ 128}$  &  $-\frac{    45}{  256}$  &  $ \frac{     7}{ 256}$  \\ \addlinespace

      5  &  0  &  $ \frac{  35}{  128}$  &  $-\frac{  45}{  32}$  &  $ \frac{  189}{  64}$  &  $-\frac{  105}{  32}$  &  $ \frac{   315}{  128}$  \\ \addlinespace
         &  1  &  $-\frac{   5}{  128}$  &  $ \frac{   7}{  32}$  &  $-\frac{   35}{  64}$  &  $ \frac{   35}{  32}$  &  $ \frac{    35}{  128}$  \\ \addlinespace
         &  2  &  $ \frac{   3}{  128}$  &  $-\frac{   5}{  32}$  &  $ \frac{   45}{  64}$  &  $ \frac{   15}{  32}$  &  $-\frac{     5}{  128}$  \\ \addlinespace
         &  3  &  $-\frac{   5}{  128}$  &  $ \frac{  15}{  32}$  &  $ \frac{   45}{  64}$  &  $-\frac{    5}{  32}$  &  $ \frac{     3}{  128}$  \\ \addlinespace
         &  4  &  $ \frac{  35}{  128}$  &  $ \frac{  35}{  32}$  &  $-\frac{   35}{  64}$  &  $ \frac{    7}{  32}$  &  $-\frac{     5}{  128}$  \\ \addlinespace

      4  &  0  &  $-\frac{   5}{   16}$  &  $ \frac{  21}{  16}$  &  $-\frac{   35}{  16}$  &  $ \frac{   35}{  16}$  \\ \addlinespace
         &  1  &  $ \frac{   1}{   16}$  &  $-\frac{   5}{  16}$  &  $ \frac{   15}{  16}$  &  $ \frac{    5}{  16}$  \\ \addlinespace
         &  2  &  $-\frac{   1}{   16}$  &  $ \frac{   9}{  16}$  &  $ \frac{    9}{  16}$  &  $-\frac{    1}{  16}$  \\ \addlinespace
         &  3  &  $ \frac{   5}{   16}$  &  $ \frac{  15}{  16}$  &  $-\frac{    5}{  16}$  &  $ \frac{    1}{  16}$  \\ \addlinespace

      3  &  0  &  $ \frac{   3}{    8}$  &  $-\frac{   5}{   4}$  &  $ \frac{   15}{   8}$  \\ \addlinespace
         &  1  &  $-\frac{   1}{    8}$  &  $ \frac{   3}{   4}$  &  $ \frac{    3}{   8}$  \\ \addlinespace
         &  2  &  $ \frac{   3}{    8}$  &  $ \frac{   3}{   4}$  &  $-\frac{    1}{   8}$  \\ \addlinespace

      2  &  0  &  $-\frac{   1}{    2}$  &  $ \frac{   3}{   2}$  \\ \addlinespace
         &  1  &  $ \frac{   1}{    2}$  &  $ \frac{   1}{   2}$  \\ \addlinespace
      \bottomrule
    \end{tabular}
  \end{center}
\end{table}

If we consider the big stencil $S = \cup_{i=0}^k S_k$, we can obtain a $(2r-1)^{th}$ accurate interpolation and \eqref{eq:Lagrange} becomes:

\begin{equation}
  \label{eq:Lagrange_big}
  P(x^*) = \sum_{j=0}^{2r-2} u_{i-r+j+1} \sum_{\substack{l=0 \\ l \neq j}}^{2r-2} \frac{x^* - x_{i-r+l+1}}{x_{i-r+j+1} - x_{i-r+l+1}} = \sum_{j=0}^{2r-2} b_{i-r+j+1} u_{i-r+j+1}
\end{equation}

where $b_{i-r+j+1}$ are the Lagrange coefficients of the stencil $S$.

Expression~\eqref{eq:Lagrange_big} can also be written as a linear convex combination of the $r$ approximations of order $r^{th}$~\eqref{eq:Lagrange}

\begin{equation}
  \label{eq:pol_convex}
  P(x^*) = \sum_{i=0}^{r-1} \gamma_i p_i(x^*) \text{, with } \sum_{i=0}^{r-1} \gamma_i = 1
\end{equation}

where $\gamma_r$ are usually referred as the linear weights. The linear weights for the point $x^*$ can be evaluated from the Lagrange coefficients $a_{k,i-r+j+1}$ and $b_{i-r+j+1}$ by means of:

\begin{equation}
  \label{eq:linear_weights}
  \gamma_k(x^*) = \frac{b_{i-r+j+1} - \sum_{l=0}^{j-1} \gamma_l(x^*) a_{k,i-r+l+1}(x^*)}{a_{0,i-r+j+1}(x^*)} \text{, } j=0, \dots, r-1
\end{equation}

In table~\ref{tab:linear_weights} are reported linear weights from $r=2$ to $r=9$ for $x^*=x_{i+\frac{1}{2}}$; linear weights for $x^*=x_{i-\frac{1}{2}}$ can be obtained by table~\ref{tab:linear_weights} by symmetry.

\begin{table}
  \begin{center}
    \caption{Linear weights from $r=2$ to $r=9$ for $x^*=x_{i+\frac{1}{2}}$}
    \label{tab:linear_weights}
    \begin{tabular}{cccccccccc}
      \toprule
      $r$  &  $j=0$  &  $j=1$  &  $j=2$  &  $j=3$  &  $j=4$  &  $j=5$  &  $j=6$  &  $j=7$  &  $j=8$  \\
      \midrule
      9  & $\frac{1}{65536}$  &  $\frac{ 17}{ 8192}$  &  $\frac{ 595}{16384}$  &  $\frac{1547}{ 8192}$  &  $\frac{12155}{32768}$  &  $\frac{2431}{ 8192}$  &  $\frac{1547}{16384}$  &  $\frac{85}{ 8192}$  &  $\frac{17}{65536}$  \\ \addlinespace
      8  & $\frac{1}{16384}$  &  $\frac{105}{16384}$  &  $\frac{1365}{16384}$  &  $\frac{5005}{16384}$  &  $\frac{ 6435}{16384}$  &  $\frac{3003}{16384}$  &  $\frac{ 455}{16384}$  &  $\frac{15}{16384}$  \\ \addlinespace
      7  & $\frac{1}{ 4096}$  &  $\frac{ 39}{ 2048}$  &  $\frac{ 179}{ 1024}$  &  $\frac{ 429}{ 1024}$  &  $\frac{ 1287}{ 4096}$  &  $\frac{ 143}{ 2048}$  &  $\frac{  13}{ 4096}$  \\ \addlinespace
      6  & $\frac{1}{ 1024}$  &  $\frac{ 55}{ 1024}$  &  $\frac{ 165}{  512}$  &  $\frac{ 231}{  512}$  &  $\frac{  165}{ 1024}$  &  $\frac{  11}{ 1024}$  \\ \addlinespace
      5  & $\frac{1}{  256}$  &  $\frac{  9}{   64}$  &  $\frac{  63}{  128}$  &  $\frac{  21}{   64}$  &  $\frac{    9}{  256}$  \\ \addlinespace
      4  & $\frac{1}{   64}$  &  $\frac{ 21}{   64}$  &  $\frac{  35}{   64}$  &  $\frac{    7}{  64}$  \\ \addlinespace
      3  & $\frac{1}{   16}$  &  $\frac{  5}{    8}$  &  $\frac{   5}{   16}$  \\ \addlinespace
      2  & $\frac{1}{    4}$  &  $\frac{  3}{    4}$  \\ \addlinespace
      \bottomrule
    \end{tabular}
  \end{center}
\end{table}

The basic idea of WENO schemes is to use a nonlinear combination of the $r$ interpolations to obtain a $(2r-1)^{th}$ order interpolation in smooth regions and handle stencil with discontinuities: the nonlinear weights, infact, are close to the linear weights if the function in the stencil is smooth and close to $0$ if in that stencil is contained a discontinuity.

\begin{equation}
  \label{eq:WENO_interp}
  u(x^*) = \sum_{i=0}^{r-1} w_i p_i(x^*)
\end{equation}

Following the work of Jiang and Shu~\cite{jiang-1996}, the nonlinear weights are evaluated as:

\begin{equation}
  \label{eq:nonlinear_weights}
  w_k = \frac{\gamma_k}{\left( \epsilon + \beta_k \right)^2}
\end{equation}

where $\epsilon$ is a parameter to avoid division by zero and $\beta_k$ are the smoothness indicators of the function $u$ on the stencil $l$:

\begin{equation}
  \label{eq:IS}
  \beta_k = \sum_{j=1}^{r-1} \Delta x^{2j-1} \int_{x_{i-\frac{1}{2}}}^{x_{i+\frac{1}{2}}} \left( \frac{d^j p_k(x)}{dx^j} \right)^2 dx
\end{equation}

This is clearly just a scaled sum of the square L2 norms of all the derivatives of the relevant interpolation polynomial $p_k(x)$ in the relevant interval $[x_{i−\frac{1}{2}},x_{i+\frac{1}{2}}]$, where the interpolating point is located. The scaling factor $\Delta_x^{2l-2}$ is to make sure that the final explicit formulas for the smoothness indicators do not depend on the mesh size $\Delta x$.

Substitution of~\eqref{eq:Lagrange} for any $k=0,\dots,r-1$ into~\eqref{eq:IS} yelds to:

\begin{equation}
  \label{eq:IS_u}
  \beta_k = \sum_{j=0}^{r-1} \sum_{l=0}^j \sigma_{k,j,l} u_{i+k-j} u_{i+k-l}
\end{equation}

The coefficients $\sigma_{k,j,l}$ are reported in table~\cref{tab:IS_2-5,tab:IS_6}.

\begin{table}
  \begin{center}
    \caption{Smoothness indicators coefficients from $r=2$ to $r=5$}
    \label{tab:IS_2-5}
    \begin{tabular}{ccccccc}
      \toprule
      $r=2$  \\
      $j$  &  $l$  &  $k=0$ &  $k=1$ \\ \addlinespace
      $1$  &  $1$  &  $-2$  &  $-2$  \\ \addlinespace
           &  $0$  &  $ 1$  &  $ 1$  \\ \addlinespace
      $0$  &  $0$  &  $ 1$  &  $ 1$  \\ \addlinespace
      \midrule
      $r=3$  \\
      $j$  &  $l$  &  $k=0$            &  $k=1$            &  $k=2$            \\ \addlinespace
      $2$  &  $2$  &  $ \frac{11}{3}$  &  $ \frac{ 5}{3}$  &  $ \frac{11}{3}$  \\ \addlinespace
           &  $1$  &  $-\frac{31}{3}$  &  $-\frac{13}{3}$  &  $-\frac{19}{3}$  \\ \addlinespace
           &  $0$  &  $ \frac{10}{3}$  &  $ \frac{ 4}{3}$  &  $ \frac{ 4}{3}$  \\ \addlinespace
      $1$  &  $1$  &  $-\frac{19}{3}$  &  $-\frac{13}{3}$  &  $-\frac{31}{3}$  \\ \addlinespace
           &  $0$  &  $ \frac{25}{3}$  &  $ \frac{13}{3}$  &  $ \frac{25}{3}$  \\ \addlinespace
      $0$  &  $0$  &  $ \frac{ 4}{3}$  &  $ \frac{ 4}{3}$  &  $ \frac{10}{3}$  \\ \addlinespace
      \midrule
      $r=4$  \\
      $j$  &  $l$  &  $k=0$                  &  $k=1$                  &  $k=2$                  &  $k=3$                  \\ \addlinespace
      $3$  &  $3$  &  $-\frac{11389}{1440}$  &  $-\frac{ 2989}{1440}$  &  $-\frac{ 2989}{1440}$  &  $-\frac{11389}{1440}$  \\ \addlinespace
           &  $2$  &  $ \frac{14369}{ 480}$  &  $ \frac{ 1283}{ 160}$  &  $ \frac{ 3169}{ 480}$  &  $ \frac{ 9449}{ 480}$  \\ \addlinespace
           &  $1$  &  $-\frac{ 6383}{ 160}$  &  $-\frac{ 5069}{ 480}$  &  $-\frac{ 3229}{ 480}$  &  $-\frac{ 2623}{ 160}$  \\ \addlinespace
           &  $0$  &  $ \frac{25729}{2880}$  &  $ \frac{ 6649}{2880}$  &  $ \frac{ 3169}{2880}$  &  $ \frac{ 6649}{2880}$  \\ \addlinespace
      $2$  &  $2$  &  $ \frac{ 9449}{ 480}$  &  $ \frac{ 3169}{ 480}$  &  $ \frac{ 1283}{ 160}$  &  $ \frac{14369}{ 480}$  \\ \addlinespace
           &  $1$  &  $-\frac{35047}{ 480}$  &  $-\frac{11767}{ 480}$  &  $-\frac{11767}{ 480}$  &  $-\frac{35047}{ 480}$  \\ \addlinespace
           &  $0$  &  $ \frac{44747}{ 960}$  &  $ \frac{13667}{ 960}$  &  $ \frac{11147}{ 960}$  &  $ \frac{28547}{ 960}$  \\ \addlinespace
      $1$  &  $1$  &  $-\frac{ 2623}{ 160}$  &  $-\frac{ 3229}{ 480}$  &  $-\frac{ 5069}{ 480}$  &  $-\frac{ 6383}{ 160}$  \\ \addlinespace
           &  $0$  &  $ \frac{28547}{ 960}$  &  $ \frac{11147}{ 960}$  &  $ \frac{13667}{ 960}$  &  $ \frac{44747}{ 960}$  \\ \addlinespace
      $0$  &  $0$  &  $ \frac{ 6649}{2880}$  &  $ \frac{ 3169}{2880}$  &  $ \frac{ 6649}{2880}$  &  $ \frac{25729}{2880}$  \\ \addlinespace
      \midrule
      $r=5$  \\
      $j$  &  $l$  &  $k=0$                      &  $k=1$                     &  $k=2$                  &  $k=3$                        &  $k=4$                      \\ \addlinespace
      $4$  &  $4$  &  $ \frac{ 1076779}{60480}$  &  $ \frac{ 221869}{60480}$  &  $ \frac{  98179}{60480}$  &  $ \frac{ 221869}{60480}$  &  $ \frac{ 1076779}{60480}$  \\ \addlinespace
           &  $3$  &  $-\frac{ 5121853}{60480}$  &  $-\frac{1079563}{60480}$  &  $-\frac{ 461113}{60480}$  &  $-\frac{ 847303}{60480}$  &  $-\frac{ 3568693}{60480}$  \\ \addlinespace
           &  $2$  &  $ \frac{ 3141559}{20160}$  &  $ \frac{ 671329}{20160}$  &  $ \frac{ 266659}{20160}$  &  $ \frac{ 395389}{20160}$  &  $ \frac{ 1501039}{20160}$  \\ \addlinespace
           &  $1$  &  $-\frac{ 8055511}{60480}$  &  $-\frac{1714561}{60480}$  &  $-\frac{ 601771}{60480}$  &  $-\frac{ 725461}{60480}$  &  $-\frac{ 2569471}{60480}$  \\ \addlinespace
           &  $0$  &  $ \frac{  668977}{30240}$  &  $ \frac{ 139567}{30240}$  &  $ \frac{  20591}{15120}$  &  $ \frac{  20591}{15120}$  &  $ \frac{  139567}{30240}$  \\ \addlinespace
      $3$  &  $3$  &  $-\frac{ 3568693}{60480}$  &  $-\frac{ 847303}{60480}$  &  $-\frac{ 461113}{60480}$  &  $-\frac{1079563}{60480}$  &  $-\frac{ 5121853}{60480}$  \\ \addlinespace
           &  $2$  &  $ \frac{ 8405471}{30240}$  &  $ \frac{2027351}{30240}$  &  $ \frac{1050431}{30240}$  &  $ \frac{2027351}{30240}$  &  $ \frac{ 8405471}{30240}$  \\ \addlinespace
           &  $1$  &  $-\frac{ 2536843}{ 5040}$  &  $-\frac{ 306569}{ 2520}$  &  $-\frac{ 291313}{ 5040}$  &  $-\frac{  57821}{  630}$  &  $-\frac{ 1751863}{ 5040}$  \\ \addlinespace
           &  $0$  &  $ \frac{12627689}{60480}$  &  $ \frac{2932409}{60480}$  &  $ \frac{1228889}{60480}$  &  $ \frac{1650569}{60480}$  &  $ \frac{ 5951369}{60480}$  \\ \addlinespace
      $2$  &  $2$  &  $ \frac{ 1501039}{20160}$  &  $ \frac{ 395389}{20160}$  &  $ \frac{ 266659}{20160}$  &  $ \frac{ 671329}{20160}$  &  $ \frac{ 3141559}{20160}$  \\ \addlinespace
           &  $1$  &  $-\frac{ 1751863}{ 5040}$  &  $-\frac{  57821}{  630}$  &  $-\frac{ 291313}{ 5040}$  &  $-\frac{ 306569}{ 2520}$  &  $-\frac{ 2536843}{ 5040}$  \\ \addlinespace
           &  $0$  &  $ \frac{ 2085371}{ 6720}$  &  $ \frac{ 539351}{ 6720}$  &  $ \frac{ 299531}{ 6720}$  &  $ \frac{ 539351}{ 6720}$  &  $ \frac{ 2085371}{ 6720}$  \\ \addlinespace
      $1$  &  $1$  &  $-\frac{ 2569471}{60480}$  &  $-\frac{ 725461}{60480}$  &  $-\frac{ 601771}{60480}$  &  $-\frac{1714561}{60480}$  &  $-\frac{ 8055511}{60480}$  \\ \addlinespace
           &  $0$  &  $ \frac{ 5951369}{60480}$  &  $ \frac{1650569}{60480}$  &  $ \frac{1228889}{60480}$  &  $ \frac{2932409}{60480}$  &  $ \frac{12627689}{60480}$  \\ \addlinespace
      $0$  &  $0$  &  $ \frac{  139567}{30240}$  &  $ \frac{  20591}{15120}$  &  $ \frac{  20591}{15120}$  &  $ \frac{ 139567}{30240}$  &  $ \frac{  668977}{30240}$  \\ \addlinespace
      \bottomrule
    \end{tabular}
  \end{center}
\end{table}

\begin{table}
  \begin{center}
    \caption{Smoothness indicators coefficients for $r=6$}
    \label{tab:IS_6}
    \begin{tabular}{cccccccc}
      \toprule
      $j$  &  $l$  &  $k=0$                           &  $k=1$                           &  $k=2$                           &  $k=3$                           &  $k=4$                           &  $k=5$                           \\ \addlinespace
      $5$  &  $5$  &  $-\frac{ 131759526}{ 3224383}$  &  $-\frac{  24044484}{ 3193217}$  &  $-\frac{  28962993}{14228092}$  &  $-\frac{  28962993}{14228092}$  &  $-\frac{  24044484}{ 3193217}$  &  $-\frac{ 131759526}{ 3224383}$  \\ \addlinespace
           &  $4$  &  $ \frac{ 295095211}{ 1259192}$  &  $ \frac{ 195395281}{ 4459947}$  &  $ \frac{  79135747}{ 6577234}$  &  $ \frac{ 251883319}{23224320}$  &  $ \frac{  26449004}{  769961}$  &  $ \frac{ 112453613}{  657635}$  \\ \addlinespace
           &  $3$  &  $-\frac{ 427867945}{  780329}$  &  $-\frac{ 146902225}{ 1415767}$  &  $-\frac{  95644735}{ 3360137}$  &  $-\frac{  61673356}{ 2721737}$  &  $-\frac{ 347085621}{ 5587817}$  &  $-\frac{ 115324682}{  395671}$  \\ \addlinespace
           &  $2$  &  $ \frac{ 497902688}{  756325}$  &  $ \frac{ 356490569}{ 2842289}$  &  $ \frac{  99590409}{ 2965471}$  &  $ \frac{ 268747951}{11612160}$  &  $ \frac{ 315600562}{ 5645537}$  &  $ \frac{ 586668707}{ 2322432}$  \\ \addlinespace
           &  $1$  &  $-\frac{ 157371280}{  384113}$  &  $-\frac{ 338120165}{ 4351341}$  &  $-\frac{  87214523}{ 4439774}$  &  $-\frac{  74146214}{ 6413969}$  &  $-\frac{ 109600459}{ 4359925}$  &  $-\frac{ 504893127}{ 4547012}$  \\ \addlinespace
           &  $0$  &  $ \frac{ 373189088}{ 7027375}$  &  $ \frac{ 105552913}{10682745}$  &  $ \frac{  30913579}{13651507}$  &  $ \frac{  15418339}{13608685}$  &  $ \frac{  30913579}{13651507}$  &  $ \frac{ 105552913}{10682745}$  \\ \addlinespace
      $4$  &  $4$  &  $ \frac{ 112453613}{  657635}$  &  $ \frac{  26449004}{  769961}$  &  $ \frac{ 251883319}{23224320}$  &  $ \frac{  79135747}{ 6577234}$  &  $ \frac{ 195395281}{ 4459947}$  &  $ \frac{ 295095211}{ 1259192}$  \\ \addlinespace
           &  $3$  &  $-\frac{ 674462631}{  691651}$  &  $-\frac{ 270758311}{ 1365867}$  &  $-\frac{1512485867}{24006092}$  &  $-\frac{1512485867}{24006092}$  &  $-\frac{ 270758311}{ 1365867}$  &  $-\frac{ 674462631}{  691651}$  \\ \addlinespace
           &  $2$  &  $ \frac{1150428332}{  508385}$  &  $ \frac{ 771393469}{ 1663855}$  &  $ \frac{  87743770}{  602579}$  &  $ \frac{ 201365679}{ 1563055}$  &  $ \frac{ 840802608}{ 2367661}$  &  $ \frac{1328498639}{  803154}$  \\ \addlinespace
           &  $1$  &  $-\frac{ 497421494}{  185427}$  &  $-\frac{2984991531}{ 5434265}$  &  $-\frac{ 370146220}{ 2226351}$  &  $-\frac{ 723607356}{ 5654437}$  &  $-\frac{ 288641753}{  912148}$  &  $-\frac{2146148426}{ 1503065}$  \\ \addlinespace
           &  $0$  &  $ \frac{ 498196769}{  609968}$  &  $ \frac{ 169505788}{ 1035915}$  &  $ \frac{  24025059}{  519766}$  &  $ \frac{ 113243845}{ 3672222}$  &  $ \frac{ 142936745}{ 2029182}$  &  $ \frac{ 453375035}{ 1449454}$  \\ \addlinespace
      $3$  &  $3$  &  $-\frac{ 115324682}{  395671}$  &  $-\frac{ 347085621}{ 5587817}$  &  $-\frac{  61673356}{ 2721737}$  &  $-\frac{  95644735}{ 3360137}$  &  $-\frac{ 146902225}{ 1415767}$  &  $-\frac{ 427867945}{  780329}$  \\ \addlinespace
           &  $2$  &  $ \frac{1328498639}{  803154}$  &  $ \frac{ 840802608}{ 2367661}$  &  $ \frac{ 201365679}{ 1563055}$  &  $ \frac{  87743770}{  602579}$  &  $ \frac{ 771393469}{ 1663855}$  &  $ \frac{1150428332}{  508385}$  \\ \addlinespace
           &  $1$  &  $-\frac{ 378281867}{   99229}$  &  $-\frac{ 479783044}{  585775}$  &  $-\frac{ 274966489}{  950662}$  &  $-\frac{ 274966489}{  950662}$  &  $-\frac{ 479783044}{  585775}$  &  $-\frac{ 378281867}{   99229}$  \\ \addlinespace
           &  $0$  &  $ \frac{2292397033}{ 1024803}$  &  $ \frac{ 471933572}{  993629}$  &  $ \frac{ 200449727}{ 1269707}$  &  $ \frac{ 586743463}{ 4237706}$  &  $ \frac{1031953342}{ 2867575}$  &  $ \frac{1406067637}{  859229}$  \\ \addlinespace
      $2$  &  $2$  &  $ \frac{ 586668707}{ 2322432}$  &  $ \frac{ 315600562}{ 5645537}$  &  $ \frac{ 268747951}{11612160}$  &  $ \frac{  99590409}{ 2965471}$  &  $ \frac{ 356490569}{ 2842289}$  &  $ \frac{ 497902668}{  756325}$  \\ \addlinespace
           &  $1$  &  $-\frac{2146148426}{ 1503065}$  &  $-\frac{ 288641753}{  912148}$  &  $-\frac{ 723607356}{ 5654437}$  &  $-\frac{ 370146220}{ 2226351}$  &  $-\frac{2984991531}{ 5434265}$  &  $-\frac{ 497421494}{  185427}$  \\ \addlinespace
           &  $0$  &  $ \frac{1406067637}{  859229}$  &  $ \frac{1031953342}{ 2867575}$  &  $ \frac{ 586743463}{ 4237706}$  &  $ \frac{ 200449727}{ 1269707}$  &  $ \frac{ 471933572}{  993629}$  &  $ \frac{2292397033}{ 1024803}$  \\ \addlinespace
      $1$  &  $1$  &  $-\frac{ 504893127}{ 4547012}$  &  $-\frac{ 109600459}{ 4359925}$  &  $-\frac{  74146214}{ 6413969}$  &  $-\frac{  87214523}{ 4439774}$  &  $-\frac{ 338120165}{ 4351341}$  &  $-\frac{ 157371280}{  384113}$  \\ \addlinespace
           &  $0$  &  $ \frac{ 453375035}{ 1449454}$  &  $ \frac{ 142936745}{ 2029182}$  &  $ \frac{ 113243845}{ 3672222}$  &  $ \frac{  24025059}{  519766}$  &  $ \frac{ 169505788}{ 1035915}$  &  $ \frac{ 498196769}{  609968}$  \\ \addlinespace
      $0$  &  $0$  &  $ \frac{ 105552913}{10682745}$  &  $ \frac{  30913579}{13651507}$  &  $ \frac{  15418339}{13608685}$  &  $ \frac{  30913579}{13651507}$  &  $ \frac{ 105552913}{10682745}$  &  $ \frac{ 373189088}{ 7027375}$  \\ \addlinespace
      \bottomrule
    \end{tabular}
  \end{center}
\end{table}

\begin{table}
  \begin{center}
    \caption{Smoothness indicators coefficients for $r=7$}
    \label{tab:IS_7}
    \begin{tabular}{ccccccccc}
      \toprule
      $j$  &  $l$  &  $k=0$                          &  $k=1$                           &  $k=2$                            &  $k=3$                           &  $k=4$                            &  $k=5$                            &  $k=6$                          \\ \addlinespace
      $6$  &  $6$  &  $ \frac{  65647731}{ 691205}$  &  $ \frac{  43003346}{ 2612319}$  &  $ \frac{   77150072}{21955151}$  &  $ \frac{  29187600}{17822477}$  &  $ \frac{   77150072}{21955151}$  &  $ \frac{  43003346}{ 2612319} $  &  $ \frac{  65647731}{ 691205}$  \\ \addlinespace
           &  $5$  &  $-\frac{ 418267211}{ 655432}$  &  $-\frac{1157045253}{10370330}$  &  $-\frac{  205305705}{ 8465339}$  &  $-\frac{  31210580}{ 2807109}$  &  $-\frac{   98152843}{ 4687720}$  &  $-\frac{ 265505701}{ 2998139} $  &  $-\frac{ 299800985}{ 620702}$  \\ \addlinespace
           &  $4$  &  $ \frac{2375865880}{1312047}$  &  $ \frac{ 200564827}{  628331}$  &  $ \frac{  164871587}{ 2347023}$  &  $ \frac{  78098218}{ 2511469}$  &  $ \frac{  143992467}{ 2811164}$  &  $ \frac{ 265135851}{ 1336964} $  &  $ \frac{ 412399715}{ 395812}$  \\ \addlinespace
           &  $3$  &  $-\frac{ 882134137}{ 316505}$  &  $-\frac{ 219042731}{  442919}$  &  $-\frac{  337645273}{ 3091776}$  &  $-\frac{  77947404}{ 1703711}$  &  $-\frac{  177311125}{ 2691566}$  &  $-\frac{ 246865952}{ 1040433} $  &  $-\frac{ 219701291}{ 180490}$  \\ \addlinespace
           &  $2$  &  $ \frac{1025357155}{ 415733}$  &  $ \frac{ 451414666}{ 1028589}$  &  $ \frac{  305770890}{ 3186613}$  &  $ \frac{  97747719}{ 2624408}$  &  $ \frac{   85769455}{ 1822342}$  &  $ \frac{1743860591}{10881504} $  &  $ \frac{ 562957181}{ 694753}$  \\ \addlinespace
           &  $1$  &  $-\frac{ 842151863}{ 702281}$  &  $-\frac{ 258813979}{ 1219012}$  &  $-\frac{  303410983}{ 6736159}$  &  $-\frac{  85952276}{ 5412389}$  &  $-\frac{   86513123}{ 4872070}$  &  $-\frac{ 483420287}{ 8336284} $  &  $-\frac{ 484093752}{1664533}$  \\ \addlinespace
           &  $0$  &  $ \frac{ 307570060}{2438487}$  &  $ \frac{ 118739219}{ 5409702}$  &  $ \frac{   76695443}{17458022}$  &  $ \frac{  20823809}{15031645}$  &  $ \frac{   20823809}{15031645}$  &  $ \frac{  76695443}{17458022} $  &  $ \frac{ 118739219}{5409702}$  \\ \addlinespace
      $5$  &  $5$  &  $-\frac{ 299800985}{ 620702}$  &  $-\frac{ 265505701}{ 2998139}$  &  $-\frac{   98152843}{ 4687720}$  &  $-\frac{  31210580}{ 2807109}$  &  $-\frac{  205305705}{ 8465339}$  &  $-\frac{1157045253}{10370330} $  &  $-\frac{ 418267211}{ 655432}$  \\ \addlinespace
           &  $4$  &  $ \frac{ 803154527}{ 248375}$  &  $ \frac{1029357835}{ 1723277}$  &  $ \frac{  154914521}{ 1081252}$  &  $ \frac{ 143433946}{ 1930931}$  &  $ \frac{  154914521}{ 1081252}$  &  $ \frac{1029357835}{ 1723277} $  &  $ \frac{ 803154527}{ 248375}$  \\ \addlinespace
           &  $3$  &  $-\frac{ 550697211}{  60310}$  &  $-\frac{ 448069659}{  263978}$  &  $-\frac{ 1002866209}{ 2445347}$  &  $-\frac{7192946466}{35277791}$  &  $-\frac{  251896262}{  725959}$  &  $-\frac{ 577579349}{  433921} $  &  $-\frac{1068783425}{ 153683}$  \\ \addlinespace
           &  $2$  &  $ \frac{2854637563}{ 204507}$  &  $ \frac{6598378479}{ 2533904}$  &  $-\frac{  470895955}{  874781}$  &  $ \frac{ 212799192}{  725717}$  &  $ \frac{13260333719}{30064515}$  &  $ \frac{ 498890606}{  314761} $  &  $ \frac{2369766527}{ 292389}$  \\ \addlinespace
           &  $1$  &  $-\frac{2727583905}{ 223057}$  &  $-\frac{2876116249}{ 1263255}$  &  $ \frac{  337717185}{  538487}$  &  $-\frac{ 735436149}{ 3170423}$  &  $-\frac{  393831298}{ 1266551}$  &  $-\frac{ 185662673}{  174204} $  &  $-\frac{3101495154}{ 576017}$  \\ \addlinespace
           &  $0$  &  $ \frac{1267010831}{ 433225}$  &  $ \frac{ 151821033}{  282817}$  &  $ \frac{  266980515}{ 2188712}$  &  $ \frac{ 151133283}{ 3169976}$  &  $ \frac{  309673793}{ 5357421}$  &  $ \frac{ 393580372}{ 2049353} $  &  $ \frac{ 368117849}{ 381597}$  \\ \addlinespace
      $4$  &  $4$  &  $ \frac{ 412399715}{ 395812}$  &  $ \frac{ 265135851}{ 1336964}$  &  $ \frac{  143992467}{ 2811164}$  &  $ \frac{  78098218}{ 2511469}$  &  $ \frac{  164871587}{ 2347023}$  &  $ \frac{ 200564827}{  628331} $  &  $ \frac{2375865880}{1312047}$  \\ \addlinespace
           &  $3$  &  $-\frac{1068783425}{ 153683}$  &  $-\frac{ 577579349}{  433921}$  &  $-\frac{  251896262}{  725959}$  &  $-\frac{7192946466}{35277791}$  &  $-\frac{ 1002866209}{ 2445347}$  &  $-\frac{ 448069659}{  263978} $  &  $-\frac{ 550697211}{  60310}$  \\ \addlinespace
           &  $2$  &  $ \frac{3315206316}{ 169489}$  &  $ \frac{ 656116894}{  174649}$  &  $ \frac{  750365573}{  765885}$  &  $ \frac{1046376941}{ 1911720}$  &  $ \frac{  750365573}{  765885}$  &  $ \frac{ 656116894}{  174649} $  &  $ \frac{3315206316}{ 169489}$  \\ \addlinespace
           &  $1$  &  $-\frac{ 485497721}{  16325}$  &  $-\frac{ 952714155}{  166894}$  &  $-\frac{  631316405}{  429286}$  &  $-\frac{ 478256390}{  624157}$  &  $-\frac{  660635886}{  538753}$  &  $-\frac{1397796418}{  314477} $  &  $-\frac{1833856939}{  80705}$  \\ \addlinespace
           &  $0$  &  $ \frac{2398154453}{ 185516}$  &  $ \frac{3295939303}{ 1339169}$  &  $ \frac{  576629617}{  938378}$  &  $ \frac{ 330842346}{ 1128355}$  &  $ \frac{  787491691}{ 1852394}$  &  $ \frac{1142129285}{  768659} $  &  $ \frac{ 384888217}{  51123}$  \\ \addlinespace
      $3$  &  $3$  &  $-\frac{ 219701291}{ 180490}$  &  $-\frac{ 246865952}{ 1040433}$  &  $-\frac{  177311125}{ 2691566}$  &  $-\frac{  77947404}{ 1703711}$  &  $-\frac{  337645273}{ 3091776}$  &  $-\frac{ 219042731}{  442919} $  &  $-\frac{ 882134137}{ 316505}$  \\ \addlinespace
           &  $2$  &  $ \frac{2369766527}{ 292389}$  &  $ \frac{ 498890606}{  314761}$  &  $ \frac{13260333719}{30064515}$  &  $ \frac{ 212799192}{  725717}$  &  $ \frac{  337717185}{  538487}$  &  $ \frac{6598378479}{ 2533904} $  &  $ \frac{2854637563}{ 204507}$  \\ \addlinespace
           &  $1$  &  $-\frac{1833856939}{  80705}$  &  $-\frac{1397796418}{  314477}$  &  $-\frac{  660635886}{  538753}$  &  $-\frac{ 478256390}{  624157}$  &  $-\frac{  631316405}{  429286}$  &  $-\frac{ 952714155}{  166894} $  &  $ \frac{-485497721}{  16325}$  \\ \addlinespace
           &  $0$  &  $ \frac{2558389867}{ 148729}$  &  $ \frac{ 353679247}{  105637}$  &  $ \frac{  449371687}{  498274}$  &  $ \frac{1393876129}{ 2686891}$  &  $ \frac{  449371687}{  498274}$  &  $ \frac{ 353679247}{  105637} $  &  $ \frac{2558389867}{ 148729}$  \\ \addlinespace
      $2$  &  $2$  &  $ \frac{ 562957181}{ 694753}$  &  $ \frac{1743860591}{10881504}$  &  $ \frac{   85769455}{ 1822342}$  &  $ \frac{  97747719}{ 2624408}$  &  $ \frac{  305770890}{ 3186613}$  &  $ \frac{ 451414666}{ 1028589} $  &  $ \frac{1025357155}{ 415733}$  \\ \addlinespace
           &  $1$  &  $-\frac{3101495154}{ 576017}$  &  $-\frac{ 185662673}{  174204}$  &  $-\frac{  393831298}{ 1266551}$  &  $-\frac{ 735436149}{ 3170423}$  &  $-\frac{  470895955}{  874781}$  &  $-\frac{2876116249}{ 1263255} $  &  $-\frac{2727583905}{ 223057}$  \\ \addlinespace
           &  $0$  &  $ \frac{ 384888217}{  51123}$  &  $ \frac{1142129285}{  768659}$  &  $ \frac{  787491691}{ 1852394}$  &  $ \frac{ 330842346}{ 1128355}$  &  $ \frac{  576629617}{  938378}$  &  $ \frac{3295939303}{ 1339169} $  &  $ \frac{2398154453}{ 185516}$  \\ \addlinespace
      $1$  &  $1$  &  $-\frac{ 484093752}{1664533}$  &  $-\frac{ 483420287}{ 8336284}$  &  $-\frac{   86513123}{ 4872070}$  &  $-\frac{  85952276}{ 5412389}$  &  $-\frac{  303410983}{ 6736159}$  &  $-\frac{ 258813979}{ 1219012} $  &  $-\frac{ 842151863}{ 702281}$  \\ \addlinespace
           &  $0$  &  $ \frac{ 368117849}{ 381597}$  &  $ \frac{ 393580372}{ 2049353}$  &  $ \frac{  309673793}{ 5357421}$  &  $ \frac{ 151133283}{ 3169976}$  &  $ \frac{  266980515}{ 2188712}$  &  $ \frac{ 151821033}{  282817} $  &  $ \frac{1267010831}{ 433225}$  \\ \addlinespace
      $0$  &  $0$  &  $ \frac{ 118739219}{5409702}$  &  $ \frac{  76695443}{17458022}$  &  $ \frac{   20823809}{15031645}$  &  $ \frac{  20823809}{15031645}$  &  $ \frac{   76695443}{17458022}$  &  $ \frac{ 118739219}{ 5409702} $  &  $ \frac{ 307570060}{2438487}$  \\ \addlinespace
      \bottomrule
    \end{tabular}
  \end{center}
\end{table}

\begin{table}
  \begin{center}
    \caption{Smoothness indicators coefficients for $r=8$}
    \label{tab:IS_8}
    \begin{tabular}{ccccccccc}
      \toprule
      $j$  &  $l$  &  $k=0$                           &  $k=1$                            &  $k=2$                           &  $k=3$                           &  $k=4$                           &  $k=5$                            &  $k=6$                            $k=7$                           \\ \addlinespace
      $7$  &  $7$  &  $-\frac{  167817292}{ 753123}$  &  $-\frac{  115902052}{ 3120403}$  &  $-\frac{  44754099}{ 6344939}$  &  $-\frac{  21873377}{10764442}$  &  $-\frac{  21873377}{10764442}$  &  $-\frac{  44754099}{ 6344939} $  &  $-\frac{  115902052}{ 3120403}$  $-\frac{  167817292}{ 753123}$  \\ \addlinespace
           &  $6$  &  $ \frac{ 1730988313}{1007913}$  &  $ \frac{  362054965}{ 1257877}$  &  $ \frac{ 112959697}{ 2041527}$  &  $ \frac{ 141070919}{ 8713488}$  &  $ \frac{  69576681}{ 4589819}$  &  $ \frac{ 129766396}{ 2754429} $  &  $ \frac{  513945629}{ 2216079}$  $ \frac{ 1606637628}{1200199}$  \\ \addlinespace
           &  $5$  &  $-\frac{ 6701525420}{1169941}$  &  $-\frac{12689783695}{13147542}$  &  $-\frac{ 179578697}{  957716}$  &  $-\frac{ 103772319}{ 1881526}$  &  $-\frac{ 398300903}{ 8329274}$  &  $-\frac{ 270604594}{ 2024029} $  &  $-\frac{  724803819}{ 1163906}$  $-\frac{ 2034860005}{ 580787}$  \\ \addlinespace
           &  $4$  &  $ \frac{ 1191775685}{ 110969}$  &  $ \frac{  847040497}{  465789}$  &  $ \frac{ 610690841}{ 1715763}$  &  $ \frac{ 148443265}{ 1427854}$  &  $ \frac{ 200885069}{ 2431769}$  &  $ \frac{ 430661427}{ 2058148} $  &  $ \frac{  779780282}{  835427}$  $ \frac{ 1168472761}{ 226223}$  \\ \addlinespace
           &  $3$  &  $-\frac{ 1384199219}{ 112909}$  &  $-\frac{ 2546573797}{ 1222381}$  &  $-\frac{ 559020701}{ 1367726}$  &  $-\frac{ 134406712}{ 1150037}$  &  $ \frac{ -65777185}{  779772}$  &  $-\frac{ 114044024}{  583601} $  &  $-\frac{ 1403389204}{ 1662883}$  $-\frac{ 1774088813}{ 383858}$  \\ \addlinespace
           &  $2$  &  $ \frac{ 1512171950}{ 176773}$  &  $ \frac{  234353207}{  161088}$  &  $ \frac{ 205707004}{  724801}$  &  $ \frac{ 268720507}{ 3437558}$  &  $ \frac{ 108380895}{ 2128121}$  &  $ \frac{ 401318077}{ 3678649} $  &  $ \frac{  281051417}{  610454}$  $ \frac{ 4932843539}{1968706}$  \\ \addlinespace
           &  $1$  &  $-\frac{ 1353623375}{ 398213}$  &  $-\frac{  464902845}{  808102}$  &  $-\frac{ 655235691}{ 5945464}$  &  $-\frac{  63831289}{ 2220847}$  &  $-\frac{  39287533}{ 2331609}$  &  $-\frac{ 141509768}{ 4191221} $  &  $-\frac{  255613952}{ 1821943}$  $-\frac{  508083143}{ 667663}$  \\ \addlinespace
           &  $0$  &  $ \frac{  561955582}{1878967}$  &  $ \frac{  151567467}{ 3038449}$  &  $ \frac{  79932001}{ 8679360}$  &  $ \frac{  35501666}{15868715}$  &  $ \frac{  12431715}{10534253}$  &  $ \frac{  35501666}{15868715} $  &  $ \frac{   79932001}{ 8679360}$  $ \frac{  151567467}{3038449}$  \\ \addlinespace
      $6$  &  $6$  &  $ \frac{ 1606637628}{1200199}$  &  $ \frac{  513945629}{ 2216079}$  &  $ \frac{ 129766396}{ 2754429}$  &  $ \frac{  69576681}{ 4589819}$  &  $ \frac{ 141070919}{ 8713488}$  &  $ \frac{ 112959697}{ 2041527} $  &  $ \frac{  362054965}{ 1257877}$  $ \frac{ 1730988313}{1007913}$  \\ \addlinespace
           &  $5$  &  $-\frac{ 8115803171}{ 788565}$  &  $-\frac{  850151296}{  474539}$  &  $-\frac{ 649079478}{ 1764673}$  &  $-\frac{ 386869123}{ 3236626}$  &  $-\frac{ 386869123}{ 3236626}$  &  $-\frac{ 649079478}{ 1764673} $  &  $-\frac{  850151296}{  474539}$  $-\frac{ 8115803171}{ 788565}$  \\ \addlinespace
           &  $4$  &  $ \frac{ 3436464517}{ 100426}$  &  $ \frac{ 4037906091}{  674921}$  &  $ \frac{ 324962019}{  262375}$  &  $ \frac{ 422372886}{ 1050263}$  &  $ \frac{ 693020919}{ 1859333}$  &  $ \frac{ 501175243}{  482649} $  &  $ \frac{ 2674480859}{  557634}$  $ \frac{ 4477231643}{ 166549}$  \\ \addlinespace
           &  $3$  &  $-\frac{ 2650855638}{  41489}$  &  $-\frac{ 3161084857}{  282001}$  &  $-\frac{ 699001320}{  299911}$  &  $-\frac{ 311872754}{  417681}$  &  $-\frac{ 543724576}{  855585}$  &  $-\frac{ 694807489}{  429931} $  &  $-\frac{ 1907782262}{  266123}$  $-\frac{ 3946887082}{  99757}$  \\ \addlinespace
           &  $2$  &  $ \frac{ 2653665219}{  36590}$  &  $ \frac{ 3431063476}{  269267}$  &  $ \frac{ 554363127}{  209623}$  &  $ \frac{3507914221}{ 4258272}$  &  $ \frac{ 379000051}{  592915}$  &  $ \frac{ 559782185}{  373076} $  &  $ \frac{ 2349626332}{  363399}$  $ \frac{12211598186}{ 345407}$  \\ \addlinespace
           &  $1$  &  $-\frac{ 6783346413}{ 135128}$  &  $-\frac{ 2039339988}{  231781}$  &  $-\frac{1032899132}{  571995}$  &  $-\frac{ 234383777}{  435589}$  &  $-\frac{ 540913157}{ 1426197}$  &  $-\frac{ 493139495}{  592214} $  &  $-\frac{  686664647}{  195106}$  $-\frac{ 1307164757}{  68276}$  \\ \addlinespace
           &  $0$  &  $ \frac{ 5230798390}{ 531001}$  &  $ \frac{  960477863}{  562021}$  &  $ \frac{ 403846727}{ 1180353}$  &  $ \frac{ 204776677}{ 2133916}$  &  $ \frac{ 358821925}{ 5833643}$  &  $ \frac{ 629957047}{ 4917482} $  &  $ \frac{   48179335}{   90019}$  $ \frac{ 1285415788}{ 442547}$  \\ \addlinespace
      $5$  &  $5$  &  $-\frac{ 2034860005}{ 580787}$  &  $-\frac{  724803819}{ 1163906}$  &  $-\frac{ 270604594}{ 2024029}$  &  $-\frac{ 398300903}{ 8329274}$  &  $-\frac{ 103772319}{ 1881526}$  &  $-\frac{ 179578697}{  957716} $  &  $-\frac{12689783695}{13147542}$  $-\frac{ 6701525420}{1169941}$  \\ \addlinespace
           &  $4$  &  $ \frac{ 4477231643}{ 166549}$  &  $ \frac{ 2674480859}{  557634}$  &  $ \frac{ 501175243}{  482649}$  &  $ \frac{ 693020919}{ 1859333}$  &  $ \frac{ 422372886}{ 1050263}$  &  $ \frac{ 324962019}{  262375} $  &  $ \frac{ 4037906091}{  674921}$  $ \frac{ 3436464517}{ 100426}$  \\ \addlinespace
           &  $3$  &  $-\frac{ 9679034365}{ 108568}$  &  $-\frac{ 2029186932}{  127189}$  &  $-\frac{8089971196}{ 2329825}$  &  $-\frac{  84200903}{   68084}$  &  $-\frac{  84200903}{   68084}$  &  $-\frac{8089971196}{ 2329825} $  &  $-\frac{ 2029186932}{  127189}$  $-\frac{ 9679034365}{ 108568}$  \\ \addlinespace
           &  $2$  &  $ \frac{ 2354499851}{  14191}$  &  $ \frac{ 4919628784}{  165435}$  &  $ \frac{1056954815}{  163259}$  &  $ \frac{1441974426}{  638695}$  &  $ \frac{ 520921076}{  250961}$  &  $ \frac{4782113096}{  891381} $  &  $ \frac{ 1773946113}{   74654}$  $ \frac{ 7936751861}{  60613}$  \\ \addlinespace
           &  $1$  &  $-\frac{ 4461330800}{  23793}$  &  $-\frac{ 2609137409}{   77728}$  &  $-\frac{1300201595}{  179203}$  &  $-\frac{ 809595667}{  331812}$  &  $-\frac{1022198433}{  498364}$  &  $-\frac{ 799191084}{  161641} $  &  $-\frac{ 1674462641}{   78375}$  $-\frac{ 2087501693}{  17871}$  \\ \addlinespace
           &  $0$  &  $ \frac{ 3382169379}{  52433}$  &  $ \frac{ 4802121175}{  418404}$  &  $ \frac{5814856284}{ 2387539}$  &  $ \frac{ 360251831}{  463656}$  &  $ \frac{ 789836795}{ 1323609}$  &  $ \frac{ 257028097}{  188691} $  &  $ \frac{ 3171324093}{  546871}$  $ \frac{ 5633451919}{ 178362}$  \\ \addlinespace
      $4$  &  $4$  &  $ \frac{ 1168472761}{ 226223}$  &  $ \frac{  779780282}{  835427}$  &  $ \frac{ 430661427}{ 2058148}$  &  $ \frac{ 200885069}{ 2431769}$  &  $ \frac{ 148443265}{ 1427854}$  &  $ \frac{ 610690841}{ 1715763} $  &  $ \frac{  847040497}{  465789}$  $ \frac{ 1191775685}{ 110969}$  \\ \addlinespace
           &  $3$  &  $-\frac{ 3946887082}{  99757}$  &  $-\frac{ 1907782262}{  266123}$  &  $-\frac{ 694807489}{  429931}$  &  $-\frac{ 543724576}{  855585}$  &  $-\frac{ 311872754}{  417681}$  &  $-\frac{ 699001320}{  299911} $  &  $-\frac{ 3161084857}{  282001}$  $-\frac{ 2650855638}{  41489}$  \\ \addlinespace
           &  $2$  &  $ \frac{ 7936751861}{  60613}$  &  $ \frac{ 1773946113}{   74654}$  &  $ \frac{4782113096}{  891381}$  &  $ \frac{ 520921076}{  250961}$  &  $ \frac{1441974426}{  638695}$  &  $ \frac{1056954815}{  163259} $  &  $ \frac{ 4919628784}{  165435}$  $ \frac{ 2354499851}{  14191}$  \\ \addlinespace
           &  $1$  &  $-\frac{10453320754}{  43009}$  &  $-\frac{ 5435379710}{  123283}$  &  $-\frac{ 823868037}{   83150}$  &  $-\frac{1353219397}{  363901}$  &  $-\frac{1353219397}{  363901}$  &  $-\frac{ 823868037}{   83150} $  &  $-\frac{ 5435379710}{  123283}$  $-\frac{10453320754}{  43009}$  \\ \addlinespace
           &  $0$  &  $ \frac{ 5383551615}{  39332}$  &  $ \frac{ 3485486425}{  140912}$  &  $ \frac{7318753887}{ 1334341}$  &  $ \frac{ 755335167}{  384508}$  &  $ \frac{1014659207}{  563712}$  &  $ \frac{1492354285}{  329872} $  &  $ \frac{ 3163565270}{  160241}$  $ \frac{15685259234}{ 144989}$  \\ \addlinespace
      $3$  &  $3$  &  $-\frac{ 1774088813}{ 383858}$  &  $-\frac{ 1403389204}{ 1662883}$  &  $-\frac{ 114044024}{  583601}$  &  $-\frac{  65777185}{  779772}$  &  $-\frac{ 134406712}{ 1150037}$  &  $-\frac{ 559020701}{ 1367726} $  &  $-\frac{ 2546573797}{ 1222381}$  $-\frac{ 1384199219}{ 112909}$  \\ \addlinespace
           &  $2$  &  $ \frac{12211598186}{ 345407}$  &  $ \frac{ 2349626332}{  363399}$  &  $ \frac{ 559782185}{  373076}$  &  $ \frac{ 379000051}{  592915}$  &  $ \frac{3507914221}{ 4258272}$  &  $ \frac{ 554363127}{  209623} $  &  $ \frac{ 3431063476}{  269267}$  $ \frac{ 2653665219}{  36590}$  \\ \addlinespace
           &  $1$  &  $-\frac{ 2087501693}{  17871}$  &  $-\frac{ 1674462641}{   78375}$  &  $-\frac{ 799191084}{  161641}$  &  $-\frac{1022198433}{  498364}$  &  $-\frac{ 809595667}{  331812}$  &  $-\frac{1300201595}{  179203} $  &  $-\frac{ 2609137409}{   77728}$  $-\frac{ 4461330800}{  23793}$  \\ \addlinespace
           &  $0$  &  $ \frac{15685259234}{ 144989}$  &  $ \frac{ 3163565270}{  160241}$  &  $ \frac{1492354285}{  329872}$  &  $ \frac{1014659207}{  563712}$  &  $ \frac{ 755335167}{  384508}$  &  $ \frac{7318753887}{ 1334341} $  &  $ \frac{ 3485486425}{  140912}$  $ \frac{ 5383551615}{  39332}$  \\ \addlinespace
      $2$  &  $2$  &  $ \frac{ 4932843539}{1968706}$  &  $ \frac{  281051417}{  610454}$  &  $ \frac{ 401318077}{ 3678649}$  &  $ \frac{ 108380895}{ 2128121}$  &  $ \frac{ 268720507}{ 3437558}$  &  $ \frac{ 205707004}{  724801} $  &  $ \frac{  234353207}{  161088}$  $ \frac{ 1512171950}{ 176773}$  \\ \addlinespace
           &  $1$  &  $-\frac{ 1307164757}{  68276}$  &  $-\frac{  686664647}{  195106}$  &  $-\frac{ 493139495}{  592214}$  &  $-\frac{ 540913157}{ 1426197}$  &  $-\frac{ 234383777}{  435589}$  &  $-\frac{1032899132}{  571995} $  &  $-\frac{ 2039339988}{  231781}$  $-\frac{ 6783346413}{ 135128}$  \\ \addlinespace
           &  $0$  &  $ \frac{ 5633451919}{ 178362}$  &  $ \frac{ 3171324093}{  546871}$  &  $ \frac{ 257028097}{  188691}$  &  $ \frac{ 789836795}{ 1323609}$  &  $ \frac{ 360251831}{  463656}$  &  $ \frac{5814856284}{ 2387539} $  &  $ \frac{ 4802121175}{  418404}$  $ \frac{ 3382169379}{  52433}$  \\ \addlinespace
      $1$  &  $1$  &  $-\frac{  508083143}{ 667663}$  &  $-\frac{  255613952}{ 1821943}$  &  $-\frac{ 141509768}{ 4191221}$  &  $-\frac{  39287533}{ 2331609}$  &  $-\frac{  63831289}{ 2220847}$  &  $-\frac{ 655235691}{ 5945464} $  &  $-\frac{  464902845}{  808102}$  $-\frac{ 1353623375}{ 398213}$  \\ \addlinespace
           &  $0$  &  $ \frac{ 1285415788}{ 442547}$  &  $ \frac{   48179335}{   90019}$  &  $ \frac{ 629957047}{ 4917482}$  &  $ \frac{ 358821925}{ 5833643}$  &  $ \frac{ 204776677}{ 2133916}$  &  $ \frac{ 403846727}{ 1180353} $  &  $ \frac{  960477863}{  562021}$  $ \frac{ 5230798390}{ 531001}$  \\ \addlinespace
      $0$  &  $0$  &  $ \frac{  151567467}{3038449}$  &  $ \frac{   79932001}{ 8679360}$  &  $ \frac{  35501666}{15868715}$  &  $ \frac{  12431715}{10534253}$  &  $ \frac{  35501666}{15868715}$  &  $ \frac{  79932001}{ 8679360} $  &  $ \frac{  151567467}{ 3038449}$  $ \frac{  561955582}{1878967}$  \\ \addlinespace
      \bottomrule
    \end{tabular}
  \end{center}
\end{table}

\begin{table}
  \begin{center}
    \caption{Smoothness indicators coefficients for $r=9$}
    \label{tab:IS_9}
    \begin{tabular}{cccccccccc}
      \toprule
      $j$  &  $l$  &  $k=0$                            &  $k=1$                            &  $k=2$                            &  $k=3$                           &  $k=4$                           &  $k=5$                           &  $k=6$                            $k=7$                            $k=8$                            \\ \addlinespace
      $8$  &  $8$  &  $ \frac{   380112881}{ 721737}$  &  $ \frac{   192493416}{2253847}$  &  $ \frac{  265338548}{17495633}$  &  $ \frac{  33222819}{ 9738314}$  &  $ \frac{  21701959}{12951510}$  &  $ \frac{  33222819}{ 9738314}$  &  $ \frac{  265338548}{17495633}$  $ \frac{   192493416}{2253847}$  $ \frac{   380112881}{ 721737}$  \\ \addlinespace
           &  $7$  &  $-\frac{  1382011106}{ 301683}$  &  $-\frac{   433682386}{ 581703}$  &  $-\frac{  173397370}{ 1299717}$  &  $-\frac{ 417266048}{13678797}$  &  $-\frac{  42281552}{ 2841263}$  &  $-\frac{ 243832589}{ 8827552}$  &  $-\frac{  264553111}{ 2333462}$  $-\frac{   759205271}{1245236}$  $-\frac{  1039356853}{ 284187}$  \\ \addlinespace
           &  $6$  &  $ \frac{  7116193241}{ 405236}$  &  $ \frac{  1206026846}{ 420471}$  &  $ \frac{  688214053}{ 1331147}$  &  $ \frac{ 184615935}{ 1542601}$  &  $ \frac{ 179193514}{ 3127239}$  &  $ \frac{  83373698}{  861333}$  &  $ \frac{  383212815}{ 1037536}$  $ \frac{  2064497172}{1078127}$  $ \frac{  2160095091}{ 191558}$  \\ \addlinespace
           &  $5$  &  $-\frac{ 12858081715}{ 331389}$  &  $-\frac{  1432715713}{ 225284}$  &  $ \frac{  -61463934}{   53285}$  &  $-\frac{ 709458479}{ 2638758}$  &  $-\frac{ 247486780}{ 1982753}$  &  $-\frac{ 135160981}{  704829}$  &  $-\frac{  427576737}{  623480}$  $-\frac{  1275601375}{ 368936}$  $-\frac{ 16400242834}{ 815393}$  \\ \addlinespace
           &  $4$  &  $ \frac{  8028408627}{ 148285}$  &  $ \frac{  2318146475}{ 260443}$  &  $ \frac{  306856831}{  189251}$  &  $ \frac{ 348597468}{  922523}$  &  $ \frac{ 323192477}{ 1923068}$  &  $ \frac{  85841095}{  365273}$  &  $ \frac{  537364516}{  676097}$  $ \frac{  1990119523}{ 506979}$  $ \frac{  1211629703}{  53483}$  \\ \addlinespace
           &  $3$  &  $-\frac{  6519672839}{ 133134}$  &  $-\frac{   351689199}{  43600}$  &  $-\frac{  289784372}{  196989}$  &  $-\frac{ 597649141}{ 1759029}$  &  $-\frac{  81991005}{  573014}$  &  $-\frac{ 185363617}{ 1015232}$  &  $-\frac{  823497572}{ 1397105}$  $-\frac{   557744521}{ 194407}$  $-\frac{   800361473}{  48582}$  \\ \addlinespace
           &  $2$  &  $ \frac{  1051885279}{  37394}$  &  $ \frac{  1919279425}{ 414313}$  &  $ \frac{  277579576}{  329887}$  &  $ \frac{ 103779883}{  544689}$  &  $ \frac{  48978927}{  651442}$  &  $ \frac{  67366110}{  766169}$  &  $ \frac{  329649921}{ 1205744}$  $ \frac{  1414733955}{1073627}$  $ \frac{  2005851423}{ 265880}$  \\ \addlinespace
           &  $1$  &  $-\frac{  1291706883}{ 137012}$  &  $-\frac{  1605498941}{1038640}$  &  $-\frac{  699447262}{ 2521667}$  &  $-\frac{ 186193587}{ 3061888}$  &  $-\frac{ 471882251}{21169910}$  &  $-\frac{  28933143}{ 1204235}$  &  $-\frac{  178701734}{ 2462661}$  $-\frac{   296572045}{ 853161}$  $-\frac{   989259649}{ 497859}$  \\ \addlinespace
           &  $0$  &  $ \frac{   191906863}{ 270061}$  &  $ \frac{    23000337}{ 199768}$  &  $ \frac{   36409563}{ 1806520}$  &  $ \frac{  14225607}{ 3370285}$  &  $ \frac{  25595175}{17925332}$  &  $ \frac{  25595175}{17925332}$  &  $ \frac{   14225607}{ 3370285}$  $ \frac{    36409563}{1806520}$  $ \frac{    23000337}{ 199768}$  \\ \addlinespace
      $7$  &  $7$  &  $-\frac{  1039356853}{ 284187}$  &  $-\frac{   759205271}{1245236}$  &  $-\frac{  264553111}{ 2333462}$  &  $-\frac{ 243832589}{ 8827552}$  &  $-\frac{  42281552}{ 2841263}$  &  $-\frac{ 417266048}{13678797}$  &  $-\frac{  173397370}{ 1299717}$  $-\frac{   433682386}{ 581703}$  $-\frac{  1382011106}{ 301683}$  \\ \addlinespace
           &  $6$  &  $ \frac{   962141663}{  30298}$  &  $ \frac{  1632642660}{ 307433}$  &  $ \frac{  127754174}{  128481}$  &  $ \frac{ 544135101}{ 2215768}$  &  $ \frac{ 145478651}{ 1112277}$  &  $ \frac{ 544135101}{ 2215768}$  &  $ \frac{  127754174}{  128481}$  $ \frac{  1632642660}{ 307433}$  $ \frac{   962141663}{  30298}$  \\ \addlinespace
           &  $5$  &  $-\frac{  7097325924}{  58429}$  &  $-\frac{   684405583}{  33590}$  &  $-\frac{ 2367490577}{  616772}$  &  $-\frac{ 931274285}{  973468}$  &  $-\frac{ 379006664}{  761061}$  &  $-\frac{ 767075415}{  896921}$  &  $-\frac{  676787627}{  209575}$  $-\frac{  2519869819}{ 151381}$  $-\frac{  7469836609}{  76401}$  \\ \addlinespace
           &  $4$  &  $ \frac{ 13666821827}{  51060}$  &  $ \frac{  2631734550}{  58459}$  &  $ \frac{ 5241495620}{  615127}$  &  $ \frac{1034492709}{  485618}$  &  $ \frac{ 889068808}{  829823}$  &  $ \frac{ 654146656}{  388723}$  &  $ \frac{ 1268411423}{  212206}$  $ \frac{  2675355119}{  89174}$  $ \frac{  8534140303}{  48995}$  \\ \addlinespace
           &  $3$  &  $-\frac{ 14121568547}{  37942}$  &  $-\frac{  2463944763}{  39286}$  &  $-\frac{10107954583}{  849559}$  &  $-\frac{ 787874261}{  266082}$  &  $-\frac{ 659953893}{  463955}$  &  $-\frac{6738238495}{ 3291754}$  &  $ \frac{-2267814051}{  328385}$  $-\frac{  5136703769}{ 151046}$  $-\frac{ 29831101642}{ 152201}$  \\ \addlinespace
           &  $2$  &  $ \frac{ 10624327325}{  31707}$  &  $ \frac{ 17759778441}{ 314408}$  &  $ \frac{ 2363787227}{  220958}$  &  $ \frac{5590654438}{ 2129495}$  &  $ \frac{1066785823}{  895146}$  &  $ \frac{ 855538459}{  542278}$  &  $ \frac{  982680142}{  192447}$  $ \frac{  1696424402}{  68349}$  $ \frac{  6203677189}{  43561}$  \\ \addlinespace
           &  $1$  &  $-\frac{  2523726139}{  13197}$  &  $-\frac{  6349489117}{ 197436}$  &  $-\frac{ 1651888798}{  273307}$  &  $-\frac{ 522065981}{  360998}$  &  $-\frac{ 257255959}{  418532}$  &  $-\frac{ 491966393}{  653081}$  &  $-\frac{ 1883344606}{  797417}$  $-\frac{  1486183058}{ 130527}$  $-\frac{  5910597075}{  90694}$  \\ \addlinespace
           &  $0$  &  $ \frac{  2789709824}{  87891}$  &  $ \frac{  8788336457}{1659246}$  &  $ \frac{  526012837}{  537300}$  &  $ \frac{ 308180301}{ 1366333}$  &  $ \frac{ 206821378}{ 2319277}$  &  $ \frac{ 193935861}{ 1901234}$  &  $ \frac{  267692197}{  856297}$  $ \frac{   550334507}{ 366830}$  $ \frac{  1207396129}{ 140764}$  \\ \addlinespace
      $6$  &  $6$  &  $ \frac{  2160095091}{ 191558}$  &  $ \frac{  2064497172}{1078127}$  &  $ \frac{  383212815}{ 1037536}$  &  $ \frac{  83373698}{  861333}$  &  $ \frac{ 179193514}{ 3127239}$  &  $ \frac{ 184615935}{ 1542601}$  &  $ \frac{  688214053}{ 1331147}$  $ \frac{  1206026846}{ 420471}$  $ \frac{  7116193241}{ 405236}$  \\ \addlinespace
           &  $5$  &  $-\frac{  7469836609}{  76401}$  &  $-\frac{  2519869819}{ 151381}$  &  $-\frac{  676787627}{  209575}$  &  $-\frac{ 767075415}{  896921}$  &  $-\frac{ 379006664}{  761061}$  &  $-\frac{ 931274285}{  973468}$  &  $-\frac{ 2367490577}{  616772}$  $-\frac{   684405583}{  33590}$  $-\frac{  7097325924}{  58429}$  \\ \addlinespace
           &  $4$  &  $ \frac{  8640690184}{  23145}$  &  $ \frac{  2904329890}{  45589}$  &  $ \frac{ 2097415117}{  168915}$  &  $ \frac{1033739711}{  312683}$  &  $ \frac{  56509897}{   30173}$  &  $ \frac{1033739711}{  312683}$  &  $ \frac{ 2097415117}{  168915}$  $ \frac{  2904329890}{  45589}$  $ \frac{  8640690184}{  23145}$  \\ \addlinespace
           &  $3$  &  $-\frac{ 13491549889}{  16436}$  &  $-\frac{ 30871077827}{ 220014}$  &  $-\frac{  765629878}{   27919}$  &  $-\frac{ 828515195}{  113623}$  &  $-\frac{1288674710}{  324261}$  &  $-\frac{ 295058921}{   45739}$  &  $-\frac{ 2468363819}{  107827}$  $-\frac{  5737609802}{  50081}$  $-\frac{ 13534679320}{  20379}$  \\ \addlinespace
           &  $2$  &  $ \frac{ 29334155111}{  25771}$  &  $ \frac{  8450768743}{  43407}$  &  $ \frac{ 1334723167}{   35090}$  &  $ \frac{ 966000775}{   96443}$  &  $ \frac{1427976276}{  274865}$  &  $ \frac{1581790037}{  203396}$  &  $ \frac{  451561861}{   17139}$  $ \frac{  4693138545}{  36209}$  $ \frac{  9817971019}{  13153}$  \\ \addlinespace
           &  $1$  &  $-\frac{ 32612776236}{  31939}$  &  $-\frac{ 12258216466}{  70285}$  &  $-\frac{ 2028942806}{   59843}$  &  $-\frac{3054791233}{  349036}$  &  $-\frac{1014379655}{  237166}$  &  $-\frac{ 628691758}{  105883}$  &  $-\frac{ 5961122741}{  307109}$  $-\frac{  7652084383}{  81028}$  $-\frac{ 10120501295}{  18678}$  \\ \addlinespace
           &  $0$  &  $ \frac{958711850795}{3306139}$  &  $ \frac{138686396638}{2813507}$  &  $ \frac{ 3248190394}{  343067}$  &  $ \frac{2349998749}{  992475}$  &  $ \frac{ 467443989}{  432139}$  &  $ \frac{2253530669}{ 1605103}$  &  $ \frac{ 2952652193}{  659941}$  $ \frac{  3171898228}{ 146643}$  $ \frac{  9873545067}{  79705}$  \\ \addlinespace
      $5$  &  $5$  &  $-\frac{ 16400242834}{ 815393}$  &  $-\frac{  1275601375}{ 368936}$  &  $-\frac{  427576737}{  623480}$  &  $-\frac{ 135160981}{  704829}$  &  $-\frac{ 247486780}{ 1982753}$  &  $-\frac{ 709458479}{ 2638758}$  &  $-\frac{   61463934}{   53285}$  $-\frac{  1432715713}{ 225284}$  $-\frac{ 12858081715}{ 331389}$  \\ \addlinespace
           &  $4$  &  $ \frac{  8534140303}{  48995}$  &  $ \frac{  2675355119}{  89174}$  &  $ \frac{ 1268411423}{  212206}$  &  $ \frac{ 654146656}{  388723}$  &  $ \frac{ 889068808}{  829823}$  &  $ \frac{1034492709}{  485618}$  &  $ \frac{ 5241495620}{  615127}$  $ \frac{  2631734550}{  58459}$  $ \frac{ 13666821827}{  51060}$  \\ \addlinespace
           &  $3$  &  $-\frac{ 13534679320}{  20379}$  &  $-\frac{  5737609802}{  50081}$  &  $-\frac{ 2468363819}{  107827}$  &  $-\frac{ 295058921}{   45739}$  &  $-\frac{1288674710}{  324261}$  &  $-\frac{ 828515195}{  113623}$  &  $-\frac{  765629878}{   27919}$  $-\frac{ 30871077827}{ 220014}$  $-\frac{ 13491549889}{  16436}$  \\ \addlinespace
           &  $2$  &  $ \frac{ 25425670807}{  17442}$  &  $ \frac{ 21903079582}{  87043}$  &  $ \frac{ 3655479387}{   72668}$  &  $ \frac{3662929022}{  260087}$  &  $ \frac{2682354099}{  322987}$  &  $ \frac{3662929022}{  260087}$  &  $ \frac{ 3655479387}{   72668}$  $ \frac{ 21903079582}{  87043}$  $ \frac{ 25425670807}{  17442}$  \\ \addlinespace
           &  $1$  &  $-\frac{ 34046474687}{  16880}$  &  $-\frac{ 21436202114}{  61611}$  &  $-\frac{ 4882065990}{   70417}$  &  $-\frac{ 305554133}{   15991}$  &  $-\frac{1890391470}{  177121}$  &  $-\frac{8099595796}{  482187}$  &  $-\frac{ 7546651472}{  130969}$  $-\frac{ 32956224478}{ 116041}$  $-\frac{ 32852743324}{  20081}$  \\ \addlinespace
           &  $0$  &  $ \frac{ 26479157148}{  29351}$  &  $ \frac{  7222761881}{  46553}$  &  $ \frac{ 2631362108}{   85845}$  &  $ \frac{1879971092}{  228557}$  &  $ \frac{1224163507}{  283894}$  &  $ \frac{4054421226}{  639143}$  &  $ \frac{ 3256858005}{  154108}$  $ \frac{ 10194856899}{  98734}$  $ \frac{181942554161}{ 306771}$  \\ \addlinespace
      $4$  &  $4$  &  $ \frac{  1211629703}{  53483}$  &  $ \frac{  1990119523}{ 506979}$  &  $ \frac{  537364516}{  676097}$  &  $ \frac{  85841095}{  365273}$  &  $ \frac{ 323192477}{ 1923068}$  &  $ \frac{ 348597468}{  922523}$  &  $ \frac{  306856831}{  189251}$  $ \frac{  2318146475}{ 260443}$  $ \frac{  8028408627}{ 148285}$  \\ \addlinespace
           &  $3$  &  $-\frac{ 29831101642}{ 152201}$  &  $-\frac{  5136703769}{ 151046}$  &  $-\frac{ 2267814051}{  328385}$  &  $-\frac{6738238495}{ 3291754}$  &  $-\frac{ 659953893}{  463955}$  &  $-\frac{ 787874261}{  266082}$  &  $-\frac{10107954583}{  849559}$  $-\frac{  2463944763}{  39286}$  $-\frac{ 14121568547}{  37942}$  \\ \addlinespace
           &  $2$  &  $ \frac{  9817971019}{  13153}$  &  $ \frac{  4693138545}{  36209}$  &  $ \frac{  451561861}{   17139}$  &  $ \frac{1581790037}{  203396}$  &  $ \frac{1427976276}{  274865}$  &  $ \frac{ 966000775}{   96443}$  &  $ \frac{ 1334723167}{   35090}$  $ \frac{  8450768743}{  43407}$  $ \frac{ 29334155111}{  25771}$  \\ \addlinespace
           &  $1$  &  $-\frac{ 32852743324}{  20081}$  &  $-\frac{ 32956224478}{ 116041}$  &  $-\frac{ 7546651472}{  130969}$  &  $-\frac{8099595796}{  482187}$  &  $-\frac{1890391470}{  177121}$  &  $-\frac{ 305554133}{   15991}$  &  $-\frac{ 4882065990}{   70417}$  $-\frac{ 21436202114}{  61611}$  $-\frac{ 34046474687}{  16880}$  \\ \addlinespace
           &  $0$  &  $ \frac{  7211727349}{   6383}$  &  $ \frac{  5232843359}{  26730}$  &  $ \frac{11322353265}{  286802}$  &  $ \frac{1548885060}{  137633}$  &  $ \frac{7446840373}{ 1106172}$  &  $ \frac{1548885060}{  137633}$  &  $ \frac{11322353265}{  286802}$  $ \frac{  5232843359}{  26730}$  $ \frac{  7211727349}{   6383}$  \\ \addlinespace
      $3$  &  $3$  &  $-\frac{   800361473}{  48582}$  &  $-\frac{   557744521}{ 194407}$  &  $-\frac{  823497572}{ 1397105}$  &  $-\frac{ 185363617}{ 1015232}$  &  $-\frac{  81991005}{  573014}$  &  $-\frac{ 597649141}{ 1759029}$  &  $-\frac{  289784372}{  196989}$  $-\frac{   351689199}{  43600}$  $-\frac{  6519672839}{ 133134}$  \\ \addlinespace
           &  $2$  &  $ \frac{  6203677189}{  43561}$  &  $ \frac{  1696424402}{  68349}$  &  $ \frac{  982680142}{  192447}$  &  $ \frac{ 855538459}{  542278}$  &  $ \frac{1066785823}{  895146}$  &  $ \frac{5590654438}{ 2129495}$  &  $ \frac{ 2363787227}{  220958}$  $ \frac{ 17759778441}{ 314408}$  $ \frac{ 10624327325}{  31707}$  \\ \addlinespace
           &  $1$  &  $-\frac{ 10120501295}{  18678}$  &  $-\frac{  7652084383}{  81028}$  &  $-\frac{ 5961122741}{  307109}$  &  $-\frac{ 628691758}{  105883}$  &  $-\frac{1014379655}{  237166}$  &  $-\frac{3054791233}{  349036}$  &  $-\frac{ 2028942806}{   59843}$  $-\frac{ 12258216466}{  70285}$  $-\frac{ 32612776236}{  31939}$  \\ \addlinespace
           &  $0$  &  $ \frac{181942554161}{ 306771}$  &  $ \frac{ 10194856899}{  98734}$  &  $ \frac{ 3256858005}{  154108}$  &  $ \frac{4054421226}{  639143}$  &  $ \frac{1224163507}{  283894}$  &  $ \frac{1879971092}{  228557}$  &  $ \frac{ 2631362108}{   85845}$  $ \frac{  7222761881}{  46553}$  $ \frac{ 26479157148}{  29351}$  \\ \addlinespace
      $2$  &  $2$  &  $ \frac{  2005851423}{ 265880}$  &  $ \frac{  1414733955}{1073627}$  &  $ \frac{  329649921}{ 1205744}$  &  $ \frac{  67366110}{  766169}$  &  $ \frac{  48978927}{  651442}$  &  $ \frac{ 103779883}{  544689}$  &  $ \frac{  277579576}{  329887}$  $ \frac{  1919279425}{ 414313}$  $ \frac{  1051885279}{  37394}$  \\ \addlinespace
           &  $1$  &  $-\frac{  5910597075}{  90694}$  &  $-\frac{  1486183058}{ 130527}$  &  $-\frac{ 1883344606}{  797417}$  &  $-\frac{ 491966393}{  653081}$  &  $-\frac{ 257255959}{  418532}$  &  $-\frac{ 522065981}{  360998}$  &  $-\frac{ 1651888798}{  273307}$  $-\frac{  6349489117}{ 197436}$  $-\frac{  2523726139}{  13197}$  \\ \addlinespace
           &  $0$  &  $ \frac{  9873545067}{  79705}$  &  $ \frac{  3171898228}{ 146643}$  &  $ \frac{ 2952652193}{  659941}$  &  $ \frac{2253530669}{ 1605103}$  &  $ \frac{ 467443989}{  432139}$  &  $ \frac{2349998749}{  992475}$  &  $ \frac{ 3248190394}{  343067}$  $ \frac{138686396638}{2813507}$  $ \frac{958711850795}{3306139}$  \\ \addlinespace
      $1$  &  $1$  &  $-\frac{   989259649}{ 497859}$  &  $-\frac{   296572045}{ 853161}$  &  $-\frac{  178701734}{ 2462661}$  &  $-\frac{  28933143}{ 1204235}$  &  $-\frac{ 471882251}{21169910}$  &  $-\frac{ 186193587}{ 3061888}$  &  $-\frac{  699447262}{ 2521667}$  $-\frac{  1605498941}{1038640}$  $-\frac{  1291706883}{ 137012}$  \\ \addlinespace
           &  $0$  &  $ \frac{  1207396129}{ 140764}$  &  $ \frac{   550334507}{ 366830}$  &  $ \frac{  267692197}{  856297}$  &  $ \frac{ 193935861}{ 1901234}$  &  $ \frac{ 206821378}{ 2319277}$  &  $ \frac{ 308180301}{ 1366333}$  &  $ \frac{  526012837}{  537300}$  $ \frac{  8788336457}{1659246}$  $ \frac{  2789709824}{  87891}$  \\ \addlinespace
      $0$  &  $0$  &  $ \frac{    23000337}{ 199768}$  &  $ \frac{    36409563}{1806520}$  &  $ \frac{   14225607}{ 3370285}$  &  $ \frac{  25595175}{17925332}$  &  $ \frac{  25595175}{17925332}$  &  $ \frac{  14225607}{ 3370285}$  &  $ \frac{   36409563}{ 1806520}$  $ \frac{    23000337}{ 199768}$  $ \frac{   191906863}{ 270061}$  \\ \addlinespace
      \bottomrule
    \end{tabular}
  \end{center}
\end{table}

For example, if $r=3$, \eqref{eq:Lagrange} can be applied to the leftmost stencil $S_0=\left\{ x_{i-2}, x_{i-1}, x_i \right\}$ to obtain the polynomial $p_0$:

\begin{equation}
  \label{eq:pol_0}
  p_0(x_{i+\frac{1}{2}}) = \frac{3}{8} u_{i-2} - \frac{5}{4} u_{i-1} + \frac{15}{8} u_i
\end{equation}

and this approximation is third order accurate if the function $u(x)$ is smooth in the stencil $S_0$. If we choose a different stencil $S_1=\left\{ x_{i-1}, x_{i}, x_{i+1} \right\}$ we obtain the polynomial $p_1$:

\begin{equation}
  \label{eq:pol_1}
  p_1(x_{i+\frac{1}{2}}) = -\frac{1}{8} u_{i-1} + \frac{3}{4} u_i + \frac{3}{8} u_{i+1}
\end{equation}

that is also third order accurate. The last stencil that can be used is the stencil $S_2=\left\{ x_{i}, x_{i+1}, x_{i+2} \right\}$ to obtain the third order accurate interpolating polynom $p_2$:

\begin{equation}
  \label{eq:pol_2}
  p_2(x_{i+\frac{1}{2}}) = \frac{3}{8} u_i + \frac{3}{4} u_{i+1} - \frac{1}{8} u_{i+2}
\end{equation}

When $r=3$, using \eqref{eq:Lagrange_big} on the stencil $S= S_0 \cup S_1 \cup S_2$, we obtain a fifth order accurate approximation of the function $u$ at the point $x_{i+\frac{1}{2}}$:

\begin{equation}
  \label{eq:pol_union}
  P(x_{i+\frac{1}{2}}) = \frac{3}{128} u_{i-2} - \frac{5}{32} u_{i-1} + \frac{45}{64} u_i + \frac{15}{32} u_{i+1} - \frac{5}{128} u_{i+2}
\end{equation}

For $r=3$, $\gamma_0 = \frac{1}{16}$, $\gamma_1 = \frac{5}{8}$, $\gamma_2 = \frac{5}{16}$.


\clearpage

%\section{Application Program Interface}\label{sec:API}

In this section we review the FOODIE API providing a detailed discussion of the implementation choices.

As aforementioned, the programming language used is the Fortran 2008 standard, that is a minor revision of the previous Fortran 2003 standard. Such a new Fortran idioms provide (among other useful features) an almost complete support for OOP, in particular for ADT concept. Fortran 2003 has introduced the \emph{abstract derived type}: it is a derived type suitable to serve as \emph{contract} for concrete type-extensions that has not any actual implementations, rather it provides a well-defined set of type bound procedures interfaces, that in Fortran nomenclature are called \emph{deferred} procedures. Using such an abstract definition, we can implement algorithms operating on only this abstract type and on \emph{all its concrete extensions}. This is the key feature of FOODIE library: all the above described ODE solvers are implemented on the knowledge of \emph{only one abstract type}, allowing an implementation-style based on a very high-level syntax. In the meanwhile, client codes must implement their own IVPs extending only one simple abstract type.

In the subsection \ref{subsec:type_integrand} a review of the FOODIE main ADT, the \emph{integrand} type, is provided, while subsections \ref{subsec:solver_euler}, \ref{subsec:solver_tvd_rk}, \ref{subsec:solver_ls_rk}, \ref{subsec:solver_ab} and \ref{subsec:solver_leapfrog} cover the API of the currently implemented solvers.

It is worth noting that all FOODIE public \emph{entities} (ADT and solvers) must be accessed by the FOODIE module, see listing \ref{list:main_module} for an example on how access to all public FOODIE entities.

\begin{lstlisting}[firstnumber=1,style=code,caption={usage example importing all public entities of FOODIE main module},label={list:main_module}]
use foodie, only: integrand, &
                  adams_bashforth_integrator, &
                  adams_moulton_integrator, &
                  adams_bashforth_moulton_integrator, &
                  euler_explicit_integrator, &
                  leapfrog_integrator, &
                  ls_runge_kutta_integrator, &
                  tvd_runge_kutta_integrator
! or simply
use foodie
\end{lstlisting}

\subsection{The main FOODIE Abstract Data Type: the integrand type}\label{subsec:type_integrand}

The implemented ACP is based on one main ADT, the \emph{integrand} type, the definition of which is shown in listing~\ref{list:type_integrand}.

\begin{lstlisting}[firstnumber=1,style=code,caption={integrand type definition},label={list:type_integrand}]
type, abstract :: integrand
  !< Abstract type for building FOODIE ODE integrators.
  contains
    ! public deferred procedures that concrete integrand-field must implement
    procedure(time_derivative),       pass(self), deferred, public:: t
    ! operators
    procedure(symmetric_operator),   pass(lhs), deferred, public:: integrand_multiply_integrand
    procedure(integrand_op_real),    pass(lhs), deferred, public:: integrand_multiply_real
    procedure(real_op_integrand),    pass(rhs), deferred, public:: real_multiply_integrand
    procedure(symmetric_operator),   pass(lhs), deferred, public:: add
    procedure(symmetric_operator),   pass(lhs), deferred, public:: sub
    procedure(assignment_integrand), pass(lhs), deferred, public:: assign_integrand
    ! operators overloading
    generic, public:: operator(+) => add
    generic, public:: operator(-) => sub
    generic, public:: operator(*) => integrand_multiply_integrand, &
                                     real_multiply_integrand, &
                                     integrand_multiply_real
    generic, public:: assignment(=) => assign_integrand
endtype integrand
\end{lstlisting}

The \emph{integrand} type does not implement any actual integrand field, it being and abstract type. It only specifies which deferred procedures are necessary for implementing an actual concrete integrand type that can use a FOODIE solver.

As shown in listing \ref{list:type_integrand}, the number of the deferred type bound procedures that clients must implement into their own concrete extension of the \emph{integrand} ADT is very limited: essentially, there are 1 ODE-specific procedure plus some operators definition constituted by symmetric operators between 2 integrand objects, asymmetric operators between integrand and real numbers (and viceversa) and an assignment statement for the creation of new integrand objects. These procedures are analyzed in the following paragraphs.

\subsubsection{Time derivative procedure, the residuals function}

The abstract interface of the time derivative procedure \emph{t} is shown in listing \ref{list:time_derivative}.

\begin{lstlisting}[firstnumber=1,style=code,caption={time derivative procedure interface},label={list:time_derivative}]
function time_derivative(self, t) result(dState_dt)
import :: integrand, R_P, I_P
class(integrand),       intent(IN) :: self      !< Integrand field.
real(R_P),    optional, intent(IN) :: t         !< Time.
class(integrand), allocatable      :: dState_dt !< Result of the time derivative function of integrand field.
endfunction time_derivative
\end{lstlisting}

This procedure-function takes two arguments, the first passed as a \emph{type bounded} argument, while the latter is optional, and it returns an integrand object. The passed dummy argument, \emph{self}, is a polymorphic argument that could be any extensions of the \emph{integrand} ADT. The optional argument \emph{t} is the \emph{time} at which the residuals function must be computed: it can be omitted in the case the residuals function does not depend directly on time.

Commonly, into the concrete implementation of this deferred abstract procedure clients embed the actual ODE equations being solved. As an example, for the Burgers equation, that is a Partial Differential Equations (PDE) system involving also a boundary value problem, this procedure embeds the spatial operator that convert the PDE to a system of algebraic ODE. As a consequence, the eventual concrete implementation of this procedure can be very complex and errors-prone. Nevertheless, the FOODIE solvers are implemented only on the above abstract interface, thus emancipating the solvers implementation from any concrete complexity.

{\color{red} Add citations to Burgers, Adams-Bashfort, leapfrog references.}

\subsubsection{Symmetric operators procedures}

The abstract interface of \emph{symmetric} procedures is shown in listing \ref{list:symmetric_operator}.

\begin{lstlisting}[firstnumber=1,style=code,caption={symmetric operator procedure interface},label={list:symmetric_operator}]
function symmetric_operator(lhs, rhs) result(operator_result)
import :: integrand
class(integrand), intent(IN)  :: lhs             !< Left hand side.
class(integrand), intent(IN)  :: rhs             !< Right hand side.
class(integrand), allocatable :: operator_result !< Operator result.
endfunction symmetric_operator
\end{lstlisting}

This interface defines a class of procedures operating on 2 \emph{integrand} objects, namely it is used for the definition of the operators \emph{multiplication}, \emph{summation} and \emph{subtraction} of integrand objects. These operators are used into the above described ODE solvers, for example see equations \ref{eq:solver-euler}, \ref{eq:RK}, \ref{eq:AB} or \ref{eq:leapfrog}. The implementation details of such a procedures class are strictly dependent on the concrete extension of the integrand type. From the FOODIE solvers point of view, we need to known only that first argument passed as bounded one, the left-hand-side of the operator, and the second argument, the right-hand-side of the operator, are two integrand object and the returned object is still an integrand one.

\subsubsection{Integrand/real and real/integrand operators procedures}

The abstract interfaces of \emph{Integrand/real and real/integrand operators} procedures are shown in listing \ref{list:integrand_real_ops}.

\begin{lstlisting}[firstnumber=1,style=code,caption={Integrand/real and real/integrand operators procedure interfaces},label={list:integrand_real_ops}]
function integrand_op_real(lhs, rhs) result(operator_result)
import :: integrand, R_P
class(integrand), intent(IN)  :: lhs              !< Left hand side.
real(R_P),        intent(IN)  :: rhs              !< Right hand side.
class(integrand), allocatable :: operator_result  !< Operator result.
endfunction integrand_op_real

function real_op_integrand(lhs, rhs) result(operator_result)
import :: integrand, R_P
real(R_P),        intent(IN)  :: lhs             !< Left hand side.
class(integrand), intent(IN)  :: rhs             !< Right hand side.
class(integrand), allocatable :: operator_result !< Operator result.
endfunction real_op_integrand
\end{lstlisting}

These two interfaces are necessary in order to complete the \emph{algebra} operating on the integrand object class, allowing the multiplication of an integrand object for a real number, circumstance that happens in all solvers, see equations \ref{eq:solver-euler}, \ref{eq:RK}, \ref{eq:AB} or \ref{eq:leapfrog}. The implementation details of these procedures are strictly dependent on the concrete extension of the integrand type. From the FOODIE solvers point of view, we need to known only that first argument passed as bounded one, the left-hand-side of the operator, and the second argument, the right-hand-side of the operator, are an integrand object and real number of viceversa and the returned object is still an integrand one.

\subsubsection{Integrand assignment procedure}

The abstract interface of \emph{integrand assignment} procedure is shown in listing \ref{list:integrand_assignment}.

\begin{lstlisting}[firstnumber=1,style=code,caption={integrand assignment procedure interface},label={list:integrand_assignment}]
subroutine assignment_integrand(lhs, rhs)
import :: integrand
class(integrand), intent(INOUT) :: lhs !< Left hand side.
class(integrand), intent(IN)    :: rhs !< Right hand side.
endsubroutine assignment_integrand
\end{lstlisting}

The assignment statement is necessary in order to complete the \emph{algebra} operating on the integrand object class, allowing the assignment of an integrand object by another one, circumstance that happens in all solvers, see equations \ref{eq:solver-euler}, \ref{eq:RK}, \ref{eq:AB} or \ref{eq:leapfrog}. The implementation details of this assignment is strictly dependent on the concrete extension of the integrand type. From the FOODIE solvers point of view, we need to known only that first argument passed as bounded one, the left-hand-side of the assignment, and the second argument, the right-hand-side of the assignment, are two integrand objects.

\clearpage

\subsection{The explicit forward Euler solver}\label{subsec:solver_euler}

The explicit forward Euler solver is exposed (by the FOODIE main module that must imported, see listing \ref{list:main_module}) as a single derived type (that is a standard convention for all FOODIE solvers) named \emph{euler\_explicit\_integrator}. It provides the type bound procedure (also referred as \emph{method}) \emph{integrate} for integrating in time an \emph{integrand} object, or any of its polymorphic concrete extensions. Consequently, for using such a solver it must be previously defined as an instance of the exposed FOODIE integrator type, see listing \ref{list:define_euler_solver}.

\begin{lstlisting}[firstnumber=1,style=code,caption={definition of an explicit forward Euler integrator},label={list:define_euler_solver}]
use FOODIE, only: euler_explicit_integrator
type(euler_explicit_integrator) :: integrator
\end{lstlisting}

Once an integrator of this type has been instantiated, it can be directly used without any initialization, for example see listing \ref{list:use_euler_solver}.

\begin{lstlisting}[firstnumber=1,style=code,caption={example of usage of an explicit forward Euler integrator},label={list:use_euler_solver}]
type(my_integrand) :: my_field
call integrator%integrate(U=my_field, Dt=0.1)
\end{lstlisting}
where \emph{my\_integrand} is a concrete (valid) extension of \emph{integrand} ADT.

The complete implementation of the \emph{integrate} method of the explicit forward Euler solver is reported in listing \ref{list:euler_integrate}.

\begin{lstlisting}[firstnumber=1,style=code,caption={implementation of the \emph{integrate} method of Euler solver},label={list:euler_integrate}]
subroutine integrate(U, Dt, t)
class(integrand),    intent(INOUT) :: U  !< Field to be integrated.
real(R_P),           intent(IN)    :: Dt !< Time step.
real(R_P), optional, intent(IN)    :: t  !< Time.
U = U + U%t(t=t) * Dt
return
endsubroutine integrate
\end{lstlisting}
This method takes three arguments, the first argument is an integrand class, it being the integrand field that must integrated one-step-over in time, the second is the time step used and the third, that is optional, is the current time value that is passed to the residuals function for taking into account the cases where the time derivative explicitly depends on time. The time step is not automatically computed (for example inspecting the passed integrand field), thus its value must be externally computed and passed to the \emph{integrate} method.

\subsection{The explicit TVD/SSP Runge-Kutta class of solvers}\label{subsec:solver_tvd_rk}

The TVD/SSP Runge-Kutta class of solvers is exposed as a single derived type named \emph{tvd\_runge\_kutta\_integrator}. This type provides three methods:

\begin{itemize}
  \item \emph{init}: initialize the integrator accordingly the possibilities offered by the class of solvers;
  \item \emph{destroy}: destroy the integrator previously initialized, eventually freeing the allocated dynamic memory registers;
  \item \emph{integrate}: integrate integrand field one-step-over in time.
  \end{itemize}

As common for FOODIE solvers, for using such a solver it must be previously defined as an instance of the exposed FOODIE integrator type, see listing \ref{list:define_tvd_rk_solver}.

\begin{lstlisting}[firstnumber=1,style=code,caption={definition of an explicit TVD/SSP Runge-Kutta integrator},label={list:define_tvd_rk_solver}]
use FOODIE, only: tvd_runge_kutta_integrator
type(tvd_runge_kutta_integrator) :: integrator
\end{lstlisting}

Once an integrator of this type has been instantiated, it must be initialized before used, for example see listing \ref{list:init_tvd_rk_solver}.

\begin{lstlisting}[firstnumber=1,style=code,caption={example of initialization of an explicit TVD/SSP Runge-Kutta integrator},label={list:init_tvd_rk_solver}]
call integrator%init(stages=3)
\end{lstlisting}

In the listing \ref{list:init_tvd_rk_solver} a 3-stages solver has been initialized. As a matter of facts, from the equations \ref{eq:RK} and \ref{eq:RK-stage} a solver belonging to this class is completely defined once the number of stages adopted has been chosen. The complete definition of the \emph{tvd\_runge\_kutta\_integrator} type is reported into listing \ref{list:tvd_rk_solver}. As shown, the Butcher's coefficients are stored as allocatable arrays the values of which are initialized by the \emph{init} method.

\begin{lstlisting}[firstnumber=1,style=code,caption={definition of \emph{tvd\_runge\_kutta\_integrator} type},label={list:tvd_rk_solver}]
type :: tvd_runge_kutta_integrator
  integer(I_P)           :: stages=0  ! Number of stages.
  real(R_P), allocatable :: alph(:,:) ! alpha Butcher's coefficients.
  real(R_P), allocatable :: beta(:)   ! beta Butcher's coefficients.
  real(R_P), allocatable :: gamm(:)   ! gamma Butcher's coefficients.
  contains
    procedure, pass(self), public :: destroy
    procedure, pass(self), public :: init
    procedure, pass(self), public :: integrate
    final                         :: finalize
endtype tvd_runge_kutta_integrator
\end{lstlisting}

After the solver has been initialized it can be used for integrating an integrand field, as shown in listing \ref{list:use_tvd_rk_solver}.

\begin{lstlisting}[firstnumber=1,style=code,caption={example of usage of a TVD/SSP Runge-Kutta integrator},label={list:use_tvd_rk_solver}]
type(my_integrand) :: my_field
type(my_integrand) :: my_stages(1:3)
call integrator%integrate(U=my_field, stage=my_stage, Dt=0.1)
\end{lstlisting}
where \emph{my\_integrand} is a concrete (valid) extension of \emph{integrand} ADT. Listing \ref{list:use_tvd_rk_solver} shows that the memory registers necessary for storing the Runge-Kutta stages must be supplied by the client code.

The complete implementation of the \emph{integrate} method of the explicit TVD/SSP Runge-Kutta class of solvers is reported in listing \ref{list:tvd_rk_integrate}.

\begin{lstlisting}[firstnumber=1,style=code,caption={implementation of the \emph{integrate} method of explicit TVD/SSP Runge-Kutta class},label={list:tvd_rk_integrate}]
subroutine integrate(self, U, stage, Dt, t)
class(tvd_runge_kutta_integrator), intent(IN)    :: self      ! Actual RK integrator.
class(integrand),                  intent(INOUT) :: U         ! Field to be integrated.
class(integrand),                  intent(INOUT) :: stage(1:) ! Runge-Kutta stages [1:stages].
real(R_P),                         intent(IN)    :: Dt        ! Time step.
real(R_P),                         intent(IN)    :: t         ! Time.
integer(I_P)                                     :: s         ! First stages counter.
integer(I_P)                                     :: ss        ! Second stages counter.
select type(stage)
class is(integrand)
  do s=1, self%stages
    stage(s) = U
    do ss=1, s - 1
      stage(s) = stage(s) + stage(ss) * (Dt * self%alph(s, ss))
    enddo
    stage(s) = stage(s)%t(t=t + self%gamm(s) * Dt)
  enddo
  do s=1, self%stages
    U = U +  stage(s) * (Dt * self%beta(s))
  enddo
endselect
return
endsubroutine integrate
\end{lstlisting}

This method takes five arguments, the first argument is passed as bounded argument and it is the solver itself, the second is of an integrand class, it being the integrand field that must integrated one-step-over in time, the third is the stages array for storing the stages computations, the fourth is the time step used and the fifth, that is optional, is the current time value that is passed to the residuals function for taking into account the cases where the time derivative explicitly depends on time. The time step is not automatically computed (for example inspecting the passed integrand field), thus its value must be externally computed and passed to the \emph{integrate} method.

It is worth noting that the stages memory registers, namely the array \emph{stage}, must be passed as argument because it is defined as a \emph{not-passed} polymorphic argument, thus we are not allowed to define it as an automatic array of the \emph{integrate} method.

\subsection{The explicit low storage Runge-Kutta class of solvers}\label{subsec:solver_ls_rk}

The low storage variant of Runge-Kutta class of solvers is exposed as a single derived type named \emph{ls\_runge\_kutta\_integrator}. This type provides three methods:

\begin{itemize}
  \item \emph{init}: initialize the integrator accordingly the possibilities offered by the class of solvers;
  \item \emph{destroy}: destroy the integrator previously initialized, eventually freeing the allocated dynamic memory registers;
  \item \emph{integrate}: integrate integrand field one-step-over in time.
  \end{itemize}

As common for FOODIE solvers, for using such a solver it must be previously defined as an instance of the exposed FOODIE integrator type, see listing \ref{list:define_ls_rk_solver}.

\begin{lstlisting}[firstnumber=1,style=code,caption={definition of an explicit low storage Runge-Kutta integrator},label={list:define_ls_rk_solver}]
use FOODIE, only: ls_runge_kutta_integrator
type(ls_runge_kutta_integrator) :: integrator
\end{lstlisting}

Once an integrator of this type has been instantiated, it must be initialized before used, for example see listing \ref{list:init_ls_rk_solver}.

\begin{lstlisting}[firstnumber=1,style=code,caption={example of initialization of an explicit low storage Runge-Kutta integrator},label={list:init_ls_rk_solver}]
call integrator%init(stages=5)
\end{lstlisting}

In the listing \ref{list:init_ls_rk_solver} a 5-stages solver has been initialized. As a matter of facts, from the equation \ref{eq:RK-ls} a solver belonging to this class is completely defined once the number of stages adopted has been chosen. The complete definition of the \emph{ls\_runge\_kutta\_integrator} type is reported into listing \ref{list:ls_rk_solver}. As shown, the Williamson's coefficients are stored as allocatable arrays the values of which are initialized by the \emph{init} method.

\begin{lstlisting}[firstnumber=1,style=code,caption={definition of \emph{ls\_runge\_kutta\_integrator} type},label={list:ls_rk_solver}]
type :: ls_runge_kutta_integrator
  integer(I_P)           :: stages=0 ! Number of stages.
  real(R_P), allocatable :: A(:)     ! Low storage *A* coefficients.
  real(R_P), allocatable :: B(:)     ! Low storage *B* coefficients.
  real(R_P), allocatable :: C(:)     ! Low storage *C* coefficients.
  contains
    procedure, pass(self), public :: destroy
    procedure, pass(self), public :: init
    procedure, pass(self), public :: integrate
    final                         :: finalize
endtype ls_runge_kutta_integrator
\end{lstlisting}

After the solver has been initialized it can be used for integrating an integrand field, as shown in listing \ref{list:use_ls_rk_solver}.

\begin{lstlisting}[firstnumber=1,style=code,caption={example of usage of a low storage Runge-Kutta integrator},label={list:use_ls_rk_solver}]
type(my_integrand) :: my_field
type(my_integrand) :: my_stages(1:2)
call integrator%integrate(U=my_field, stage=my_stage, Dt=0.1)
\end{lstlisting}
where \emph{my\_integrand} is a concrete (valid) extension of \emph{integrand} ADT. Listing \ref{list:use_ls_rk_solver} shows that the memory registers necessary for storing the Runge-Kutta stages must be supplied by the client code, as it happens of the TVD/SSP Runge-Kutta class. However, now the registers necessary is always 2, independently on the number of stages used, that in the example considered are 5.

The complete implementation of the \emph{integrate} method of the explicit low storage Runge-Kutta class of solvers is reported in listing \ref{list:ls_rk_integrate}.

\begin{lstlisting}[firstnumber=1,style=code,caption={implementation of the \emph{integrate} method of explicit low storage Runge-Kutta class},label={list:ls_rk_integrate}]
subroutine integrate(self, U, stage, Dt, t)
class(ls_runge_kutta_integrator), intent(IN)    :: self       ! Actual RK integrator.
class(integrand),                 intent(INOUT) :: U          ! Field to be integrated.
class(integrand),                 intent(INOUT) :: stage(1:2) ! Runge-Kutta registers [1:2].
real(R_P),                        intent(IN)    :: Dt         ! Time step.
real(R_P),                        intent(IN)    :: t          ! Time.
integer(I_P)                                    :: s          ! First stages counter.
select type(stage)
class is(integrand)
  stage(1) = U
  stage(2) = U*0._R_P
  do s=1, self%stages
    stage(2) = stage(2) * self%A(s) + stage(1)%t(t=t + self%C(s) * Dt) * Dt
    stage(1) = stage(1) + stage(2) * self%B(s)
  enddo
  U = stage(1)
endselect
return
endsubroutine integrate
\end{lstlisting}

This method takes five arguments, the first argument is passed as bounded argument and it is the solver itself, the second is of an integrand class, it being the integrand field that must integrated one-step-over in time, the third is the stages array for storing the stages computations, the fourth is the time step used and the fifth, that is optional, is the current time value that is passed to the residuals function for taking into account the cases where the time derivative explicitly depends on time. The time step is not automatically computed (for example inspecting the passed integrand field), thus its value must be externally computed and passed to the \emph{integrate} method.

It is worth noting that the stages memory registers, namely the array \emph{stage}, must be passed as argument because it is defined as a \emph{not-passed} polymorphic argument, thus we are not allowed to define it as an automatic array of the \emph{integrate} method.

\subsection{The explicit Adams-Bashforth class of solvers}\label{subsec:solver_ab}

The explicit Adams-Bashforth class of solvers is exposed as a single derived type named \emph{adams\_bashforth\_integrator}. This type provides three methods:

\begin{itemize}
  \item \emph{init}: initialize the integrator accordingly the possibilities offered by the class of solvers;
  \item \emph{destroy}: destroy the integrator previously initialized, eventually freeing the allocated dynamic memory registers;
  \item \emph{integrate}: integrate integrand field one-step-over in time;
  \item \emph{update\_previous}: auto update (cyclically) previous time steps solutions.
  \end{itemize}

As common for FOODIE solvers, for using such a solver it must be previously defined as an instance of the exposed FOODIE integrator type, see listing \ref{list:define_ab_solver}.

\begin{lstlisting}[firstnumber=1,style=code,caption={definition of an explicit Adams-Bashforth integrator},label={list:define_ab_solver}]
use FOODIE, only: adams_bashforth_integrator
type(adams_bashforth_integrator) :: integrator
\end{lstlisting}

Once an integrator of this type has been instantiated, it must be initialized before used, for example see listing \ref{list:init_ab_solver}.

\begin{lstlisting}[firstnumber=1,style=code,caption={example of initialization of an explicit Adams-Bashforth integrator},label={list:init_ab_solver}]
call integrator%init(steps=4)
\end{lstlisting}

In the listing \ref{list:init_ab_solver} a 4-steps solver has been initialized. As a matter of facts, from the equation \ref{eq:AB} a solver belonging to this class is completely defined once the number of time steps adopted has been chosen. The complete definition of the \emph{adams\_bashforth\_integrator} type is reported into listing \ref{list:ab_solver}. As shown, the linear coefficients are stored as allocatable arrays the values of which are initialized by the \emph{init} method.

\begin{lstlisting}[firstnumber=1,style=code,caption={definition of \emph{adams\_bashforth\_integrator} type},label={list:ab_solver}]
type :: adams_bashforth_integrator
  private
  integer(I_P)           :: steps=0 ! Number of time steps.
  real(R_P), allocatable :: b(:)    ! \(b\) coefficients.
  contains
    procedure, pass(self), public :: destroy
    procedure, pass(self), public :: init
    procedure, pass(self), public :: integrate
    procedure, pass(self), public :: update_previous
    final                         :: finalize
endtype adams_bashforth_integrator
\end{lstlisting}

After the solver has been initialized it can be used for integrating an integrand field, as shown in listing \ref{list:use_ab_solver}.

\begin{lstlisting}[firstnumber=1,style=code,caption={example of usage of an Adams-Bashforth integrator},label={list:use_ab_solver}]
real               :: times(1:4)
type(my_integrand) :: my_field
type(my_integrand) :: previous(1:4)
call integrator%integrate(U=my_field, previous=previous, Dt=Dt, t=times)
\end{lstlisting}
where \emph{my\_integrand} is a concrete (valid) extension of \emph{integrand} ADT, \emph{times} are the time at each 4 steps considered for the current one-step-over integration and \emph{previous} are the memory registers where previous time steps solutions are saved.

The complete implementation of the \emph{integrate} method of the explicit Adams-Bashforth class of solvers is reported in listing \ref{list:ab_integrate}.

\begin{lstlisting}[firstnumber=1,style=code,caption={implementation of the \emph{integrate} method of explicit Adams-Bashforth class},label={list:ab_integrate}]
subroutine integrate(self, U, previous, Dt, t, autoupdate)
class(adams_bashforth_integrator), intent(IN)    :: self         ! Actual AB integrator.
class(integrand),                  intent(INOUT) :: U            ! Field to be integrated.
class(integrand),                  intent(INOUT) :: previous(1:) ! Previous time steps solutions.
real(R_P),                         intent(IN)    :: Dt           ! Time steps.
real(R_P),                         intent(IN)    :: t(:)         ! Times.
logical, optional,                 intent(IN)    :: autoupdate   ! Autoupdate previous time steps.
logical                                          :: autoupdate_  ! autoupdate previous time steps, dummy var.
integer(I_P)                                     :: s            ! Steps counter.
autoupdate_ = .true. ; if (present(autoupdate)) autoupdate_ = autoupdate
do s=1, self%steps
  U = U + previous(s)%t(t=t(s)) * (Dt * self%b(s))
enddo
if (autoupdate_) call self%update_previous(U=U, previous=previous)
return
endsubroutine integrate
\end{lstlisting}

This method takes five arguments, the first argument is passed as bounded argument and it is the solver itself, the second is of an integrand class, it being the integrand field that must integrated one-step-over in time, the third are the previous time steps solutions, the fourth is the time step used, the fifth is an array of the time values of the steps considered for the current one-step-over integration that are passed to the residuals function for taking into account the cases where the time derivative explicitly depends on time and the sixth is a logical flag for enabling/disabling the cyclic update of previous time steps solutions. The time step is not automatically computed (for example inspecting the passed integrand field), thus its value must be externally computed and passed to the \emph{integrate} method.

It is worth noting that the method also performs the cyclic update of the previous time steps solutions memory registers. This can be disable passing \emph{autoupdate=.false.}: it is useful in the framework of predictor-corrector solvers.

\subsection{The implicit Adams-Moulton class of solvers}\label{subsec:solver_am}

The explicit Adams-Moulton class of solvers is exposed as a single derived type named \emph{adams\_moulton\_integrator}. This type provides three methods:

\begin{itemize}
  \item \emph{init}: initialize the integrator accordingly the possibilities offered by the class of solvers;
  \item \emph{destroy}: destroy the integrator previously initialized, eventually freeing the allocated dynamic memory registers;
  \item \emph{integrate}: integrate integrand field one-step-over in time;
  \item \emph{update\_previous}: auto update (cyclically) previous time steps solutions.
  \end{itemize}

As common for FOODIE solvers, for using such a solver it must be previously defined as an instance of the exposed FOODIE integrator type, see listing \ref{list:define_am_solver}.

\begin{lstlisting}[firstnumber=1,style=code,caption={definition of an implicit Adams-Moulton integrator},label={list:define_am_solver}]
use FOODIE, only: adams_moulton_integrator
type(adams_moulton_integrator) :: integrator
\end{lstlisting}

Once an integrator of this type has been instantiated, it must be initialized before used, for example see listing \ref{list:init_am_solver}.

\begin{lstlisting}[firstnumber=1,style=code,caption={example of initialization of an implicit Adams-Moulton integrator},label={list:init_am_solver}]
call integrator%init(steps=3)
\end{lstlisting}

In the listing \ref{list:init_am_solver} a 3-steps solver has been initialized. As a matter of facts, from the equation \ref{eq:AM} a solver belonging to this class is completely defined once the number of time steps adopted has been chosen. The complete definition of the \emph{adams\_moulton\_integrator} type is reported into listing \ref{list:am_solver}. As shown, the linear coefficients are stored as allocatable arrays the values of which are initialized by the \emph{init} method.

\begin{lstlisting}[firstnumber=1,style=code,caption={definition of \emph{adams\_moulton\_integrator} type},label={list:am_solver}]
type :: adams_moulton_integrator
  private
  integer(I_P)           :: steps=-1 ! Number of time steps.
  real(R_P), allocatable :: b(:)     ! \(b\) coefficients.
  contains
    procedure, pass(self), public :: destroy
    procedure, pass(self), public :: init
    procedure, pass(self), public :: integrate
    procedure, pass(self), public :: update_previous
    final                         :: finalize
endtype adams_moulton_integrator
\end{lstlisting}

After the solver has been initialized it can be used for integrating an integrand field, as shown in listing \ref{list:use_am_solver}.

\begin{lstlisting}[firstnumber=1,style=code,caption={example of usage of an Adams-Moulton integrator},label={list:use_am_solver}]
real               :: times(1:3)
type(my_integrand) :: my_field
type(my_integrand) :: previous(1:3)
call integrator%integrate(U=my_field, previous=previous, Dt=Dt, t=times)
\end{lstlisting}
where \emph{my\_integrand} is a concrete (valid) extension of \emph{integrand} ADT, \emph{times} are the time at each 4 steps considered for the current one-step-over integration and \emph{previous} are the memory registers where previous time steps solutions are saved.

The complete implementation of the \emph{integrate} method of the implicit Adams-Moulton class of solvers is reported in listing \ref{list:am_integrate}.

\begin{lstlisting}[firstnumber=1,style=code,caption={implementation of the \emph{integrate} method of explicit Adams-Moulton class},label={list:am_integrate}]
subroutine integrate(self, U, previous, Dt, t, autoupdate)
class(adams_bashforth_integrator), intent(IN)    :: self         ! Actual AB integrator.
class(integrand),                  intent(INOUT) :: U            ! Field to be integrated.
class(integrand),                  intent(INOUT) :: previous(1:) ! Previous time steps solutions.
real(R_P),                         intent(IN)    :: Dt           ! Time steps.
real(R_P),                         intent(IN)    :: t(:)         ! Times.
logical, optional,                 intent(IN)    :: autoupdate   ! Autoupdate previous time steps.
logical                                          :: autoupdate_  ! autoupdate previous time steps, dummy var.
integer(I_P)                                     :: s            ! Steps counter.
autoupdate_ = .true. ; if (present(autoupdate)) autoupdate_ = autoupdate
if (autoupdate_) call self%update_previous(U=U, previous=previous)
if (self%steps>0) then
  U = previous(self%steps) + U%t(t=t(self%steps) + Dt) * (Dt * self%b(self%steps))
  do s=0, self%steps - 1
    U = U + previous(s+1)%t(t=t(s+1)) * (Dt * self%b(s))
  enddo
  if (autoupdate_) call self%update_previous(U=U, previous=previous)
else
  U = U + U%t(t=t(s+1)) * (Dt * self%b(0))
endif
return
endsubroutine integrate
\end{lstlisting}

This method takes six arguments, the first argument is passed as bounded argument and it is the solver itself, the second is of an integrand class, it being the integrand field that must integrated one-step-over in time, the third are the previous time steps solutions, the fourth is the time step used, the fifth is an array of the time values of the steps considered for the current one-step-over integration that are passed to the residuals function for taking into account the cases where the time derivative explicitly depends on time and the sixth is a logical flag for enabling/disabling the cyclic update of previous time steps solutions. The time step is not automatically computed (for example inspecting the passed integrand field), thus its value must be externally computed and passed to the \emph{integrate} method.

It is worth noting that the method also performs the cyclic update of the previous time steps solutions memory registers. This can be disable passing \emph{autoupdate=.false.}: it is useful in the framework of predictor-corrector solvers.

\subsection{The predictor-corrector Adams-Bashforth-Moulton class of solvers}\label{subsec:solver_abm}

{\color{red} To be written.}

\subsection{The leapfrog solver}\label{subsec:solver_leapfrog}

The explicit Leapfrog class of solvers is exposed as a single derived type named \emph{leapfrog\_integrator}. This type provides three methods:

\begin{itemize}
  \item \emph{init}: initialize the integrator accordingly the possibilities offered by the class of solvers;
  \item \emph{integrate}: integrate integrand field one-step-over in time.
  \end{itemize}

As common for FOODIE solvers, for using such a solver it must be previously defined as an instance of the exposed FOODIE integrator type, see listing \ref{list:define_lf_solver}.

\begin{lstlisting}[firstnumber=1,style=code,caption={definition of an explicit Leapfrog integrator},label={list:define_lf_solver}]
use FOODIE, only: leapfrog_integrator
type(leapfrog_integrator) :: integrator
\end{lstlisting}

Once an integrator of this type has been instantiated, it must be initialized before used, for example see listing \ref{list:init_lf_solver}.

\begin{lstlisting}[firstnumber=1,style=code,caption={example of initialization of an explicit Leapfrog integrator},label={list:init_lf_solver}]
! default coefficients nu=0.01, alpha=0.53
call integrator%init()
! custom coefficients
call integrator%init(nu=0.015, alpha=0.6)
\end{lstlisting}

The complete definition of the \emph{leapfrog\_integrator} type is reported into listing \ref{list:lf_solver}. As shown, the filter coefficients are initialized to zero, suitable values are initialized by the \emph{init} method.

\begin{lstlisting}[firstnumber=1,style=code,caption={definition of \emph{leapfrog\_integrator} type},label={list:lf_solver}]
type :: leapfrog_integrator
  private
  real(R_P) :: nu=0.01_R_P    ! Robert-Asselin filter coefficient.
  real(R_P) :: alpha=0.53_R_P ! Robert-Asselin-Williams filter coefficient.
  contains
    procedure, pass(self), public :: init
    procedure, pass(self), public :: integrate
endtype leapfrog_integrator
\end{lstlisting}

After the solver has been initialized it can be used for integrating an integrand field, as shown in listing \ref{list:use_lf_solver}.

\begin{lstlisting}[firstnumber=1,style=code,caption={example of usage of a Leapfrog integrator},label={list:use_lf_solver}]
real               :: times(1:2)
type(my_integrand) :: filter_displacement
type(my_integrand) :: my_field
type(my_integrand) :: previous(1:2)
call integrator%integrate(U=my_field, previous=previous, filter=filter_displacement, Dt=Dt, &
                          t=times)
\end{lstlisting}
where \emph{my\_integrand} is a concrete (valid) extension of \emph{integrand} ADT, \emph{previous} are the memory registers where previous time steps solutions are saved, \emph{filter\_displacement} is the register necessary for computing the eventual displacement of the applied filter and \emph{times} are the time at each 2 steps considered for the current one-step-over integration.

The complete implementation of the \emph{integrate} method of the explicit Leapfrog class of solvers is reported in listing \ref{list:lf_integrate}.

\begin{lstlisting}[firstnumber=1,style=code,caption={implementation of the \emph{integrate} method of explicit Leapfrog class},label={list:lf_integrate}]
subroutine integrate(self, U, previous, Dt, t, filter)
class(leapfrog_integrator), intent(IN)    :: self          ! LF integrator.
class(integrand),           intent(INOUT) :: U             ! Field to be integrated.
class(integrand),           intent(INOUT) :: previous(1:2) ! Previous time steps solutions.
real(R_P),                  intent(in)    :: Dt            ! Time step.
real(R_P),                  intent(IN)    :: t             ! Time.
class(integrand), optional, intent(INOUT) :: filter        ! Filter field displacement.
U = previous(1) + previous(2)%t(t=t) * (Dt * 2._R_P)
if (present(filter)) then
  filter = (previous(1) - previous(2) * 2._R_P + U) * self%nu * 0.5_R_P
  previous(2) = previous(2) + filter * self%alpha
  U = U + filter * (self%alpha - 1._R_P)
endif
previous(1) = previous(2)
previous(2) = U
return
endsubroutine integrate
\end{lstlisting}

This method takes six arguments, the first argument is passed as bounded argument and it is the solver itself, the second is of an integrand class, it being the integrand field that must integrated one-step-over in time, the third are the previous time steps solutions, the fourth is the optional filter-displacement-register, the fifth is the time step used and the sixth is an array of the time values of the steps considered for the current one-step-over integration that are passed to the residuals function for taking into account the cases where the time derivative explicitly depends on time. The time step is not automatically computed (for example inspecting the passed integrand field), thus its value must be externally computed and passed to the \emph{integrate} method. It is worth noting that if the filter displacement argument is not passed, the solver reverts back to the standard unfiltered Leapfrog method.

It is worth noting that the method also performs the cyclic update of the previous time steps solutions memory registers. In particular, if the filter displacement argument is passed the method performs the RAW filtering.

\subsection{General Remarks}\label{subsec:API-remarks}

Table \ref{tab:foodie-vs-math} presents a comparison of the relevant parts of equations \ref{eq:solver-euler}, \ref{eq:RK}, \ref{eq:RK-stage}, \ref{eq:RK-ls}, \ref{eq:AB} and \ref{eq:leapfrog} with the corresponding FOODIE implementations reported in listings \ref{list:euler_integrate}, \ref{list:tvd_rk_integrate}, \ref{list:ls_rk_integrate}, \ref{list:ab_integrate} and \ref{list:lf_integrate}, respectively. This comparison proves that the \emph{integrand} ADT has allowed a very high-level implementation syntax. The Fortran implementation is almost equivalent to the rigorous mathematical formulation. This aspect directly implies that the implementation of a ODE solver into the FOODIE library is very clear, concise and less-errors-prone than an \emph{hard-coded} implementation where the solvers must be implemented for each specific definition of the integrand type, it being not abstract.

\begin{table}[!ht]
  \centering
  \caption{Comparison between rigorous mathematical formulation and FOODIE high-level implementation; the syntax "$(s)$" and "$(ss)$" imply the summation operation}\label{tab:foodie-vs-math}
  \resizebox{0.99\textwidth}{!}{%
  \begin{tabular}{c|cc}
    \toprule
    {\sc Solver} & {\sc Mathematical formulation} & {\sc FOODIE implementation} \\
    \hline
    explicit forward Euler & $U\left(t^{n+1}\right) = U\left(t^n\right) +\Delta t \cdot R\left[t^n, U\left(t^n\right)\right]$ & $U = U + U\%t(t=t) * Dt$\\
    \hline
    TVD/SSP Runge-Kutta & $\begin{matrix}
  K_s = R\left(t^n + \gamma_s \Delta t, U^n+\Delta t\sum\limits_{l=1}^{s-1}{\alpha_{s,l} K_l} \right) \\
  U^{n+1} = U^{n} + \Delta t \cdot\sum\limits_{s=1}^{N_{s}}{\beta_s K_s}
\end{matrix}$ & $\begin{matrix}
  stage(s) = stage(s) + stage(ss) * (Dt * self\%alph(s, ss)) \\
  U = U +  stage(s) * (Dt * self\%beta(s))
\end{matrix}$\\
    \hline
    low storage Runge-Kutta & $\begin{matrix}
  K_2 = A_s K_2 + \Delta t \cdot R\left(t^n + C_s \Delta t, K_1\right) \\
  K_1 = K_1 + B_s K_2
\end{matrix}$ & $\begin{matrix}
  stage(2) = stage(2) * self\%A(s) + \\ + stage(1)\%t(t=t + self\%C(s) * Dt) * Dt \\
  stage(1) = stage(1) + stage(2) * self\%B(s)
\end{matrix}$\\
    \hline
  explicit Adams-Bashforth & $\begin{matrix}U\left(t^{n+N_s}\right) = U\left(t^{n+N_s-1}\right) + \\ +\Delta t \sum_{s=1}^{n+N_s}{ b_s \cdot R\left[t^{n+s-1}, U\left(t^{n+s-1}\right)\right]}\end{matrix}$ & $U = U + U\%t(n=s, t=t(s)) * (Dt * self\%b(s))$\\
    \hline
    explicit Leapfrog & $ U\left(t^{n+2}\right) = U\left(t^{n}\right) + 2\Delta t \cdot R\left[t^{n+1}, U\left(t^{n+1}\right)\right]$ & $U = U\%previous\_step(n=1) + U\%t(n=2, t=t) * (Dt * 2.)$\\

  \end{tabular}}
\end{table}



\clearpage

%\section{Tests and Examples}\label{sec:Tests}

For the assessment of FOODIE capabilities three different tests are discussed:

\begin{itemize}
  \item Oscillation equations IVP: this test is a pure ODEs, non-stiff problem;
  \item Lorenz equations IVP: this test is a pure ODEs, stiff problem;
  \item 1D Euler equations IVP: this test is a PDEs problem;
  \end{itemize}

With the first two pure ODEs tests we assess the accuracy of the FOODIE built-in solvers and demonstrate how to use the library in real applications. The last test discussion is mainly devoted to the demonstration of FOODIE application into a real, complex PDEs problem.

\subsection{Oscillation equations test}

Let us consider the \emph{oscillation} problem, it being a simple, yet interesting IVP. Briefly, the oscillation problem is a prototype problem of non dissipative, oscillatory phenomena. For example, let us consider a pendulum subjected to the Coriolis accelerations without dissipation, the motion equations of which can be described by the ODE system \ref{eq:oscillation}.

\begin{equation}
\begin{matrix}
U_t = R(U)  \\
U = \begin{bmatrix}
x \\
y
\end{bmatrix}\;\;\;
R(U) = \begin{bmatrix}
-f y \\
f x
\end{bmatrix}
\end{matrix}
\label{eq:oscillation}
\end{equation}
where the frequency is chosen as $f=10^4$. The ODE system \ref{eq:oscillation} is completed by the following initial conditions:

\begin{equation}
\begin{matrix}
  x(t_0) = 0 \\
  y(t_0) = 1
\end{matrix}
\label{eq:oscillation-ic}
\end{equation}
where $t_0=0$ is the initial time considered.

The IVP constituted by equations \ref{eq:oscillation} and \ref{eq:oscillation-ic} is (apparently) simple and its exact solution is known:

\begin{equation}
\begin{matrix}
  x(t_0 + \Delta t) = X_0 cos(f \Delta t) - y_0 sin(f \Delta t) \\
  y(t_0 + \Delta t) = X_0 sin(f \Delta t) + y_0 cos(f \Delta t)
\end{matrix}
\label{eq:oscillation-exact}
\end{equation}
where $\Delta t$ is an arbitrary time step. This problem is non-stiff meaning that the solution is constituted by only one time-scale, namely the single frequency $f$.

This problem is only apparently simple. As a matter of facts, in a non dissipative oscillatory problem the eventual errors in the amplitude approximation can rapidly drive the subsequent series of approximations to an unphysical solution. This is of particular relevance if the solution (that is numerically approximated) constitutes a \emph{prediction} far from the initial conditions, that is the common case in weather forecasting.

Because the Oscillation system \ref{eq:oscillation} posses a closed exact solution, the discussion on this test has twofolds aims: to assess the accuracy of the FOODIE's built-in solvers comparing the numerical solutions with the exact one and to demonstrate how it is simple to solve this prototypical problem by means of FOODIE.

\subsubsection{Errors Analysis}

For the analysis of the accuracy of each solver, we have integrated the Oscillation equations \ref{eq:oscillation} with different, decreasing time steps in the range $[5000, 2500, 1250, 625, 320, 100]$. The error is estimated by the L2 norm of the difference between the exact ($U_e$) and the numerical ($U_{\Delta t}$) solutions for each time step:

\begin{equation}
  \varepsilon (\Delta t) = || U_e - U_{\Delta t} ||_2 = \sqrt{ \sum_{s=1}^{N_s} { \left(U_e(t_0 + s * \Delta t) - U_{\Delta t}(t_0 + s * \Delta t) \right)^2 }}
\label{eq:oscillation-error}
\end{equation}
where $N_s$ is the total number of time steps performed to reach the final integration time.

Using two pairs of subsequent-decreasing time steps solution is possible to estimate the order of accuracy of the solver employed computing the \emph{observed order} of accuracy:

\begin{equation}
  p = \frac{log10 \left( \frac{\varepsilon (\Delta t_1)}{\varepsilon (\Delta t_2)} \right)}{log10 \left( \frac{\Delta t_1}{\Delta t_2} \right)}
\label{eq:oscillation-observed-order}
\end{equation}

where $\frac{\Delta t_1}{\Delta t_2}>1$.

\subsubsection{FOODIE aware implementation of an oscillation numerical solver}

The IVP \ref{eq:oscillation} can be easily solved by means of FOODIE library. The first block of a FOODIE aware solution consists to define an \emph{oscillation integrand field} defining a concrete extension of the FOODIE \emph{integrand} type. Listing \ref{list:oscillation_type} reports the implementation of such an integrand field that is contained into the tests suite shipped within the FOODIE library.

\begin{lstlisting}[firstnumber=1,style=code,caption={implementation of the \emph{oscillation integrand} type},label={list:oscillation_type}]
type, extends(integrand) :: oscillation
  private
  integer(I_P)                           :: dims=0   ! Space dimensions.
  real(R_P)                              :: f=0._R_P ! Oscillation frequency (Hz).
  real(R_P), dimension(:),   allocatable :: U        ! Integrand (state) variables, [1:dims].
  contains
    ! auxiliary methods
    procedure, pass(self), public :: init
    procedure, pass(self), public :: output
    ! type\_integrand deferred methods
    procedure, pass(self), public :: t => dOscillation_dt
    procedure, pass(lhs),  public :: integrand_multiply_integrand => &
                                     oscillation_multiply_oscillation
    procedure, pass(lhs),  public :: integrand_multiply_real => oscillation_multiply_real
    procedure, pass(rhs),  public :: real_multiply_integrand => real_multiply_oscillation
    procedure, pass(lhs),  public :: add => add_oscillation
    procedure, pass(lhs),  public :: sub => sub_oscillation
    procedure, pass(lhs),  public :: assign_integrand => oscillation_assign_oscillation
    procedure, pass(lhs),  public :: assign_real => oscillation_assign_real
endtype oscillation
\end{lstlisting}

The \emph{oscillation} field extends the \emph{integrand} ADT making it a concrete type. This derived type is very simple: it has 5 data members for storing the state vector and some auxiliary variables, and it implements all the deferred methods necessary for defining a valid concrete extension of the \emph{integrand} ADT (plus 2 auxiliary methods that are not relevant for our discussion). The key point is here constituted by the implementation of the deferred methods: the \emph{integrand} ADT does not impose any structure for the data members, that are consequently free to be customized by the client code. In this example the data members have a very simple, clean and concise structure:
\begin{itemize}
  \item $dims$ is the number of space dimensions; in the case of equation \ref{eq:oscillation} we have $dims=2$, however the test implementation has been kept more general parametrizing this dimension in order to easily allow future modification of the test-program itself;
  \item $f$ stores the frequency of the oscillatory problem solved, that is here set to $10^{4}$, but it can be changed at runtime in the test-program;
  \item $U$ is the state vector corresponding directly to the state vector of equation \ref{eq:oscillation};
  \end{itemize}

As the listing \ref{list:oscillation_type} shows, the FOODIE implementation strictly corresponds to the mathematical formulation embracing all the relevant mathematical aspects into one derived type, a single \emph{object}. Here we not review the implementation of all deferred methods, this being out of the scope of the present work: the interested reader can see the tests suite sources shipped within the FOODIE library. However, some of these methods are relevant for our discussion, thus they are reviewed.

\paragraph{dOscillation\_dt, the oscillation residuals function}

Probably, the most important methods for an IVP solver is the residuals function. As a matter of facts, the ODE equations are implemented into the residuals function. However, the FOODIE ADT strongly alleviates the subtle problems that could arise when the ODE solver is hard-implemented within the specific ODE equations. As a matter of facts, the \emph{integrand} ADT specifies the precise interface the residuals function must have: if the client code implements a compliant interface, the FOODIE solvers will work as expected, reducing the possible errors location into the ODE equations, having designed the solvers on the ADT and not on the concrete type.

\begin{lstlisting}[firstnumber=1,style=code,caption={implementation of the \emph{oscillation integrand} residuals function},label={list:oscillation_t}]
function dOscillation_dt(self, t) result(dState_dt)
class(oscillation),     intent(IN) :: self      ! Oscillation field.
real(R_P),    optional, intent(IN) :: t         ! Time.
class(integrand),  allocatable     :: dState_dt ! Oscillation field time derivative.
integer(I_P)                       :: dn        ! Time level, dummy variable.
allocate(oscillation :: dState_dt)
select type(dState_dt)
class is(oscillation)
  dState_dt = self
  dState_dt%U(1) = -self%f * self%U(2)
  dState_dt%U(2) =  self%f * self%U(1)
endselect
return
endfunction dOscillation_dt
\end{lstlisting}

Listing \ref{list:oscillation_t} reports the implementation of the oscillation residuals function: it is very clear and concise. Moreover, comparing this listing with the equation \ref{eq:oscillation} the close correspondence between the mathematical formulation and Fortran implementation is evident.

\paragraph{add method, an example of oscillation symmetric operator}

As a prototype of the operators overloading let us consider the \emph{add} operator, it being a prototype of symmetric operators, the implementation of which is presented in listing \ref{list:oscillation_add}.

\begin{lstlisting}[firstnumber=1,style=code,caption={implementation of the \emph{oscillation integrand} add operator},label={list:oscillation_add}]
function add_oscillation(lhs, rhs) result(opr)
class(oscillation), intent(IN) :: lhs ! Left hand side.
class(integrand),   intent(IN) :: rhs ! Right hand side.
class(integrand), allocatable  :: opr ! Operator result.
allocate(oscillation :: opr)
select type(opr)
class is(oscillation)
  opr = lhs
  select type(rhs)
  class is (oscillation)
    opr%U = lhs%U + rhs%U
  endselect
endselect
return
endfunction add_Oscillation
\end{lstlisting}
It is very simple and clear: firstly all the auxiliary data are copied into the operator result, then the state vector of the result is populated with the addiction between the state vectors of the left-hand-side and right-hand-side. This is very intuitive from the mathematical point of view and it helps to reduce implementation errors. Similar implementations are possible for all the other operators necessary to define a valid \emph{intregrand} ADT concrete extension.

\paragraph{assignment of an oscillation object}

The assignment overloading of the \emph{oscillation} type is the last key-method that enforces the conciseness of the FOODIE aware implementation. Listing \ref{list:oscillation_assign} reports the implementation of the assignment overloading. Essentially, to all the data members of the left-hand-side are assigned the values of the corresponding right-hand-side. Notably, for the assignment of the state vector and of the previous time steps solution array we take advantage of the automatic re-allocation of the left-hand-side variables when they are not allocated or allocated differently from the right-hand-side, that is a Fortran 2003 feature. In spite its simplicity, the assignment overloading is a key-method enabling the usage of FOODIE solver: effectively, the assignment between two \emph{integrand} ADT variables is ubiquitous into the solvers implementations, see equation \ref{eq:RK} for example.

\begin{lstlisting}[firstnumber=1,style=code,caption={implementation of the \emph{oscillation integrand} assignment},label={list:oscillation_assign}]
subroutine oscillation_assign_oscillation(lhs, rhs)
class(oscillation), intent(INOUT) :: lhs ! Left hand side.
class(integrand),   intent(IN)    :: rhs ! Right hand side.
select type(rhs)
class is (oscillation)
  lhs%dims = rhs%dims
  lhs%f = rhs%f
  if (allocated(rhs%U)) lhs%U = rhs%U
endselect
return
endsubroutine oscillation_assign_oscillation
\end{lstlisting}

\paragraph{FOODIE numerical integration}

Using the above discussed \emph{oscillation} type it is very easy to solve IVP \ref{eq:oscillation} by means of FOODIE library. Listing \ref{list:oscillation_leapfrog} presents the numerical integration of system \ref{eq:oscillation} by means of the Leapfrog RAW-filtered method. In the example, the integration is performed with $10^4$ steps with a fixed $\Delta t=10^2$ until the time $t=10^6$ is reached. The example shows also that for starting a multi-step scheme such as the Leapfrog one a lower-oder or equivalent order one-scheme is necessary: in the example the first 2 steps are computed by means of one-step TVD/SSP Runge-Kutta 2-stages schemes. Note that the memory registers for storing the Runge-Kutta stages and the RAW filter displacement must be handled by the client code. Listing \ref{list:oscillation_leapfrog} demonstrates how it is simple, clear and concise to solve a IVP by FOODIE solvers. Moreover, it proves how it is simple and effective to apply different solvers in a coupled algorithm, that greatly simplify the development of new hybrid solvers for self-adaptive time step size.

\begin{lstlisting}[firstnumber=1,style=code,caption={numerical integration of the \emph{oscillation} system by means of Leapfrog RAW-filtered method},label={list:oscillation_leapfrog}]
use foodie, only: leapfrog_integrator, tvd_runge_kutta_integrator
type(leapfrog_integrator)        :: lf_integrator ! Leapfrog integrator.
type(tvd_runge_kutta_integrator) :: rk_integrator ! Runge-Kutta integrator.
type(oscillation)                :: rk_stage(1:2) ! Runge-Kutta stages.
type(oscillation)                :: previous(1:2) ! Previous time steps solution.
type(oscillation)                :: oscillator    ! Oscillation field.
type(oscillation)                :: filter        ! Filter displacement.
integer                          :: step          ! Time steps counter.
real                             :: Dt            ! Time step.
call lf_integrator%init()
call rk_integrator%init(stages=2)
call oscillator%init(initial_state=[0.0,1.0], f=10e4, steps=2)
Dt = 100.0
do step=1, 10000
  if (2>=step) then
    call rk_integrator%integrate(U=oscillator, stage=rk_stage, Dt=Dt, t=step*Dt)
    previous(step) = oscillator
  else
    call lf_integrator%integrate(U=oscillator, previous=previous, filter=filter, Dt=Dt, &
                                 t=step*Dt)
  endif
enddo
call print_results(U=oscillator)
\end{lstlisting}



\subsubsection{Adams-Bashforth}

Table \ref{tab:oscillation_errors_ab} summarizes the Adams-Bashforth error analysis. As expected, the Adams-Bashforth 1 step solution, that reverts back to the explicit forward Euler one, is unstable for all the $\Delta t$ exercised.

The expected observed orders of accuracy for the Adams-Bashforth solvers using 2, 3 and 4 time steps tend to 1.5, 2.5 and 3.5 that are in agreement with the expected formal order of 2, 3 and 4, respectively. Comparing the errors of the finest time resolution, i.e. $\Delta t=100$, we find that the L2 norm decreases of the 2 orders of magnitude as the solver's accuracy increases by 1 order. This also means that fixing a tolerance on the errors, the higher is the solver's accuracy the larger is the time resolution available. As an example, assuming that admissible errors are of $O(10^{-2})$ with the 4-steps solver we can use $\Delta t=625$ performing $N_s=t_final/625$ numerical integration steps, whereas using a 3-steps solvers we must adopt $\Delta t=100$ performing $6.25 \times N_s$ numerical integration steps instead of $N_s$. Considering that the computational costs is only slightly affected by the number of previous time steps considered\footnote{Recalling equation \ref{eq:AB} one can observe that there is only one new evaluation of the residuals function $R$ independently of the previous time steps considered. Thus, the computational costs is affected only by the increasing number of residuals summations, the costs of which are typically negligible with respect the cost of $R$ evaluation.}, the accuracy order has strong impact on the overall numerical efficiency: to improve the numerical efficiency reducing the computational costs, the usage of high order Adams-Bashforth solvers with larger time steps should be preferred instead of low order solvers with smaller time steps.

\begin{table}[!ht]
  \centering
  \caption{Oscillation test: errors analysis of explicit Adams-Bashforth solvers}\label{tab:oscillation_errors_ab}
  \begin{subtable}[b]{0.40\textwidth}
    \centering
    \caption{1 step}\label{tab:oscillation-ab-1}
    \resizebox{1.00\textwidth}{!}{%
    \begin{tabular}{ccccc}
      \toprule
      {\sc Time Step} & {\sc Error X} & {\sc Error Y} & {\sc Order X} & {\sc Order Y} \\
      \hline
      5000.0          &  0.840E+10    &  0.706E+10    & /             & /             \\
      2500.0          &  0.503E+06    &  0.570E+06    & 14.03         & 13.60         \\
      1250.0          &  0.289E+04    &  0.272E+04    &  7.45         &  7.71         \\
       625.0          &  0.239E+03    &  0.232E+03    &  3.59         &  3.55         \\
       320.0          &  0.737E+02    &  0.722E+02    &  1.76         &  1.74         \\
       100.0          &  0.250E+02    &  0.247E+02    &  0.93         &  0.92         \\
      \bottomrule
    \end{tabular}}
  \end{subtable}\quad%
  \begin{subtable}[b]{0.40\textwidth}
    \centering
    \caption{2 steps}\label{tab:oscillation-ab-2}
    \resizebox{1.00\textwidth}{!}{%
    \begin{tabular}{ccccc}
      \toprule
      {\sc Time Step} & {\sc Error X} & {\sc Error Y} & {\sc Order X} & {\sc Order Y} \\
      \hline
      5000.0          &  0.596E+03    &  0.583E+03    & /             & /             \\
      2500.0          &  0.221E+02    &  0.218E+02    & 4.75          & 4.74          \\
      1250.0          &  0.764E+01    &  0.769E+01    & 1.53          & 1.50          \\
       625.0          &  0.265E+01    &  0.268E+01    & 1.53          & 1.52          \\
       320.0          &  0.968E+00    &  0.981E+00    & 1.51          & 1.50          \\
       100.0          &  0.169E+00    &  0.171E+00    & 1.50          & 1.50          \\
      \bottomrule
    \end{tabular}}
  \end{subtable}\\
  \begin{subtable}[b]{0.40\textwidth}
    \centering
    \caption{3 steps}\label{tab:oscillation-ab-3}
    \resizebox{1.00\textwidth}{!}{%
    \begin{tabular}{ccccc}
      \toprule
      {\sc Time Step} & {\sc Error X} & {\sc Error Y} & {\sc Order X} & {\sc Order Y} \\
      \hline
      5000.0          &  0.857E+01    &  0.854E+01    & /             & /             \\
      2500.0          &  0.391E+01    &  0.386E+01    & 1.13          & 1.14          \\
      1250.0          &  0.825E+00    &  0.814E+00    & 2.24          & 2.25          \\
       625.0          &  0.150E+00    &  0.148E+00    & 2.46          & 2.46          \\
       320.0          &  0.282E-01    &  0.278E-01    & 2.49          & 2.49          \\
       100.0          &  0.154E-02    &  0.152E-02    & 2.50          & 2.50          \\
      \bottomrule
    \end{tabular}}
  \end{subtable}\quad%
  \begin{subtable}[b]{0.40\textwidth}
    \centering
    \caption{4 steps}\label{tab:oscillation-ab-4}
    \resizebox{1.00\textwidth}{!}{%
    \begin{tabular}{ccccc}
      \toprule
      {\sc Time Step} & {\sc Error X} & {\sc Error Y} & {\sc Order X} & {\sc Order Y} \\
      \hline
      5000.0          &  0.128E+07    &  0.143E+07    & /             & /             \\
      2500.0          &  0.106E+01    &  0.107E+01    & 20.21         & 20.34         \\
      1250.0          &  0.967E-01    &  0.981E-01    &  3.45         &  3.45         \\
       625.0          &  0.859E-02    &  0.871E-02    &  3.49         &  3.49         \\
       320.0          &  0.827E-03    &  0.838E-03    &  3.50         &  3.50         \\
       100.0          &  0.141E-04    &  0.143E-04    &  3.50         &  3.50         \\
      \bottomrule
    \end{tabular}}
  \end{subtable}\\
\end{table}

Figure \ref{fig:results-oscillation-adams-bashforth} shows, for each solver exercised, the $X(t)$ and $Y(t)$ solution for $t \in [0, 10^6]$: the plots into the figure report a global overview of the solution for all the instants considered (left subplots) and a detailed zoom over the last instants of the integration (right subplots) for evaluating the numerical errors accumulation. For the sake of clarity, the strongly unstable solutions are omitted into the subplots concerning the final integration instants, namely the solutions for large $\Delta t$. Figure \ref{fig:results-oscillation-adams-bashforth} emphasizes the instability generation for some pairs steps number/$\Delta t$. The 2 and 4 steps solutions are instable for $\Delta t=5000 \rightarrow f*\Delta t=0.5$. On the contrary, the 3 steps solution is stable, but the amplitude is dumped and the solution vanishes as the integration proceeds. The 2 and 4 steps solutions show a phase error that decreases as the time resolution increases, whereas 3 steps solution has null phase error.

\begin{figure}[!ht]
  \centering
  \begin{subfigure}[b]{0.45\textwidth}
    \centering
    \includegraphics[width=1.00\textwidth]{errors-analysis/oscillation/errors_analysis-oscillation-adams-bashforth-1.png}
    \caption{1 step}\label{fig:results-oscillation-adams-bashforth-1}
  \end{subfigure}\quad%
  \begin{subfigure}[b]{0.45\textwidth}
    \centering
    \includegraphics[width=1.00\textwidth]{errors-analysis/oscillation/errors_analysis-oscillation-adams-bashforth-2.png}
    \caption{2 steps}\label{fig:results-oscillation-adams-bashforth-2}
  \end{subfigure}\\
  \begin{subfigure}[b]{0.45\textwidth}
    \centering
    \includegraphics[width=1.00\textwidth]{errors-analysis/oscillation/errors_analysis-oscillation-adams-bashforth-3.png}
    \caption{3 steps}\label{fig:results-oscillation-adams-bashforth-3}
  \end{subfigure}\quad%
  \begin{subfigure}[b]{0.45\textwidth}
    \centering
    \includegraphics[width=1.00\textwidth]{errors-analysis/oscillation/errors_analysis-oscillation-adams-bashforth-4.png}
    \caption{4 steps}\label{fig:results-oscillation-adams-bashforth-4}
  \end{subfigure}
  \caption{Oscillation equations solutions computed by means of Adams-Bashforth solvers}\label{fig:results-oscillation-adams-bashforth}
\end{figure}



\subsubsection{Adams-Bashforth-Moulton}

Table \ref{tab:oscillation_errors_abm} summarizes the Adams-Bashforth-Moulton error analysis. The same considerations done for the Adams-Bashforth solutions can repeated for the Adams-Bashforth-Moulton ones, thus they are omitted for the sake of conciseness. An interesting result concerns the observed errors: the $O(10^{-2})$ error is now obtained with $\Delta t=1250$ for the 4-steps solver, thus it is 2 times faster than the corresponding Adams-Bashforth 4-step solver. Considering that the computational costs of a single Adams-Bashforth-Moulton step is only slightly greater than the corresponding Adamas-Bashforth step, the efficiency increasing is not negligible.

\begin{table}[!ht]
  \centering
  \caption{Oscillation test: errors analysis of predictor-corrector Adams-Bashforth-Moulton solvers}\label{tab:oscillation_errors_abm}
  \begin{subtable}[b]{0.40\textwidth}
    \centering
    \caption{1 step}\label{tab:oscillation-abm-1}
    \resizebox{1.00\textwidth}{!}{%
    \begin{tabular}{ccccc}
      \toprule
      {\sc Time Step} & {\sc Error X} & {\sc Error Y} & {\sc Order X} & {\sc Order Y} \\
      \hline
      5000.0          &  0.241E+20    &  0.266E+20    & /             & /             \\
      2500.0          &  0.664E+11    &  0.716E+11    & 28.44         & 28.47         \\
      1250.0          &  0.952E+06    &  0.100E+07    & 16.09         & 16.12         \\
       625.0          &  0.413E+04    &  0.407E+04    &  7.85         &  7.95         \\
       320.0          &  0.387E+03    &  0.383E+03    &  3.54         &  3.53         \\
       100.0          &  0.145E+03    &  0.145E+03    &  0.84         &  0.83         \\
      \bottomrule
    \end{tabular}}
  \end{subtable}\quad%
  \begin{subtable}[b]{0.40\textwidth}
    \centering
    \caption{2 steps}\label{tab:oscillation-abm-2}
    \resizebox{1.00\textwidth}{!}{%
    \begin{tabular}{ccccc}
      \toprule
      {\sc Time Step} & {\sc Error X} & {\sc Error Y} & {\sc Order X} & {\sc Order Y} \\
      \hline
      5000.0          &  0.704E+01    &  0.701E+01    & /             & /             \\
      2500.0          &  0.392E+01    &  0.395E+01    & 0.84          & 0.83          \\
      1250.0          &  0.148E+01    &  0.150E+01    & 1.40          & 1.39          \\
       625.0          &  0.526E+00    &  0.534E+00    & 1.49          & 1.49          \\
       320.0          &  0.193E+00    &  0.196E+00    & 1.50          & 1.50          \\
       100.0          &  0.338E-01    &  0.342E-01    & 1.50          & 1.50          \\
      \bottomrule
    \end{tabular}}
  \end{subtable}\\
  \begin{subtable}[b]{0.40\textwidth}
    \centering
    \caption{3 steps}\label{tab:oscillation-abm-3}
    \resizebox{1.00\textwidth}{!}{%
    \begin{tabular}{ccccc}
      \toprule
      {\sc Time Step} & {\sc Error X} & {\sc Error Y} & {\sc Order X} & {\sc Order Y} \\
      \hline
      5000.0          &  0.457E+01    &  0.464E+01    & /             & /             \\
      2500.0          &  0.656E+00    &  0.654E+00    & 2.80          & 2.83          \\
      1250.0          &  0.100E+00    &  0.987E-01    & 2.71          & 2.73          \\
       625.0          &  0.169E-01    &  0.167E-01    & 2.56          & 2.56          \\
       320.0          &  0.314E-02    &  0.310E-02    & 2.52          & 2.51          \\
       100.0          &  0.171E-03    &  0.169E-03    & 2.50          & 2.50          \\
      \bottomrule
    \end{tabular}}
  \end{subtable}\quad%
  \begin{subtable}[b]{0.40\textwidth}
    \centering
    \caption{4 steps}\label{tab:oscillation-abm-4}
    \resizebox{1.00\textwidth}{!}{%
    \begin{tabular}{ccccc}
      \toprule
      {\sc Time Step} & {\sc Error X} & {\sc Error Y} & {\sc Order X} & {\sc Order Y} \\
      \hline
      5000.0          &  0.229E+01    &  0.225E+01    & /             & /             \\
      2500.0          &  0.119E+00    &  0.118E+00    & 4.26          & 4.25          \\
      1250.0          &  0.825E-02    &  0.833E-02    & 3.85          & 3.83          \\
       625.0          &  0.671E-03    &  0.681E-03    & 3.62          & 3.61          \\
       320.0          &  0.631E-04    &  0.640E-04    & 3.53          & 3.53          \\
       100.0          &  0.107E-05    &  0.108E-05    & 3.51          & 3.51          \\
      \bottomrule
    \end{tabular}}
  \end{subtable}\\
\end{table}

Figure \ref{fig:results-oscillation-adams-bashforth-moulton} shows similar plots of figure \ref{fig:results-oscillation-adams-bashforth} above discussed. Differently from the Adams-Bashforth class, the amplitude damping feature is now possessed by the 2-steps solver, see plot \ref{fig:results-oscillation-adams-bashforth-moulton-2}, while all solutions show phase errors that decrease as the time resolution increases.

\begin{figure}[!ht]
  \centering
  \begin{subfigure}[b]{0.45\textwidth}
    \centering
    \includegraphics[width=1.00\textwidth]{errors-analysis/oscillation/errors_analysis-oscillation-adams-bashforth-moulton-1.png}
    \caption{1 step}\label{fig:results-oscillation-adams-bashforth-moulton-1}
  \end{subfigure}\quad%
  \begin{subfigure}[b]{0.45\textwidth}
    \centering
    \includegraphics[width=1.00\textwidth]{errors-analysis/oscillation/errors_analysis-oscillation-adams-bashforth-moulton-2.png}
    \caption{2 steps}\label{fig:results-oscillation-adams-bashforth-moulton-2}
  \end{subfigure}\\
  \begin{subfigure}[b]{0.45\textwidth}
    \centering
    \includegraphics[width=1.00\textwidth]{errors-analysis/oscillation/errors_analysis-oscillation-adams-bashforth-moulton-3.png}
    \caption{3 steps}\label{fig:results-oscillation-adams-bashforth-moulton-3}
  \end{subfigure}\quad%
  \begin{subfigure}[b]{0.45\textwidth}
    \centering
    \includegraphics[width=1.00\textwidth]{errors-analysis/oscillation/errors_analysis-oscillation-adams-bashforth-moulton-4.png}
    \caption{4 steps}\label{fig:results-oscillation-adams-bashforth-moulton-4}
  \end{subfigure}
  \caption{Oscillation equations solutions computed by means of Adams-Bashforth-Moulton solvers}\label{fig:results-oscillation-adams-bashforth-moulton}
\end{figure}
\clearpage



\subsubsection{Adams-Moulton}

Table \ref{tab:oscillation_errors_am} summarizes the Adams-Moulton error analysis. The implicit Adams-Moulton solvers behave much like the Adams-Bashforth-Moulton ones: they have similar errors and observed orders for the same formal order considered. However, the implicit Adams-Moulton class uses one less step with respect the corresponding Adams-Bashforth-Moulton class: this could lead to the promise of higher computational efficiency. Notwithstanding, for solving the implicit non-linearity embedded into the Adams-Moulton solvers an iterative algorithm must be employed: for the results presented, a 5 iterations of \emph{fixed point algorithm} have been computed. This strongly reduces the eventual gain of computational efficiency with respect the Adams-Bashforth-Moulton class.

\begin{table}[!ht]
  \centering
  \caption{Oscillation test: errors analysis of explicit Adams-Moulton solvers; the implicit non-linearity is solved by 5 iterations of \emph{fixed point algorithm}}\label{tab:oscillation_errors_am}
  \begin{subtable}[b]{0.40\textwidth}
    \centering
    \caption{1 step}\label{tab:oscillation-am-1}
    \resizebox{1.00\textwidth}{!}{%
    \begin{tabular}{ccccc}
      \toprule
      {\sc Time Step} & {\sc Error X} & {\sc Error Y} & {\sc Order X} & {\sc Order Y} \\
      \hline
      5000.0          &  0.840E+10    &  0.706E+10    & /             & /             \\
      2500.0          &  0.503E+06    &  0.570E+06    & 14.03         & 13.60         \\
      1250.0          &  0.289E+04    &  0.272E+04    &  7.45         &  7.71         \\
       625.0          &  0.239E+03    &  0.232E+03    &  3.59         &  3.55         \\
       320.0          &  0.737E+02    &  0.722E+02    &  1.76         &  1.74         \\
       100.0          &  0.250E+02    &  0.247E+02    &  0.93         &  0.92         \\
      \bottomrule
    \end{tabular}}
  \end{subtable}\quad%
  \begin{subtable}[b]{0.40\textwidth}
    \centering
    \caption{2 steps}\label{tab:oscillation-am-2}
    \resizebox{1.00\textwidth}{!}{%
    \begin{tabular}{ccccc}
      \toprule
      {\sc Time Step} & {\sc Error X} & {\sc Error Y} & {\sc Order X} & {\sc Order Y} \\
      \hline
      5000.0          &  0.108E+02    &  0.109E+02    & /             & /             \\
      2500.0          &  0.412E+01    &  0.419E+01    & 1.39          & 1.38          \\
      1250.0          &  0.148E+01    &  0.150E+01    & 1.48          & 1.48          \\
       625.0          &  0.527E+00    &  0.533E+00    & 1.49          & 1.49          \\
       320.0          &  0.193E+00    &  0.196E+00    & 1.50          & 1.50          \\
       100.0          &  0.338E-01    &  0.342E-01    & 1.50          & 1.50          \\
      \bottomrule
    \end{tabular}}
  \end{subtable}\\
  \begin{subtable}[b]{0.40\textwidth}
    \centering
    \caption{3 steps}\label{tab:oscillation-am-3}
    \resizebox{1.00\textwidth}{!}{%
    \begin{tabular}{ccccc}
      \toprule
      {\sc Time Step} & {\sc Error X} & {\sc Error Y} & {\sc Order X} & {\sc Order Y} \\
      \hline
      5000.0          &  0.390E+01    &  0.384E+01    & /             & /             \\
      2500.0          &  0.551E+00    &  0.544E+00    & 2.82          & 2.82          \\
      1250.0          &  0.947E-01    &  0.934E-01    & 2.54          & 2.54          \\
       625.0          &  0.167E-01    &  0.165E-01    & 2.50          & 2.50          \\
       320.0          &  0.313E-02    &  0.309E-02    & 2.50          & 2.50          \\
       100.0          &  0.171E-03    &  0.169E-03    & 2.50          & 2.50          \\
      \bottomrule
    \end{tabular}}
  \end{subtable}\quad%
  \begin{subtable}[b]{0.40\textwidth}
    \centering
    \caption{4 steps}\label{tab:oscillation-am-4}
    \resizebox{1.00\textwidth}{!}{%
    \begin{tabular}{ccccc}
      \toprule
      {\sc Time Step} & {\sc Error X} & {\sc Error Y} & {\sc Order X} & {\sc Order Y} \\
      \hline
      5000.0          &  0.983E+00    &  0.999E+00    & /             & /             \\
      2500.0          &  0.832E-01    &  0.845E-01    & 3.56          & 3.56          \\
      1250.0          &  0.736E-02    &  0.746E-02    & 3.50          & 3.50          \\
       625.0          &  0.652E-03    &  0.660E-03    & 3.50          & 3.50          \\
       320.0          &  0.626E-04    &  0.635E-04    & 3.50          & 3.50          \\
       100.0          &  0.107E-05    &  0.108E-05    & 3.50          & 3.50          \\
      \bottomrule
    \end{tabular}}
  \end{subtable}\\
\end{table}

Figure \ref{fig:results-oscillation-adams-moulton} shows similar plots of figure \ref{fig:results-oscillation-adams-bashforth-moulton} above discussed: there are not relevant differences between the 2 classes of solvers.

\begin{figure}[!ht]
  \centering
  \begin{subfigure}[b]{0.45\textwidth}
    \centering
    \includegraphics[width=1.00\textwidth]{errors-analysis/oscillation/errors_analysis-oscillation-adams-moulton-0.png}
    \caption{0 step}\label{fig:results-oscillation-adams-moulton-0}
  \end{subfigure}\quad%
  \begin{subfigure}[b]{0.45\textwidth}
    \centering
    \includegraphics[width=1.00\textwidth]{errors-analysis/oscillation/errors_analysis-oscillation-adams-moulton-1.png}
    \caption{1 step}\label{fig:results-oscillation-adams-moulton-1}
  \end{subfigure}\\
  \begin{subfigure}[b]{0.45\textwidth}
    \centering
    \includegraphics[width=1.00\textwidth]{errors-analysis/oscillation/errors_analysis-oscillation-adams-moulton-2.png}
    \caption{2 steps}\label{fig:results-oscillation-adams-moulton-2}
  \end{subfigure}\quad%
  \begin{subfigure}[b]{0.45\textwidth}
    \centering
    \includegraphics[width=1.00\textwidth]{errors-analysis/oscillation/errors_analysis-oscillation-adams-moulton-3.png}
    \caption{3 steps}\label{fig:results-oscillation-adams-moulton-3}
  \end{subfigure}
  \caption{Oscillation equations solutions computed by means of Adams-Moulton solvers}\label{fig:results-oscillation-adams-moulton}
\end{figure}



\subsubsection{Leapfrog}

The Leapfrog solutions are in agreement with the expected results: both unfiltered and RAW-filtered solutions show an observed order of accuracy that tends to the formal $2^{nd}$ order. The two solutions are almost the same.

\begin{table}[!ht]
  \centering
  \caption{Oscillation test: errors analysis of explicit Leapfrog solvers}\label{tab:oscillation_errors_lf}
  \begin{subtable}[b]{0.40\textwidth}
    \centering
    \caption{Unfiltered}\label{tab:oscillation-leapfrog}
    \resizebox{1.00\textwidth}{!}{%
    \begin{tabular}{ccccc}
      \toprule
      {\sc Time Step} & {\sc Error X} & {\sc Error Y} & {\sc Order X} & {\sc Order Y} \\
      \hline
      5000.0          &  0.156E+02    &  0.156E+02    & /             & /             \\
      2500.0          &  0.849E+01    &  0.846E+01    & 0.87          & 0.88          \\
      1250.0          &  0.300E+01    &  0.303E+01    & 1.50          & 1.48          \\
       625.0          &  0.106E+01    &  0.107E+01    & 1.51          & 1.50          \\
       320.0          &  0.387E+00    &  0.392E+00    & 1.50          & 1.50          \\
       100.0          &  0.676E-01    &  0.685E-01    & 1.50          & 1.50          \\
      \bottomrule
    \end{tabular}}
  \end{subtable}\quad%
  \begin{subtable}[b]{0.40\textwidth}
    \centering
    \caption{RAW-filtered}\label{tab:oscillation-leapfrog-raw}
    \resizebox{1.00\textwidth}{!}{%
    \begin{tabular}{ccccc}
      \toprule
      {\sc Time Step} & {\sc Error X} & {\sc Error Y} & {\sc Order X} & {\sc Order Y} \\
      \hline
      5000.0          &  0.156E+02    &  0.156E+02    & /             & /             \\
      2500.0          &  0.855E+01    &  0.852E+01    & 0.86          & 0.87          \\
      1250.0          &  0.303E+01    &  0.305E+01    & 1.50          & 1.48          \\
       625.0          &  0.107E+01    &  0.108E+01    & 1.51          & 1.50          \\
       320.0          &  0.390E+00    &  0.395E+00    & 1.50          & 1.50          \\
       100.0          &  0.685E-01    &  0.692E-01    & 1.50          & 1.50          \\
      \bottomrule
    \end{tabular}}
  \end{subtable}\\
\end{table}

\begin{figure}[!ht]
  \centering
  \begin{subfigure}[b]{0.45\textwidth}
    \centering
    \includegraphics[width=1.00\textwidth]{errors-analysis/oscillation/errors_analysis-oscillation-leapfrog.png}
    \caption{Unfiltered}\label{fig:results-oscillation-leapfrog-unfiltered}
  \end{subfigure}\quad%
  \begin{subfigure}[b]{0.45\textwidth}
    \centering
    \includegraphics[width=1.00\textwidth]{errors-analysis/oscillation/errors_analysis-oscillation-leapfrog-raw.png}
    \caption{RAW-filtered}\label{fig:results-oscillation-leapfrog-raw}
  \end{subfigure}
  \caption{Oscillation equations solutions computed by means of Leapfrog solvers}\label{fig:results-oscillation-leapfrog}
\end{figure}



\subsubsection{Low Storage Runge-Kutta}

\begin{table}[!ht]
  \centering
  \caption{Oscillation test: errors analysis of explicit Low Storage Runge-Kutta solvers}\label{tab:oscillation_errors_ls_rk}
  \begin{subtable}[b]{0.40\textwidth}
    \centering
    \caption{1 stage}\label{tab:oscillation-ls-rk-1}
    \resizebox{1.00\textwidth}{!}{%
    \begin{tabular}{ccccc}
      \toprule
      {\sc Time Step} & {\sc Error X} & {\sc Error Y} & {\sc Order X} & {\sc Order Y} \\
      \hline
      5000.0          &  0.840E+10    &  0.706E+10    & /             & /             \\
      2500.0          &  0.503E+06    &  0.570E+06    & 14.03         & 13.60         \\
      1250.0          &  0.289E+04    &  0.272E+04    &  7.45         &  7.71         \\
       625.0          &  0.239E+03    &  0.232E+03    &  3.59         &  3.55         \\
       320.0          &  0.737E+02    &  0.722E+02    &  1.76         &  1.74         \\
       100.0          &  0.250E+02    &  0.247E+02    &  0.93         &  0.92         \\
      \bottomrule
    \end{tabular}}
  \end{subtable}\quad%
  \begin{subtable}[b]{0.40\textwidth}
    \centering
    \caption{5 stages}\label{tab:oscillation-ls-rk-5}
    \resizebox{1.00\textwidth}{!}{%
    \begin{tabular}{ccccc}
      \toprule
      {\sc Time Step} & {\sc Error X} & {\sc Error Y} & {\sc Order X} & {\sc Order Y} \\
      \hline
      5000.0          &  0.120E+00    &  0.122E+00    & /             & /             \\
      2500.0          &  0.106E-01    &  0.107E-01    & 3.51          & 3.51          \\
      1250.0          &  0.935E-03    &  0.947E-03    & 3.50          & 3.50          \\
       625.0          &  0.826E-04    &  0.836E-04    & 3.50          & 3.50          \\
       320.0          &  0.793E-05    &  0.803E-05    & 3.50          & 3.50          \\
       100.0          &  0.135E-06    &  0.137E-06    & 3.50          & 3.50          \\
      \bottomrule
    \end{tabular}}
  \end{subtable}\\
  \begin{subtable}[b]{0.40\textwidth}
    \centering
    \caption{6 stages}\label{tab:oscillation-ls-rk-6}
    \resizebox{1.00\textwidth}{!}{%
    \begin{tabular}{ccccc}
      \toprule
      {\sc Time Step} & {\sc Error X} & {\sc Error Y} & {\sc Order X} & {\sc Order Y} \\
      \hline
      5000.0          &  0.979E-01    &  0.994E-01    & /             & /             \\
      2500.0          &  0.876E-02    &  0.888E-02    & 3.48          & 3.48          \\
      1250.0          &  0.776E-03    &  0.786E-03    & 3.50          & 3.50          \\
       625.0          &  0.686E-04    &  0.695E-04    & 3.50          & 3.50          \\
       320.0          &  0.659E-05    &  0.667E-05    & 3.50          & 3.50          \\
       100.0          &  0.112E-06    &  0.114E-06    & 3.50          & 3.50          \\
      \bottomrule
    \end{tabular}}
  \end{subtable}\quad%
  \begin{subtable}[b]{0.40\textwidth}
    \centering
    \caption{7 stages}\label{tab:oscillation-ls-rk-7}
    \resizebox{1.00\textwidth}{!}{%
    \begin{tabular}{ccccc}
      \toprule
      {\sc Time Step} & {\sc Error X} & {\sc Error Y} & {\sc Order X} & {\sc Order Y} \\
      \hline
      5000.0          &  0.238E-01    &  0.240E-01    & /             & /             \\
      2500.0          &  0.203E-02    &  0.205E-02    & 3.55          & 3.55          \\
      1250.0          &  0.177E-03    &  0.180E-03    & 3.51          & 3.51          \\
       625.0          &  0.156E-04    &  0.158E-04    & 3.50          & 3.50          \\
       320.0          &  0.150E-05    &  0.152E-05    & 3.50          & 3.50          \\
       100.0          &  0.269E-07    &  0.273E-07    & 3.46          & 3.46          \\
      \bottomrule
    \end{tabular}}
  \end{subtable}\\
  \begin{subtable}[b]{0.40\textwidth}
    \centering
    \caption{12 stages}\label{tab:oscillation-ls-rk-12}
    \resizebox{1.00\textwidth}{!}{%
    \begin{tabular}{ccccc}
      \toprule
      {\sc Time Step} & {\sc Error X} & {\sc Error Y} & {\sc Order X} & {\sc Order Y} \\
      \hline
      5000.0          &  0.195E-01    &  0.198E-01    & /             & /             \\
      2500.0          &  0.175E-02    &  0.177E-02    & 3.48          & 3.48          \\
      1250.0          &  0.155E-03    &  0.157E-03    & 3.50          & 3.50          \\
       625.0          &  0.137E-04    &  0.139E-04    & 3.50          & 3.50          \\
       320.0          &  0.132E-05    &  0.133E-05    & 3.50          & 3.50          \\
       100.0          &  0.225E-07    &  0.228E-07    & 3.50          & 3.50          \\
      \bottomrule
    \end{tabular}}
  \end{subtable}\quad%
  \begin{subtable}[b]{0.40\textwidth}
    \centering
    \caption{13 stages}\label{tab:oscillation-ls-rk-13}
    \resizebox{1.00\textwidth}{!}{%
    \begin{tabular}{ccccc}
      \toprule
      {\sc Time Step} & {\sc Error X} & {\sc Error Y} & {\sc Order X} & {\sc Order Y} \\
      \hline
      5000.0          &  0.795E-02    &  0.805E-02    & /             & /             \\
      2500.0          &  0.703E-03    &  0.712E-03    & 3.50          & 3.50          \\
      1250.0          &  0.621E-04    &  0.629E-04    & 3.50          & 3.50          \\
       625.0          &  0.549E-05    &  0.556E-05    & 3.50          & 3.50          \\
       320.0          &  0.527E-06    &  0.534E-06    & 3.50          & 3.50          \\
       100.0          &  0.899E-08    &  0.911E-08    & 3.50          & 3.50          \\
      \bottomrule
    \end{tabular}}
  \end{subtable}\\
  \begin{subtable}[b]{0.40\textwidth}
    \centering
    \caption{14 stages}\label{tab:oscillation-ls-rk-14}
    \resizebox{1.00\textwidth}{!}{%
    \begin{tabular}{ccccc}
      \toprule
      {\sc Time Step} & {\sc Error X} & {\sc Error Y} & {\sc Order X} & {\sc Order Y} \\
      \hline
      5000.0          &  0.849E-02    &  0.860E-02    & /             & /             \\
      2500.0          &  0.750E-03    &  0.759E-03    & 3.50          & 3.50          \\
      1250.0          &  0.662E-04    &  0.671E-04    & 3.50          & 3.50          \\
       625.0          &  0.585E-05    &  0.593E-05    & 3.50          & 3.50          \\
       320.0          &  0.562E-06    &  0.569E-06    & 3.50          & 3.50          \\
       100.0          &  0.959E-08    &  0.972E-08    & 3.50          & 3.50          \\
      \bottomrule
    \end{tabular}}
  \end{subtable}\quad%
\end{table}

\begin{figure}[!ht]
  \centering
  \begin{subfigure}[b]{0.45\textwidth}
    \centering
    \includegraphics[width=1.00\textwidth]{errors-analysis/oscillation/errors_analysis-oscillation-ls-runge-kutta-1.png}
    \caption{1 stage}\label{fig:results-oscillation-ls-runge-kutta-1}
  \end{subfigure}\quad%
  \begin{subfigure}[b]{0.45\textwidth}
    \centering
    \includegraphics[width=1.00\textwidth]{errors-analysis/oscillation/errors_analysis-oscillation-ls-runge-kutta-5.png}
    \caption{5 stages}\label{fig:results-oscillation-ls-runge-kutta-5}
  \end{subfigure}
  \caption{Oscillation equations solutions computed by means of low storage Runge-Kutta solvers}\label{fig:results-oscillation-ls-runge-kutta-1-5}
\end{figure}

\begin{figure}[!ht]
  \centering
  \begin{subfigure}[b]{0.45\textwidth}
    \centering
    \includegraphics[width=1.00\textwidth]{errors-analysis/oscillation/errors_analysis-oscillation-ls-runge-kutta-6.png}
    \caption{6 stages}\label{fig:results-oscillation-ls-runge-kutta-6}
  \end{subfigure}\quad%
  \begin{subfigure}[b]{0.45\textwidth}
    \centering
    \includegraphics[width=1.00\textwidth]{errors-analysis/oscillation/errors_analysis-oscillation-ls-runge-kutta-7.png}
    \caption{7 stages}\label{fig:results-oscillation-ls-runge-kutta-7}
  \end{subfigure}
  \begin{subfigure}[b]{0.45\textwidth}
    \centering
    \includegraphics[width=1.00\textwidth]{errors-analysis/oscillation/errors_analysis-oscillation-ls-runge-kutta-12.png}
    \caption{12 stages}\label{fig:results-oscillation-ls-runge-kutta-12}
  \end{subfigure}\quad%
  \begin{subfigure}[b]{0.45\textwidth}
    \centering
    \includegraphics[width=1.00\textwidth]{errors-analysis/oscillation/errors_analysis-oscillation-ls-runge-kutta-13.png}
    \caption{13 stages}\label{fig:results-oscillation-ls-runge-kutta-13}
  \end{subfigure}
  \begin{subfigure}[b]{0.45\textwidth}
    \centering
    \includegraphics[width=1.00\textwidth]{errors-analysis/oscillation/errors_analysis-oscillation-ls-runge-kutta-14.png}
    \caption{14 stages}\label{fig:results-oscillation-ls-runge-kutta-14}
  \end{subfigure}\quad%
  \caption{Oscillation equations solutions computed by means of low storage Runge-Kutta solvers}\label{fig:results-oscillation-ls-runge-kutta-6-14}
\end{figure}



\subsubsection{TVD/SSP Runge-Kutta}

\begin{table}[!ht]
  \centering
  \caption{Oscillation test: errors analysis of explicit TVD/SSP Runge-Kutta}\label{tab:oscillation_errors_tvd_rk}
  \begin{subtable}[b]{0.40\textwidth}
    \centering
    \caption{1 stage}\label{tab:oscillation-tvd-rk-1}
    \resizebox{1.00\textwidth}{!}{%
    \begin{tabular}{ccccc}
      \toprule
      {\sc Time Step} & {\sc Error X} & {\sc Error Y} & {\sc Order X} & {\sc Order Y} \\
      \hline
      5000.0          &  0.840E+10    &  0.706E+10    & /             & /             \\
      2500.0          &  0.503E+06    &  0.570E+06    & 14.03         & 13.60         \\
      1250.0          &  0.289E+04    &  0.272E+04    &  7.45         &  7.71         \\
       625.0          &  0.239E+03    &  0.232E+03    &  3.59         &  3.55         \\
       320.0          &  0.737E+02    &  0.722E+02    &  1.76         &  1.74         \\
       100.0          &  0.250E+02    &  0.247E+02    &  0.93         &  0.92         \\
      \bottomrule
    \end{tabular}}
  \end{subtable}\quad%
  \begin{subtable}[b]{0.40\textwidth}
    \centering
    \caption{2 stages}\label{tab:oscillation-tvd-rk-2}
    \resizebox{1.00\textwidth}{!}{%
    \begin{tabular}{ccccc}
      \toprule
      {\sc Time Step} & {\sc Error X} & {\sc Error Y} & {\sc Order X} & {\sc Order Y} \\
      \hline
      5000.0          &  0.316E+02    &  0.319E+02    & /             & /             \\
      2500.0          &  0.892E+01    &  0.894E+01    & 1.83          & 1.84          \\
      1250.0          &  0.301E+01    &  0.305E+01    & 1.57          & 1.55          \\
       625.0          &  0.106E+01    &  0.107E+01    & 1.51          & 1.51          \\
       320.0          &  0.387E+00    &  0.392E+00    & 1.50          & 1.50          \\
       100.0          &  0.676E-01    &  0.685E-01    & 1.50          & 1.50          \\
      \bottomrule
    \end{tabular}}
  \end{subtable}\\
  \begin{subtable}[b]{0.40\textwidth}
    \centering
    \caption{3 stages}\label{tab:oscillation-tvd-rk-3}
    \resizebox{1.00\textwidth}{!}{%
    \begin{tabular}{ccccc}
      \toprule
      {\sc Time Step} & {\sc Error X} & {\sc Error Y} & {\sc Order X} & {\sc Order Y} \\
      \hline
      5000.0          &  0.255E+01    &  0.252E+01    & /             & /             \\
      2500.0          &  0.523E+00    &  0.516E+00    & 2.28          & 2.29          \\
      1250.0          &  0.944E-01    &  0.931E-01    & 2.47          & 2.47          \\
       625.0          &  0.167E-01    &  0.165E-01    & 2.50          & 2.50          \\
       320.0          &  0.314E-02    &  0.310E-02    & 2.50          & 2.50          \\
       100.0          &  0.171E-03    &  0.169E-03    & 2.50          & 2.50          \\
      \bottomrule
    \end{tabular}}
  \end{subtable}\quad%
  \begin{subtable}[b]{0.40\textwidth}
    \centering
    \caption{5 stages}\label{tab:oscillation-tvd-rk-5}
    \resizebox{1.00\textwidth}{!}{%
    \begin{tabular}{ccccc}
      \toprule
      {\sc Time Step} & {\sc Error X} & {\sc Error Y} & {\sc Order X} & {\sc Order Y} \\
      \hline
      5000.0          &  0.139E+00    &  0.141E+00    & /             & /             \\
      2500.0          &  0.122E-01    &  0.124E-01    & 3.50          & 3.50          \\
      1250.0          &  0.108E-02    &  0.110E-02    & 3.50          & 3.50          \\
       625.0          &  0.956E-04    &  0.969E-04    & 3.50          & 3.50          \\
       320.0          &  0.937E-05    &  0.949E-05    & 3.47          & 3.47          \\
       100.0          &  0.512E-06    &  0.519E-06    & 2.50          & 2.50          \\
      \bottomrule
    \end{tabular}}
  \end{subtable}
\end{table}

\begin{figure}[!ht]
  \centering
  \begin{subfigure}[b]{0.45\textwidth}
    \centering
    \includegraphics[width=1.00\textwidth]{errors-analysis/oscillation/errors_analysis-oscillation-tvd-runge-kutta-1.png}
    \caption{1 stage}\label{fig:results-oscillation-tvd-runge-kutta-1}
  \end{subfigure}\quad%
  \begin{subfigure}[b]{0.45\textwidth}
    \centering
    \includegraphics[width=1.00\textwidth]{errors-analysis/oscillation/errors_analysis-oscillation-tvd-runge-kutta-2.png}
    \caption{2 stages}\label{fig:results-oscillation-tvd-runge-kutta-2}
  \end{subfigure}\\
  \begin{subfigure}[b]{0.45\textwidth}
    \centering
    \includegraphics[width=1.00\textwidth]{errors-analysis/oscillation/errors_analysis-oscillation-tvd-runge-kutta-3.png}
    \caption{3 stages}\label{fig:results-oscillation-tvd-runge-kutta-3}
  \end{subfigure}\quad%
  \begin{subfigure}[b]{0.45\textwidth}
    \centering
    \includegraphics[width=1.00\textwidth]{errors-analysis/oscillation/errors_analysis-oscillation-tvd-runge-kutta-5.png}
    \caption{5 stages}\label{fig:results-oscillation-tvd-runge-kutta-5}
  \end{subfigure}\\
  \caption{Oscillation equations solutions computed by means of TVD/SSP Runge-Kutta solvers}\label{fig:results-oscillation-tvd-runge-kutta}
\end{figure}



\subsubsection{Embedded Runge-Kutta}

\begin{table}[!ht]
  \centering
  \caption{Oscillation test: errors analysis of explicit embedded Runge-Kutta solvers}\label{tab:oscillation_errors_emd_rk}
  \begin{subtable}[b]{0.40\textwidth}
    \centering
    \caption{7 stages}\label{tab:oscillation-emd-rk-7}
    \resizebox{1.00\textwidth}{!}{%
    \begin{tabular}{ccccc}
      \toprule
      {\sc Mean Time Step} & {\sc Error X} & {\sc Error Y} & {\sc Order X} & {\sc Order Y} \\
      \hline
      7352.9               &  0.626E-01    &  0.622E-01    & /             & /             \\
      3759.4               &  0.482E-02    &  0.480E-02    & 3.82          & 3.82          \\
      2272.7               &  0.793E-03    &  0.786E-03    & 3.59          & 3.59          \\
      1379.3               &  0.393E-04    &  0.390E-04    & 6.02          & 6.02          \\
       618.8               &  0.216E-05    &  0.213E-05    & 3.62          & 3.62          \\
       356.0               &  0.252E-06    &  0.249E-06    & 3.89          & 3.88          \\
      \bottomrule
    \end{tabular}}
  \end{subtable}\\
\end{table}

\begin{figure}[!ht]
  \centering
  \begin{subfigure}[b]{0.45\textwidth}
    \centering
    \includegraphics[width=1.00\textwidth]{errors-analysis/oscillation/errors_analysis-oscillation-emd-runge-kutta-7.png}
    \caption{7 stages}\label{fig:results-oscillation-emd-runge-kutta-7}
  \end{subfigure}\quad%
  \caption{Oscillation equations solutions computed by means of embedded Runge-Kutta solvers}\label{fig:results-oscillation-emd-runge-kutta-1-7}
\end{figure}




\clearpage

%\section{Benchmarks on parallel frameworks}\label{sec:parallel}

As aforementioned, FOODIE is unaware of any parallel paradigms or programming models the client codes adopt. As a consequence, the parallel performances measurements presented into this section are aimed only to prove that FOODIE environment does not destroy the parallel scaling of the baseline code implemented without FOODIE.

To obtain such a prove, the 1D Euler PDE system described previously is numerically solved with FOODIE-aware test codes that in turn exploit parallel resources by means:

\begin{itemize}
  \item CoArray Fortran (CAF) model, for shared and distributed memory architectures;
  \item OpenMP directive-based model, for only shared memory architectures;
  \end{itemize}

In order to measure the performances of the parallel-enabled FOODIE tests, the \emph{strong} and \emph{weak} scaling have been considered. For the strong scaling the \emph{speedup} has been computed:

\begin{equation}
  speedup(N, k) = \frac{T_{serial}(N)}{T_{parallel}(N, k)}
  \label{eq:strong-scaling-speedup}
\end{equation}
where $N$ is the problem size, $K$ the number of parallel resources used (namely the physical cores), $T_{serial}$ is the CPU time of the serial code and $T_{parallel}$ the one of the parallel code. The ideal speedup is linear with slop equals to 1. The efficiency correlated to the strong scaling measurement is defined as:

\begin{equation}
  efficiency(N, k) = \frac{speedup(N, k)}{k}
  \label{eq:strong-scaling-efficiency}
\end{equation}
The maximum ideal efficiency is obviously the unity.

For the of weak scaling measurement the \emph{sizeup} has been computed:

\begin{equation}
  sizeup(N, k) = \frac{N_k}{N_1} \cdot \frac{T_{serial}(N_1)}{T_{parallel}(N_k, k)}
  \label{eq:weak-scaling-sizeup}
\end{equation}
where $N_1$ is the minimum size considered and $N_K$ is the size used for the test computed with $k$ parallel resources. If $N_K$ is scaled proportional to $N_1$, the ideal sizeup is again linear and if $N_k = k \cdot N_1$ the slope is again linear. The efficiency correlated to the weak scaling is defined as:

\begin{equation}
  efficiency(N, k) = \frac{sizeup(N, k)}{k}
  \label{eq:weak-scaling-efficiency}
\end{equation}
The maximum ideal efficiency is obviously the unity.

The same 1D Euler PDEs problem is also solved by parallel-enabled codes that are not based on FOODIE: their solutions provide a reference for measuring the effect of FOODIE abstraction on the parallel scaling.

\subsection{CAF benchmark}\label{subsec:caf}

This subsection reports the parallel scaling analysis of Euler 1D test programs (with and without FOODIE) being parallelized by means of CoArrays Fortran (CAF) model. This parallel model is based on the concept of \emph{coarray} introduced into the Fortran 2008 standard: the array syntax is extended introducing the so called \emph{codimension} that is a new arrays indexing. Essentially, a CAF enabled code is designed to be replicated a certain number of times and all copies, conventionally named \emph{images}, are executed asynchronously. Each image has its own set of data (memory) and the codimension indexes are used to access to the (remote) memory of the other images. The CAF approach allows the implementation of Partitioned Global Address Space (PGAS) model following the SPMD (single program, multiple data) parallelization paradigm. The programmer must take care of defining the coarray variables and of synchronizing the images when necessary. This approach requires the refactoring of legacy serial codes, but it allows the exploitation of both shared and distributed memory architectures. Moreover, it is a standard feature of Fortran (2008), thus it is not chained to any particular compiler vendors extension.

\begin{figure}[!ht]
  \centering
  \begin{subfigure}[b]{0.80\textwidth}
    \centering
    \includegraphics[width=1.00\textwidth]{caf_benchmark/euler-1D-caf/strong-scaling-comparison.png}
    \caption{Strong scaling, number of cells 240000}\label{fig:strong-scaling-caf}
  \end{subfigure}\\
  \begin{subfigure}[b]{0.80\textwidth}
    \centering
    \includegraphics[width=1.00\textwidth]{caf_benchmark/euler-1D-caf/weak-scaling-comparison.png}
    \caption{Weak scaling, minimum number of cells 24000}\label{fig:weak-scaling-caf}
  \end{subfigure}\\
  \caption{Scaling efficiency with CAF programming model}\label{fig:scaling-caf}
\end{figure}

The benchmarks shows in this section have been done on a \emph{dual Intel(R) Xeon(R) CPU X5650} exacores workstation for a total of 12 physical cores, coupled with 24GB of RAM. In order to perform an accurate analysis 4 different codes have considered:

\begin{itemize}
  \item FOODIE-aware codes:
    \begin{itemize}
      \item serial code;
      \item CAF-enabled code;
      \end{itemize}
  \item procedural codes without using FOODIE library:
    \begin{itemize}
      \item serial code;
      \item CAF-enabled code;
      \end{itemize}
  \end{itemize}

These codes (see \ref{subsec:euler-1D-CAF-API} for the implementation details) have been compiled by means of the GNU gfortran compiler v5.2.0 coupled with OpenCoarrays v1.1.0\footnote{OpenCoarrays is an open-source software project for developing, porting and tuning transport layers that support coarray Fortran (CAF) compilers, see \cite{opencoarrays}.}.

The Euler conservation laws are integrated for 30 time steps by means of the TVD RK(5,4) solver: the measured CPU time used for computing the scaling efficiencies is the average of the 30 integrations, thus representing the mean CPU time for computing one time step integration.

For the strong scaling, the benchmark has been conducted with 240000 finite volumes. Figure \ref{fig:strong-scaling-caf} summarizes the strong scaling analysis: it shows that FOODIE-based code scales similarly to the baseline code without FOODIE.

For the weak scaling the minimum size is 24000 finite volumes and the size is scaled linearly with the CAF images, thus $N_{12} = 288000$ cells. Figure \ref{fig:weak-scaling-caf} summarizes the weak scaling analysis and it essentially confirms that FOODIE-based code scales similarly to the baseline code without FOODIE.

Both strong and weak scaling analysis point out that for the computing architecture considered the parallel scaling is reasonable up to 12 cores, the efficiency being always satisfactory.

To complete the comparison, the absolute CPU-time consumed by the two families of codes (with and without FOODIE) must be considered. Table \ref{tab:caf-results} summarizes the benchmarks results. As shown, procedural and FOODIE-aware codes consume a very similar CPU-time for both the strong and the weak benchmarks. The same results are shown in figure \ref{fig:cpu-time-caf}. These results prove that the abstraction of FOODIE environment does not degrade the computational efficiency.

\begin{table}[!ht]
  \centering
  \caption{caf benchmarks results\label{tab:caf-results}}
  \begin{subtable}[b]{0.80\textwidth}
    \centering
    \caption{Strong benchmarks, number of cells 240000\label{tab:caf-results-strong}}
    \resizebox{1.00\textwidth}{!}{%
    \begin{tabular}{ccccc}
      {\sc Number of caf threads} & \multicolumn{4}{c}{\sc CPU time for 1 time step integration}              \\
      \hline
                                     & FOODIE serial & FOODIE parallel & procedural serial & procedural parallel \\
      \cmidrule{2-5}
      1                              & 3.2970        & 3.3297          & 3.0049            & 3.0563              \\
      2                              & /             & 1.6536          & /                 & 1.5686              \\
      4                              & /             & 0.8515          & /                 & 1.8116              \\
      8                              & /             & 0.4296          & /                 & 0.4130              \\
      12                             & /             & 0.3094          & /                 & 0.2839              \\
      \hline
    \end{tabular}}
  \end{subtable}\\
  \begin{subtable}[b]{0.80\textwidth}
    \centering
    \caption{Weak benchmarks, minimum number of cells 24000\label{tab:caf-results-weak}}
    \resizebox{1.00\textwidth}{!}{%
    \begin{tabular}{cccccc}
      {\sc Number of caf threads} & {\sc Number of Cells} & \multicolumn{4}{c}{\sc CPU time for 1 time step integration} \\
      \hline
                                     &                       & FOODIE serial & FOODIE parallel & procedural serial & procedural parallel \\
      \cmidrule{3-6}
      1                              & 24000                 & 0.3105        & 0.3159          & 0.3089            & 0.3133              \\
      2                              & 48000                 & /             & 0.3209          & /                 & 0.3185              \\
      4                              & 96000                 & /             & 0.3384          & /                 & 0.3269              \\
      8                              & 192000                & /             & 0.3449          & /                 & 0.3369              \\
      12                             & 288000                & /             & 0.4291          & /                 & 0.3657              \\
      \hline
    \end{tabular}}
  \end{subtable}
\end{table}

\begin{figure}[!ht]
  \centering
  \begin{subfigure}[b]{0.40\textwidth}
    \centering
    \includegraphics[width=1.00\textwidth]{caf_benchmark/euler-1D-caf/strong-cpu-time-comparison.png}
    \caption{Strong benchmark, number of cells 240000}\label{fig:strong-cpu-time-caf}
  \end{subfigure}\quad%
  \begin{subfigure}[b]{0.40\textwidth}
    \centering
    \includegraphics[width=1.00\textwidth]{caf_benchmark/euler-1D-caf/weak-cpu-time-comparison.png}
    \caption{Weak benchmark, minimum number of cells 24000}\label{fig:weak-cpu-time-caf}
  \end{subfigure}\\
  \caption{CPU time consumed with caf programming model}\label{fig:cpu-time-caf}
\end{figure}



\subsection{OpenMP benchmark}\label{subsec:openmp}

This subsection reports the parallel scaling analysis of Euler 1D test programs (with and without FOODIE) being parallelized by means of OpenMP directives-based paradigm. This parallel model is based on the concept of \emph{threads}: an OpenMP enabled code start a single (master) threaded program and, at run-time, it is able to generate a team of (many) threads that work concurrently on the parallelized parts of the code, thus reducing the CPU time necessary for completing such parts. The parallelization is made by means of \emph{directives} explicitly inserted by the programmer: the communications between threads are automatically handled by the compiler (through the provided OpenMP library used as back-end). OpenMP parallel paradigm is not a standard feature of Fortran, rather it is an extension provided by the compiler vendors. This parallel paradigm constitutes an effective and easy approach for parallelizing legacy serial codes, however its usage is limited to shared memory architectures because all threads must have access to the same memory.

\begin{figure}[!ht]
  \centering
  \begin{subfigure}[b]{0.80\textwidth}
    \centering
    \includegraphics[width=1.00\textwidth]{openmp_benchmark/euler-1D-openmp/strong-scaling-comparison.png}
    \caption{Strong scaling, number of cells 240000}\label{fig:strong-scaling-openmp}
  \end{subfigure}\\
  \begin{subfigure}[b]{0.80\textwidth}
    \centering
    \includegraphics[width=1.00\textwidth]{openmp_benchmark/euler-1D-openmp/weak-scaling-comparison.png}
    \caption{Weak scaling, minimum number of cells 24000}\label{fig:weak-scaling-openmp}
  \end{subfigure}\\
  \caption{Scaling efficiency with OpenMP programming model}\label{fig:scaling-openmp}
\end{figure}

The benchmarks shown in this section have been done on a \emph{dual Intel(R) Xeon(R) CPU X5650} exacores workstation for a total of 12 physical cores, coupled with 24GB of RAM. In order to perform an accurate analysis 4 different codes have considered:

\begin{itemize}
  \item FOODIE-aware codes:
    \begin{itemize}
      \item serial code;
      \item OpenMP-enabled code;
      \end{itemize}
  \item procedural codes without using FOODIE library:
    \begin{itemize}
      \item serial code;
      \item OpenMP-enabled code;
      \end{itemize}
  \end{itemize}

These codes (see \ref{subsec:euler-1D-OpenMP-API} for the implementation details) have been compiled by means of the GNU gfortran compiler v5.2.0 with \emph{-O2 -fopenmp} compilation flags.

The Euler conservation laws are integrated for 30 time steps by means of the TVD RK(5,4) solver: the measured CPU time used for computing the scaling efficiencies is the average of the 30 integrations, thus representing the mean CPU time for computing one time step integration.

For the strong scaling, the benchmark has been conducted with 240000 finite volumes. Figure \ref{fig:strong-scaling-openmp} summarizes the strong scaling analysis: it shows that FOODIE-based code scales similarly to the baseline code without FOODIE.

For the weak scaling the minimum size is 24000 finite volumes and the size is scaled linearly with the OpenMP threads, thus $N_{12} = 288000$ cells. Figure \ref{fig:weak-scaling-openmp} summarizes the weak scaling analysis and it essentially confirms that FOODIE-based code scales similarly to the baseline code without FOODIE.

Both strong and weak scaling analysis point out that for the computing architecture considered the parallel scaling is reasonable up to 8 cores: using 12 cores the measured efficiencies become unsatisfactory, reducing below the 60\%.

To complete the comparison, the absolute CPU-time consumed by the two families of codes (with and without FOODIE) must be considered. Table \ref{tab:openmp-results} summarizes the benchmarks results. As shown, procedural and FOODIE-aware codes consume a very similar CPU-time for both the strong and the weak benchmarks. The same results are shown in figure \ref{fig:cpu-time-openmp}. These results prove that the abstraction of FOODIE environment does not degrade the computational efficiency.

\begin{table}[!ht]
  \centering
  \caption{OpenMP benchmarks results\label{tab:openmp-results}}
  \begin{subtable}[b]{0.80\textwidth}
    \centering
    \caption{Strong benchmarks, number of cells 240000\label{tab:openmp-results-strong}}
    \resizebox{1.00\textwidth}{!}{%
    \begin{tabular}{ccccc}
      {\sc Number of OpenMP threads} & \multicolumn{4}{c}{\sc CPU time for 1 time step integration}              \\
      \hline
                                     & FOODIE serial & FOODIE parallel & procedural serial & procedural parallel \\
      \cmidrule{2-5}
      1                              & 3.3466        & 3.3076          & 3.1252            & 3.0873              \\
      2                              & /             & 1.8166          & /                 & 1.7765              \\
      4                              & /             & 0.9798          & /                 & 1.0085              \\
      8                              & /             & 0.5192          & /                 & 0.5055              \\
      12                             & /             & 0.4847          & /                 & 0.4748              \\
      \hline
    \end{tabular}}
  \end{subtable}\\
  \begin{subtable}[b]{0.80\textwidth}
    \centering
    \caption{Weak benchmarks, minimum number of cells 24000\label{tab:openmp-results-weak}}
    \resizebox{1.00\textwidth}{!}{%
    \begin{tabular}{cccccc}
      {\sc Number of OpenMP threads} & {\sc Number of Cells} & \multicolumn{4}{c}{\sc CPU time for 1 time step integration} \\
      \hline
                                     &                       & FOODIE serial & FOODIE parallel & procedural serial & procedural parallel \\
      \cmidrule{3-6}
      1                              & 24000                 & 0.3171        & 0.3162          & 0.3089            & 0.3111              \\
      2                              & 48000                 & /             & 0.3492          & /                 & 0.3854              \\
      4                              & 96000                 & /             & 0.3666          & /                 & 0.4069              \\
      8                              & 192000                & /             & 0.3862          & /                 & 0.4142              \\
      12                             & 288000                & /             & 0.5727          & /                 & 0.6142              \\
      \hline
    \end{tabular}}
  \end{subtable}
\end{table}

\begin{figure}[!ht]
  \centering
  \begin{subfigure}[b]{0.40\textwidth}
    \centering
    \includegraphics[width=1.00\textwidth]{openmp_benchmark/euler-1D-openmp/strong-cpu-time-comparison.png}
    \caption{Strong benchmark, number of cells 240000}\label{fig:strong-cpu-time-openmp}
  \end{subfigure}\quad%
  \begin{subfigure}[b]{0.40\textwidth}
    \centering
    \includegraphics[width=1.00\textwidth]{openmp_benchmark/euler-1D-openmp/weak-cpu-time-comparison.png}
    \caption{Weak benchmark, minimum number of cells 24000}\label{fig:weak-cpu-time-openmp}
  \end{subfigure}\\
  \caption{CPU time consumed with OpenMP programming model}\label{fig:cpu-time-openmp}
\end{figure}




\clearpage

%\section{Concluding Remarks and Perspectives}\label{sec:conclusions}

{\color{red} To be written.}



\appendix

%\section{Euler 1D Parallel Tests API}

In subsections \ref{subsec:caf} and \ref{subsec:openmp} it has been proved that FOODIE usage does not penalize the parallel scaling of an equivalent procedural code implemented without FOODIE. To this aim, we have solved the Euler's conservation laws (in one dimension) by means of FOODIE: as a matter of fact, Euler 1D PDEs constitutes a complex test retaining many difficulties of real applications, but it is still simple enough to serve as benchmark test. In this section we report the implementation details of the codes developed to solve (with serial and parallel models) the Euler 1D PDEs system.

\subsection{Euler 1D baseline API}\label{subsec:euler-1D-API}

The 1D Euler PDEs system is a non linear, hyperbolic (inviscid) system of conservation laws for compressible gas dynamics, that reads
\begin{equation}
\begin{matrix}
U_t = R(U)  \Leftrightarrow U_t = F(U)_x \\
U = \begin{bmatrix}
\rho \\
\rho u \\
\rho E
\end{bmatrix}\;\;\;
F(U) = \begin{bmatrix}
\rho u \\
\rho u^2 + p \\
\rho u H
\end{bmatrix}
\end{matrix}
\label{eq:euler-1D}
\end{equation}
where $\rho$ is the density, $u$ is the velocity, $p$ the pressure, $E$ the total internal specific energy and $H$ the total specific enthalpy. The PDEs system must completed with the proper initial and boundary conditions. Moreover, an ideal (thermally and calorically perfect) gas is considered
\begin{equation}
\begin{matrix}
R = c_p - c_v \\
\gamma = \frac{c_p}{c_v}\\
e = c_v T \\
h = c_p T
\end{matrix}
\end{equation}
where $R$ is the gas constant, $c_p$ and $c_v$ are the specific heats at constant pressure and volume (respectively), $e$ is the internal energy, $h$ is the internal enthalpy and *T* is the temperature. The following addition equations of state hold:
\begin{equation}
\begin{matrix}
T = \frac{p}{\rho R} \\
E = \rho e + \frac{1}{2} \rho u^2 \\
H = \rho h + \frac{1}{2} \rho u^2 \\
a = \sqrt{\frac{\gamma p}{\rho}}
\end{matrix}
\end{equation}

An extension of the above Euler system is considered allowing the modelling of a multi-fluid mixture of different gas (with different physical characteristics). The well known Standard Thermodynamic Model is used to model the gas mixture replacing the density with the density fraction of each specie composing the mixture. This led to the following system:
\begin{equation}
\begin{matrix}
U_t = R(U)  \Leftrightarrow U_t = F(U)_x \\
U = \begin{bmatrix}
\rho_s \\
\rho u \\
\rho E
\end{bmatrix}\;\;\;
F(U) = \begin{bmatrix}
\rho_s u \\
\rho u^2 + p \\
\rho u H
\end{bmatrix}\;\;\; for\; s=1,2,...N_s \\
\rho = \sum_{s=1}^{N_s}\rho_s \\
c_p = \sum_{s=1}^{N_S} \frac{\rho_s}{\rho} c_{p,s} \quad  c_v = \sum_{s=1}^{N_S} \frac{\rho_s}{\rho} c_{v,s}
\end{matrix}
\end{equation}
where $N_s$ is the number of initial species composing the gas mixture.

\subsubsection{Memory organization}
The finite volume, Godunov's like approach is employed. Essentially, the method of Lines is used to decouple the space operator from the time one. Firstly, the space operator (the residual function of equation \ref{eq:euler-1D}) is numerically solved in order to reduce the original PDEs system to a system of ODEs that is then integrated over time by means of FOODIE solvers. Here we omit the details of the numerical models, interested readers can see \cite{}. On the contrary, some details on the memory organization is necessary to explaining the implemented API.

The conservative variables are co-located at the cell center. The cell and (inter)faces numeration is as shown in listing \ref{list:grid}.
\begin{lstlisting}[firstnumber=1,style=codesimple,caption={Numerical grid organization},label={list:grid}]
        cell   (inter)faces
         |            |
         v            v
 |----|----|-...-|----|---|---|---|-...-|-----|-----|-----|-...-|-------|-------|
 |1-Ng|2-Ng| ... |-1  | 0 | 1 | 2 | ... |  Ni | Ni+1| Ni+1| ... |Ni+Ng-1| Ni+Ng |
 |----|----|-...-|----|---|---|---|-...-|-----|-----|-----|-...-|-------|-------|
0-Ng                 -1   0   1   2    Ni-1   Ni                      Ni+Ng
\end{lstlisting}
In listing \ref{list:grid} $Ni$ is the number of finite volumes (cells) used for discretizing the domain and $Ng$ is the number of ghost cells used for imposing the left and right boundary conditions (for a total of $2Ng$ cells).
For each cell the conservative variables must be stored: this is done by means of of rank 2 array where the first index refers to the conservative variables (densities, momentum or energy) while the second index refers to the space location, namely the cell index.

The most CPU time consuming part of a finite volume scheme is the fluxes computation across the cells interfaces. Such a computation corresponds to a loop over all the cells interfaces. Listing \ref{list:pseudo-fluxes} shows a pseudo-code example of such a computation.

\begin{lstlisting}[firstnumber=1,style=code,caption={Pseudo-code example of fluxes computation},label={list:pseudo-fluxes}]
do i=0, Ni
  F(:, i) = compute_fluxes(U(:, i), U(:, i+1))
enddo
\end{lstlisting}

In the pseudo-code of listing \ref{list:pseudo-fluxes} it has been emphasized that the fluxes across an interface depends on the cells at left and right of the interface itself. The key point for the parallelization of such an algorithm is to compute the fluxes concurrently using as much as possible the parallel resources provided by the running architecture. As a consequence, the above showed loop over the whole domain is split into sub-domains (explicitly or implicitly accordingly to the parallel model adopted) and the fluxed of each sub-domain are computed concurrently.

\subsubsection{The serial API}

The conservative variables of 1D Euler's system can be easily implemented as a \emph{FOODIE integrand field} defining a concrete extension of the FOODIE \emph{integrand} type. Listing \ref{list:euler_1D_type} reports the implementation of such an integrand field that is contained into the tests suite shipped within the FOODIE library.

\begin{lstlisting}[firstnumber=1,style=code,caption={implementation of the \emph{Euler 1D integrand} type},label={list:euler_1D_type}]
type, extends(integrand) :: euler_1D
  private
  integer(I_P)                   :: ord=0     ! Space accuracy formal order.
  integer(I_P)                   :: Ni=0      ! Space dimension.
  integer(I_P)                   :: Ng=0      ! Number of ghost cells for boundary conditions handling.
  integer(I_P)                   :: Ns=0      ! Number of initial species.
  integer(I_P)                   :: Nc=0      ! Number of conservative variables, Ns+2.
  integer(I_P)                   :: Np=0      ! Number of primitive variables, Ns+4.
  real(R_P)                      :: Dx=0._R_P ! Space step.
  type(weno_interpolator_upwind) :: weno      ! WENO interpolator.
  real(R_P),    allocatable      :: U(:,:)    ! Integrand (state) variables, whole physical domain [1:Nc,1:Ni].
  real(R_P),    allocatable      :: cp0(:)    ! Specific heat cp of initial species [1:Ns].
  real(R_P),    allocatable      :: cv0(:)    ! Specific heat cv of initial species [1:Ns].
  character(:), allocatable      :: BC_L      ! Left boundary condition type.
  character(:), allocatable      :: BC_R      ! Right boundary condition type.
  integer(I_P)                   :: me=0      ! ID of this\_image().
  integer(I_P)                   :: we=0      ! Number of CAF images used.
  contains
    ! auxiliary methods
    procedure, pass(self), public :: init
    procedure, pass(self), public :: destroy
    procedure, pass(self), public :: output
    procedure, pass(self), public :: dt => compute_dt
    ! ADT integrand deferred methods
    procedure, pass(self), public :: t => dEuler_dt
    procedure, pass(lhs),  public :: local_error => euler_local_error
    procedure, pass(lhs),  public :: integrand_multiply_integrand => euler_multiply_euler
    procedure, pass(lhs),  public :: integrand_multiply_real => euler_multiply_real
    procedure, pass(rhs),  public :: real_multiply_integrand => real_multiply_euler
    procedure, pass(lhs),  public :: add => add_euler
    procedure, pass(lhs),  public :: sub => sub_euler
    procedure, pass(lhs),  public :: assign_integrand => euler_assign_euler
    procedure, pass(lhs),  public :: assign_real => euler_assign_real
    ! private methods
    procedure, pass(self), private :: primitive2conservative
    procedure, pass(self), private :: conservative2primitive
    procedure, pass(self), private :: synchronize
    procedure, pass(self), private :: impose_boundary_conditions
    procedure, pass(self), private :: reconstruct_interfaces_states
    procedure, pass(self), private :: riemann_solver
    final                          :: finalize
endtype euler_1D
\end{lstlisting}

\subsection{Euler 1D CAF Enabled API}\label{subsec:euler-1D-CAF-API}

\subsection{Euler 1D OpenMP Enabled API}\label{subsec:euler-1D-OpenMP-API}


\bibliographystyle{mycpc2}
\bibliography{Reference}

\end{document}
