%&pdflatex
\documentclass[pdftex,preprint,3p,times,numbers]{elsarticle}

\journal{Computer Physics Communications}

%\usepackage{moreverb}

\usepackage{hyperref}
\hypersetup{pdfborder={0 0 0}}

\usepackage{float}
\usepackage{wrapfig}
\usepackage{caption}
\usepackage{subcaption}
\usepackage{multirow}
\graphicspath{{images/}}

\usepackage[pdftex,usenames]{xcolor}

\usepackage{booktabs}
%\usepackage{colortbl}
\usepackage{multirow}

\usepackage{amsmath}
\usepackage{amssymb}
\usepackage[utf8x]{inputenc}
\usepackage[T1]{fontenc}

\usepackage{xspace}

\usepackage{xcolor}
\definecolor{Maroon}{cmyk}{0,0.87,0.68,0.32}
\definecolor{RoyalBlue}{cmyk}{1,0.50,0,0}
\definecolor{gray}{cmyk}{0.01,0.01,0.01,0.01}
\usepackage{listings}
\lstdefinelanguage{MyFortran}[08]{Fortran}{morecomment=[l]{\#},morestring=[d]',morekeywords={procedure,pass,deferred,non_overridable,generic,class,is}}
\lstdefinestyle{code}{%
  basicstyle=\footnotesize,%
  backgroundcolor=\color{gray},%
  language=MyFortran,%
  captionpos=b,%
  columns=fixed,%
  keepspaces=true,%
  xleftmargin=10pt,%
  numbers=none,%
  numberstyle={\tiny},%
  keywordstyle=\color{RoyalBlue},%
  % stringstyle={\sffamily},%
  texcl=true,%
  upquote=true,%
  commentstyle=\color{Maroon}%
}

\lstdefinestyle{codesimple}{%
  basicstyle=\footnotesize,%
  backgroundcolor=\color{gray},%
  captionpos=b,%
  columns=fixed,%
  keepspaces=true,%
  xleftmargin=10pt,%
  numbers=none,%
}

\DeclareSymbolFont{extraup}{U}{zavm}{m}{n}
\DeclareMathSymbol{\vardiamond}{\mathalpha}{extraup}{87}
\definecolor{OMP}{RGB}{255,127,0}
\definecolor{MPI}{RGB}{0,127,255}
\definecolor{HYB}{RGB}{127,0,255}

\DeclareGraphicsExtensions{.pdf,.png,.jpg,.bmp,.mps}
\newcommand{\citeh}[1]{\citeauthor{#1} \citenum{#1}}

\renewcommand{\thesubfigure}{\Alph{subfigure}}

\begin{document}

\begin{frontmatter}

\title{WenOOF, WENO interpolation Object Oriented Fortran library based on Abstract Calculus Pattern}

\author[insean]{Zaghi S.\corref{cor1}\fnref{sz}}
\ead{stefano.zaghi@cnr.it}
\fntext[sz]{Ph. D., Aerospace Engineer, Research Scientist, Dept. of Computational Hydrodynamics at CNR-INSEAN.}
\address[insean]{CNR-INSEAN, Istituto Nazionale per Studi ed Esperienze di Architettura Navale, Via di Vallerano 139, Rome, Italy, 00128}
\cortext[cor1]{Corresponding author}

\author[dima]{Rossi G.\fnref{gr}}
\ead{giacomo.rossi@uniroma1.it}
\fntext[gr]{Ph.D., Space Engineer, Research Fellow, Dept. of Mechanical and Aerospace Engineering at Universit\'{a} di Roma ``Sapienza''}
\address[dima]{Dipartimento di Ingegneria Meccanica e Aerospaziale, Universit\'{a} di Roma ``Sapienza'', Via Eudossiana 18, Rome, Italy, 00184}

\begin{abstract}
  The (numerical) solution of partial differential equations (PDEs) can lead to discontinuous solutions. Weighted Essentially Non-Oscillatory shock capturing schemes can handle such solutions and show desiderable properties as Total Variation Diminishing and high order of accuracy in smooth regions. The present paper is the first \emph{manifesto} of WenOOF, a library aimed to implement WENO reconstruction and interpolation schemes by means of a clear, concise and efficient \emph{abstract} interface. WenOOF, meaning WENO interpolation Object Oriented Fortran library, has manifolds aims: to provide a set to built-in numerical schemes that are accurate, robust, validated and efficient and to allow easy application of these schemes to (almost) all PDEs by means of an effective Abstract Calculus Pattern. The key idea is to allow the same scheme-implementation to be applied to all reconstruction/interpolation problems thus avoiding the re-implementation of the reconstruction scheme for each different conservation problem: code re-usability is consequently maximized, WenOOF being a general robust framework. Besides, the same framework also allows rapid development of new WENO schemes due to the high abstraction level of the library itself.

WenOOF is a modern Fortran library which main features are:
\begin{description}
  \item[Free] WenOOF is a free software;
  \item[OOP] WenOOF is based on Object Oriented Programming paradigm;
  \item[TDD] the WenOOF development follows the Test Driven Development software process;
  \item[Accurately documented] the WenOOF documentation is based on high quality, first class solutions embedding detailed (mathematical) descriptions directly inside code sources; moreover, comprehensive hyper-linked documentations is also provided;
  \item[Collaborative] the development of WenOOF takes advantage of web communications, the main project being hosted on GitHub.
\end{description}

The present paper is the first announcement of WenOOF project: the current implementation is extensively discussed and its capabilities are proved by means of tests and examples.
\end{abstract}

\begin{keyword}
  Weighted Essentially Non-Oscillatory (WENO) \sep
  Partial Differential Equations (PDEs) \sep
  Object Oriented Programming (OOP) \sep
  Abstract Calculus Pattern (ACP) \sep
  Fortran
\end{keyword}

\end{frontmatter}

{\bf PROGRAM SUMMARY}

\begin{small}
\noindent
\emph{Manuscript Title:} WenOOF, WENO interpolation Object Oriented Fortran library based on Abstract Calculus Pattern \\
\emph{Authors:} Zaghi, S., Rossi, G. \\
\emph{Program title:} WenOOF \\
\emph{Journal Reference:} \\
\emph{Catalogue identifier:} \\
\emph{Licensing provisions:} GNU General Public License (GPL) v3 \\
\emph{Programming language:} Fortran (standard 2008 or newer); developed and tested with GNU gfortran 6.2 or newer \\
\emph{Computer(s) for which the program has been designed:} designed for shared-memory multi-cores workstations and for hybrid distributed/shared-memory supercomputers, but any computer system with a Fortran (2008+) compiler is suited \\
\emph{Operating system(s) for which the program has been designed:} designed for POSIX architecture and tested on GNU/Linux one \\
\emph{RAM required to execute with typical data: bytes:} $[1MB,1GB]\times core$, simulation-dependent \\
\emph{Has the code been vectorised or parallelized?:} the library is not aware of the parallel back-end, it providing a high-level models, but the provided tests suite shows parallel usage by means of MPI library and OpenMP paradigm \\
\emph{Number of processors used:} tested up to 256 \\
\emph{Supplementary material:}    \\
\emph{Keywords:} WENO, OOP, ACP, Fortran \\
\emph{CPC Library Classification:} 4.3 Differential Equations, 4.10 Interpolation, 12 Gases and Fluids \\
\emph{External routines/libraries used:} \\
\emph{CPC Program Library subprograms used:} \\
\emph{Nature of problem:} \\
Numerical integration of (general) Partial Differential Equations system \\
\emph{Solution method:} \\
\emph{Restrictions:} \\
\emph{Unusual features:} \\
\emph{Additional comments:} \\
\emph{Running time:} \\
\emph{References:} \\
% \begin{thebibliography}{0}
% \end{thebibliography}
\end{small}

\section{Introduction}\label{sec:introduction}

Interpolation is the process of deriving a simple function from a set of discrete data points so that the function passes through all the given data points (i.e. reproduces the data points exactly) and can be used to estimate data points in-between the given ones.

Interpolation is also used to simplify complicated functions by sampling data points and then interpolating them using a simpler function. Polynomials are commonly used for interpolation because they are easier to evaluate, differentiate, and integrate. Unfortunately, interpolation of order greater than one can suffer of the Gibbs' phenomenon~\cite{gibbs-b-1906} next to discontinuities.

The original idea of WENO schemes~\cite{liu-1994} is to use a convex combination of all candidate stencils (instead of using only the smoothest one as in ENO schemes~\cite{harten-1987}) to obtain high order reconstruction: this approach can obviously be extended to interpolation process, leading to an high order oscillatory free interpolation.

{\color{red} Add interpolation background and citation to interpolation related works.}


\clearpage

\section{Mathematical and Numerical Models}\label{sec:MNmodels}

Assume we have a uniform mesh $x_1, x_2, \dots x_n$ with $\Delta x = x_{n+1} - x_n$ and that we know the values of a function $u$ at all the grid points, that is $u_i = u(x_i)$ for all $i$. We would like to find an approximation of the function $u(x)$ at the point $x^*$ other than the nodes $x_i$, with $x_{i-\frac{1}{2}} < x^* < x_{i+\frac{1}{2}}$, where $x_{i-\frac{1}{2}}$ and $x_{i+\frac{1}{2}}$ are the cell interfaces.

For a $r^{th}$ order accurate interpolation, there are $r$ candidate stencils next to the target point $x^*$: we denote these stencil as $S_k$, where $k=0, \dots, r-1$ labels the stencils from the leftmost stencil to the rightmost stencil in that order. Using  the Lagrange form of the interpolation polynomial, the polynom $p_k(x)$ over the stencil $S_k$ can be written as:

\begin{equation}
  \label{eq:Lagrange}
  p_k(x^*) = \sum_{j=0}^{r-1} u_{i-r+k+j+1} \sum_{\substack{l=0 \\ l \neq j}}^{r-1} \frac{x^* - x_{i-r+k+l+1}}{x_{i-r+k+j+1} - x_{i-r+k+l+1}} = \sum_{j=0}^{r-1} a_{k,i-r+j+1} u_{i-r+k+j+1}
\end{equation}

where $a_{k,i-r+j+1}$ are the Lagrange coefficients of the stencil $S_k$.

In table~\ref{tab:polynomial_coefficients} are reported the polynomial coefficients from $r=2$ to $r=9$ for all the interpolating stencils, for $x^* = x_{i+\frac{1}{2}}$; polynomial coefficients for $x^*=x_{i-\frac{1}{2}}$ can be obtained by table~\ref{tab:polynomial_coefficients} by symmetry.

\begin{table}
  \begin{center}
    \caption{Polynomial coefficients from $r=2$ to $r=9$ for $x^*=x_{i+\frac{1}{2}}$}
    \label{tab:polynomial_coefficients}
    \begin{tabular}{ccccccccccc}
      \toprule
      $r$  &  $k$  &  $j=0$  &  $j=1$  &  $j=2$  &  $j=3$  &  $j=4$  &  $j=5$  &  $j=6$  &  $j=7$  &  $j=8$  \\
      \midrule
      9  &  0  &  $ \frac{6435}{32768}$  &  $-\frac{7293}{4096}$  &  $ \frac{58905}{8192}$  &  $-\frac{69615}{4096}$  &  $ \frac{425425}{16384}$  &  $-\frac{109395}{4096}$  &  $ \frac{153153}{8192}$  &  $-\frac{36465}{4096}$  &  $ \frac{109395}{32768}$ \\ \addlinespace
         &  1  &  $-\frac{ 429}{32768}$  &  $ \frac{ 495}{4096}$  &  $-\frac{ 4095}{8192}$  &  $ \frac{ 5005}{4096}$  &  $-\frac{ 32175}{16384}$  &  $ \frac{  9009}{4096}$  &  $-\frac{ 15015}{8192}$  &  $ \frac{ 6435}{4096}$  &  $ \frac{  6435}{32768}$ \\ \addlinespace
         &  2  &  $ \frac{  99}{32768}$  &  $-\frac{ 117}{4096}$  &  $ \frac{ 1001}{8192}$  &  $-\frac{ 1287}{4096}$  &  $ \frac{  9009}{16384}$  &  $-\frac{  3003}{4096}$  &  $ \frac{  9009}{8192}$  &  $ \frac{ 1287}{4096}$  &  $-\frac{   429}{32768}$ \\ \addlinespace
         &  3  &  $-\frac{  45}{32768}$  &  $ \frac{  55}{4096}$  &  $-\frac{  495}{8192}$  &  $ \frac{  693}{4096}$  &  $-\frac{  5775}{16384}$  &  $ \frac{  3465}{4096}$  &  $ \frac{  3465}{8192}$  &  $-\frac{  165}{4096}$  &  $ \frac{    99}{32768}$ \\ \addlinespace
         &  4  &  $ \frac{  35}{32768}$  &  $-\frac{  45}{4096}$  &  $ \frac{  441}{8192}$  &  $-\frac{  735}{4096}$  &  $ \frac{ 11025}{16384}$  &  $ \frac{  2205}{4096}$  &  $-\frac{   735}{8192}$  &  $ \frac{   63}{4096}$  &  $-\frac{    45}{32768}$ \\ \addlinespace
         &  5  &  $-\frac{  45}{32768}$  &  $ \frac{  63}{4096}$  &  $-\frac{  735}{8192}$  &  $ \frac{ 2205}{4096}$  &  $ \frac{ 11025}{16384}$  &  $-\frac{   735}{4096}$  &  $ \frac{   441}{8192}$  &  $-\frac{   45}{4096}$  &  $ \frac{    35}{32768}$ \\ \addlinespace
         &  6  &  $ \frac{  99}{32768}$  &  $-\frac{ 165}{4096}$  &  $ \frac{ 3465}{8192}$  &  $ \frac{ 3465}{4096}$  &  $-\frac{  5775}{16384}$  &  $ \frac{   693}{4096}$  &  $-\frac{   495}{8192}$  &  $ \frac{   55}{4096}$  &  $-\frac{    45}{32768}$ \\ \addlinespace
         &  7  &  $-\frac{ 429}{32768}$  &  $ \frac{1287}{4096}$  &  $ \frac{ 9009}{8192}$  &  $-\frac{ 3003}{4096}$  &  $ \frac{  9009}{16384}$  &  $-\frac{  1287}{4096}$  &  $ \frac{  1001}{8192}$  &  $-\frac{  117}{4096}$  &  $ \frac{    99}{32768}$ \\ \addlinespace
         &  8  &  $ \frac{6435}{32768}$  &  $ \frac{6435}{4096}$  &  $-\frac{15015}{8192}$  &  $ \frac{ 9009}{4096}$  &  $-\frac{ 32175}{16384}$  &  $ \frac{  5005}{4096}$  &  $-\frac{  4095}{8192}$  &  $ \frac{  495}{4096}$  &  $-\frac{   429}{32768}$ \\ \addlinespace

      8  &  0  &  $-\frac{ 429}{ 2048}$  &  $ \frac{3465}{2048}$  &  $-\frac{12285}{2048}$  &  $ \frac{25025}{2048}$  &  $-\frac{ 32175}{ 2048}$  &  $ \frac{ 27027}{2048}$  &  $-\frac{ 15015}{2048}$  &  $ \frac{ 6435}{2048}$  \\ \addlinespace
         &  1  &  $ \frac{  33}{ 2048}$  &  $-\frac{ 273}{2048}$  &  $ \frac{ 1001}{2048}$  &  $-\frac{ 2145}{2048}$  &  $ \frac{  3003}{ 2048}$  &  $-\frac{  3003}{2048}$  &  $ \frac{  3003}{2048}$  &  $ \frac{  429}{2048}$  \\ \addlinespace
         &  2  &  $-\frac{   9}{ 2048}$  &  $ \frac{  77}{2048}$  &  $-\frac{  297}{2048}$  &  $ \frac{  693}{2048}$  &  $-\frac{  1155}{ 2048}$  &  $ \frac{  2079}{2048}$  &  $ \frac{   693}{2048}$  &  $-\frac{   33}{2048}$  \\ \addlinespace
         &  3  &  $ \frac{   5}{ 2048}$  &  $-\frac{  45}{2048}$  &  $ \frac{  189}{2048}$  &  $-\frac{  525}{2048}$  &  $ \frac{  1575}{ 2048}$  &  $ \frac{   945}{2048}$  &  $-\frac{   105}{2048}$  &  $ \frac{    9}{2048}$  \\ \addlinespace
         &  4  &  $-\frac{   5}{ 2048}$  &  $ \frac{  49}{2048}$  &  $-\frac{  245}{2048}$  &  $ \frac{ 1225}{2048}$  &  $ \frac{  1225}{ 2048}$  &  $-\frac{   245}{2048}$  &  $ \frac{    49}{2048}$  &  $-\frac{    5}{2048}$  \\ \addlinespace
         &  5  &  $ \frac{   9}{ 2048}$  &  $-\frac{ 105}{2048}$  &  $ \frac{  945}{2048}$  &  $ \frac{ 1575}{2048}$  &  $-\frac{   525}{ 2048}$  &  $ \frac{   189}{2048}$  &  $-\frac{    45}{2048}$  &  $ \frac{    5}{2048}$  \\ \addlinespace
         &  6  &  $-\frac{  33}{ 2048}$  &  $ \frac{ 693}{2048}$  &  $ \frac{ 2079}{2048}$  &  $-\frac{ 1155}{2048}$  &  $ \frac{   693}{ 2048}$  &  $-\frac{   297}{2048}$  &  $ \frac{    77}{2048}$  &  $-\frac{    9}{2048}$  \\ \addlinespace
         &  7  &  $ \frac{ 429}{ 2048}$  &  $ \frac{3003}{2048}$  &  $-\frac{ 3003}{2048}$  &  $ \frac{ 3003}{2048}$  &  $-\frac{  2145}{ 2048}$  &  $ \frac{  1001}{2048}$  &  $-\frac{   273}{2048}$  &  $ \frac{   33}{2048}$  \\ \addlinespace

      7  &  0  &  $ \frac{ 231}{ 1024}$  &  $-\frac{ 819}{ 512}$  &  $ \frac{ 5005}{1024}$  &  $-\frac{ 2145}{ 256}$  &  $ \frac{  9009}{ 1024}$  &  $-\frac{  3003}{ 512}$  &  $ \frac{  3003}{1024}$  \\ \addlinespace
         &  1  &  $-\frac{  21}{ 1024}$  &  $ \frac{  77}{ 512}$  &  $-\frac{  495}{1024}$  &  $ \frac{  231}{ 256}$  &  $-\frac{  1155}{ 1024}$  &  $ \frac{   693}{ 512}$  &  $ \frac{   231}{1024}$  \\ \addlinespace
         &  2  &  $ \frac{   7}{ 1024}$  &  $-\frac{  27}{ 512}$  &  $ \frac{  189}{1024}$  &  $-\frac{  105}{ 256}$  &  $ \frac{   945}{ 1024}$  &  $ \frac{   189}{ 512}$  &  $-\frac{    21}{1024}$  \\ \addlinespace
         &  3  &  $-\frac{   5}{ 1024}$  &  $ \frac{  21}{ 512}$  &  $-\frac{  175}{1024}$  &  $ \frac{  175}{ 256}$  &  $ \frac{   525}{ 1024}$  &  $-\frac{    35}{ 512}$  &  $ \frac{     7}{1024}$  \\ \addlinespace
         &  4  &  $ \frac{   7}{ 1024}$  &  $-\frac{  35}{ 512}$  &  $ \frac{  525}{1024}$  &  $ \frac{  175}{ 256}$  &  $-\frac{   175}{ 1024}$  &  $ \frac{    21}{ 512}$  &  $-\frac{     5}{1024}$  \\ \addlinespace
         &  5  &  $-\frac{  21}{ 1024}$  &  $ \frac{ 189}{ 512}$  &  $ \frac{  945}{1024}$  &  $-\frac{  105}{ 256}$  &  $ \frac{   189}{ 1024}$  &  $-\frac{    27}{ 512}$  &  $ \frac{     7}{1024}$  \\ \addlinespace
         &  6  &  $ \frac{ 231}{ 1024}$  &  $ \frac{ 693}{ 512}$  &  $-\frac{ 1155}{1024}$  &  $ \frac{  231}{ 256}$  &  $-\frac{   495}{ 1024}$  &  $ \frac{    77}{ 512}$  &  $-\frac{    21}{1024}$  \\ \addlinespace

      6  &  0  &  $-\frac{  63}{  256}$  &  $ \frac{ 385}{ 256}$  &  $-\frac{  495}{ 128}$  &  $ \frac{  693}{ 128}$  &  $-\frac{  1155}{  256}$  &  $ \frac{   693}{ 256}$  \\ \addlinespace
         &  1  &  $ \frac{   7}{  256}$  &  $-\frac{  45}{ 256}$  &  $ \frac{   63}{ 128}$  &  $-\frac{  105}{ 128}$  &  $ \frac{   315}{  256}$  &  $ \frac{    63}{ 256}$  \\ \addlinespace
         &  2  &  $-\frac{   3}{  256}$  &  $ \frac{  21}{ 256}$  &  $-\frac{   35}{ 128}$  &  $ \frac{  105}{ 128}$  &  $ \frac{   105}{  256}$  &  $-\frac{     7}{ 256}$  \\ \addlinespace
         &  3  &  $ \frac{   3}{  256}$  &  $-\frac{  25}{ 256}$  &  $ \frac{   75}{ 128}$  &  $ \frac{   75}{ 128}$  &  $-\frac{    25}{  256}$  &  $ \frac{     3}{ 256}$  \\ \addlinespace
         &  4  &  $-\frac{   7}{  256}$  &  $ \frac{ 105}{ 256}$  &  $ \frac{  105}{ 128}$  &  $-\frac{   35}{ 128}$  &  $ \frac{    21}{  256}$  &  $-\frac{     3}{ 256}$  \\ \addlinespace
         &  5  &  $ \frac{  63}{  256}$  &  $ \frac{ 315}{ 256}$  &  $-\frac{  105}{ 128}$  &  $ \frac{   63}{ 128}$  &  $-\frac{    45}{  256}$  &  $ \frac{     7}{ 256}$  \\ \addlinespace

      5  &  0  &  $ \frac{  35}{  128}$  &  $-\frac{  45}{  32}$  &  $ \frac{  189}{  64}$  &  $-\frac{  105}{  32}$  &  $ \frac{   315}{  128}$  \\ \addlinespace
         &  1  &  $-\frac{   5}{  128}$  &  $ \frac{   7}{  32}$  &  $-\frac{   35}{  64}$  &  $ \frac{   35}{  32}$  &  $ \frac{    35}{  128}$  \\ \addlinespace
         &  2  &  $ \frac{   3}{  128}$  &  $-\frac{   5}{  32}$  &  $ \frac{   45}{  64}$  &  $ \frac{   15}{  32}$  &  $-\frac{     5}{  128}$  \\ \addlinespace
         &  3  &  $-\frac{   5}{  128}$  &  $ \frac{  15}{  32}$  &  $ \frac{   45}{  64}$  &  $-\frac{    5}{  32}$  &  $ \frac{     3}{  128}$  \\ \addlinespace
         &  4  &  $ \frac{  35}{  128}$  &  $ \frac{  35}{  32}$  &  $-\frac{   35}{  64}$  &  $ \frac{    7}{  32}$  &  $-\frac{     5}{  128}$  \\ \addlinespace

      4  &  0  &  $-\frac{   5}{   16}$  &  $ \frac{  21}{  16}$  &  $-\frac{   35}{  16}$  &  $ \frac{   35}{  16}$  \\ \addlinespace
         &  1  &  $ \frac{   1}{   16}$  &  $-\frac{   5}{  16}$  &  $ \frac{   15}{  16}$  &  $ \frac{    5}{  16}$  \\ \addlinespace
         &  2  &  $-\frac{   1}{   16}$  &  $ \frac{   9}{  16}$  &  $ \frac{    9}{  16}$  &  $-\frac{    1}{  16}$  \\ \addlinespace
         &  3  &  $ \frac{   5}{   16}$  &  $ \frac{  15}{  16}$  &  $-\frac{    5}{  16}$  &  $ \frac{    1}{  16}$  \\ \addlinespace

      3  &  0  &  $ \frac{   3}{    8}$  &  $-\frac{   5}{   4}$  &  $ \frac{   15}{   8}$  \\ \addlinespace
         &  1  &  $-\frac{   1}{    8}$  &  $ \frac{   3}{   4}$  &  $ \frac{    3}{   8}$  \\ \addlinespace
         &  2  &  $ \frac{   3}{    8}$  &  $ \frac{   3}{   4}$  &  $-\frac{    1}{   8}$  \\ \addlinespace

      2  &  0  &  $-\frac{   1}{    2}$  &  $ \frac{   3}{   2}$  \\ \addlinespace
         &  1  &  $ \frac{   1}{    2}$  &  $ \frac{   1}{   2}$  \\ \addlinespace
      \bottomrule
    \end{tabular}
  \end{center}
\end{table}

If we consider the big stencil $S = \cup_{i=0}^k S_k$, we can obtain a $(2r-1)^{th}$ accurate interpolation and \eqref{eq:Lagrange} becomes:

\begin{equation}
  \label{eq:Lagrange_big}
  P(x^*) = \sum_{j=0}^{2r-2} u_{i-r+j+1} \sum_{\substack{l=0 \\ l \neq j}}^{2r-2} \frac{x^* - x_{i-r+l+1}}{x_{i-r+j+1} - x_{i-r+l+1}} = \sum_{j=0}^{2r-2} b_{i-r+j+1} u_{i-r+j+1}
\end{equation}

where $b_{i-r+j+1}$ are the Lagrange coefficients of the stencil $S$.

Expression~\eqref{eq:Lagrange_big} can also be written as a linear convex combination of the $r$ approximations of order $r^{th}$~\eqref{eq:Lagrange}

\begin{equation}
  \label{eq:pol_convex}
  P(x^*) = \sum_{i=0}^{r-1} \gamma_i p_i(x^*) \text{, with } \sum_{i=0}^{r-1} \gamma_i = 1
\end{equation}

where $\gamma_r$ are usually referred as the linear weights. The linear weights for the point $x^*$ can be evaluated from the Lagrange coefficients $a_{k,i-r+j+1}$ and $b_{i-r+j+1}$ by means of:

\begin{equation}
  \label{eq:linear_weights}
  \gamma_k(x^*) = \frac{b_{i-r+j+1} - \sum_{l=0}^{j-1} \gamma_l(x^*) a_{k,i-r+l+1}(x^*)}{a_{0,i-r+j+1}(x^*)} \text{, } j=0, \dots, r-1
\end{equation}

In table~\ref{tab:linear_weights} are reported linear weights from $r=2$ to $r=9$ for $x^*=x_{i+\frac{1}{2}}$; linear weights for $x^*=x_{i-\frac{1}{2}}$ can be obtained by table~\ref{tab:linear_weights} by symmetry.

\begin{table}
  \begin{center}
    \caption{Linear weights from $r=2$ to $r=9$ for $x^*=x_{i+\frac{1}{2}}$}
    \label{tab:linear_weights}
    \begin{tabular}{cccccccccc}
      \toprule
      $r$  &  $j=0$  &  $j=1$  &  $j=2$  &  $j=3$  &  $j=4$  &  $j=5$  &  $j=6$  &  $j=7$  &  $j=8$  \\
      \midrule
      9  & $\frac{1}{65536}$  &  $\frac{ 17}{ 8192}$  &  $\frac{ 595}{16384}$  &  $\frac{1547}{ 8192}$  &  $\frac{12155}{32768}$  &  $\frac{2431}{ 8192}$  &  $\frac{1547}{16384}$  &  $\frac{85}{ 8192}$  &  $\frac{17}{65536}$  \\ \addlinespace
      8  & $\frac{1}{16384}$  &  $\frac{105}{16384}$  &  $\frac{1365}{16384}$  &  $\frac{5005}{16384}$  &  $\frac{ 6435}{16384}$  &  $\frac{3003}{16384}$  &  $\frac{ 455}{16384}$  &  $\frac{15}{16384}$  \\ \addlinespace
      7  & $\frac{1}{ 4096}$  &  $\frac{ 39}{ 2048}$  &  $\frac{ 179}{ 1024}$  &  $\frac{ 429}{ 1024}$  &  $\frac{ 1287}{ 4096}$  &  $\frac{ 143}{ 2048}$  &  $\frac{  13}{ 4096}$  \\ \addlinespace
      6  & $\frac{1}{ 1024}$  &  $\frac{ 55}{ 1024}$  &  $\frac{ 165}{  512}$  &  $\frac{ 231}{  512}$  &  $\frac{  165}{ 1024}$  &  $\frac{  11}{ 1024}$  \\ \addlinespace
      5  & $\frac{1}{  256}$  &  $\frac{  9}{   64}$  &  $\frac{  63}{  128}$  &  $\frac{  21}{   64}$  &  $\frac{    9}{  256}$  \\ \addlinespace
      4  & $\frac{1}{   64}$  &  $\frac{ 21}{   64}$  &  $\frac{  35}{   64}$  &  $\frac{    7}{  64}$  \\ \addlinespace
      3  & $\frac{1}{   16}$  &  $\frac{  5}{    8}$  &  $\frac{   5}{   16}$  \\ \addlinespace
      2  & $\frac{1}{    4}$  &  $\frac{  3}{    4}$  \\ \addlinespace
      \bottomrule
    \end{tabular}
  \end{center}
\end{table}

The basic idea of WENO schemes is to use a nonlinear combination of the $r$ interpolations to obtain a $(2r-1)^{th}$ order interpolation in smooth regions and handle stencil with discontinuities: the nonlinear weights, infact, are close to the linear weights if the function in the stencil is smooth and close to $0$ if in that stencil is contained a discontinuity.

\begin{equation}
  \label{eq:WENO_interp}
  u(x^*) = \sum_{i=0}^{r-1} w_i p_i(x^*)
\end{equation}

Following the work of Jiang and Shu~\cite{jiang-1996}, the nonlinear weights are evaluated as:

\begin{equation}
  \label{eq:nonlinear_weights}
  w_k = \frac{\gamma_k}{\left( \epsilon + \beta_k \right)^2}
\end{equation}

where $\epsilon$ is a parameter to avoid division by zero and $\beta_k$ are the smoothness indicators of the function $u$ on the stencil $l$:

\begin{equation}
  \label{eq:IS}
  \beta_k = \sum_{j=1}^{r-1} \Delta x^{2j-1} \int_{x_{i-\frac{1}{2}}}^{x_{i+\frac{1}{2}}} \left( \frac{d^j p_k(x)}{dx^j} \right)^2 dx
\end{equation}

This is clearly just a scaled sum of the square L2 norms of all the derivatives of the relevant interpolation polynomial $p_k(x)$ in the relevant interval $[x_{i−\frac{1}{2}},x_{i+\frac{1}{2}}]$, where the interpolating point is located. The scaling factor $\Delta_x^{2l-2}$ is to make sure that the final explicit formulas for the smoothness indicators do not depend on the mesh size $\Delta x$.

Substitution of~\eqref{eq:Lagrange} for any $k=0,\dots,r-1$ into~\eqref{eq:IS} yelds to:

\begin{equation}
  \label{eq:IS_u}
  \beta_k = \sum_{j=0}^{r-1} \sum_{l=0}^j \sigma_{k,j,l} u_{i+k-j} u_{i+k-l}
\end{equation}

The coefficients $\sigma_{k,j,l}$ are reported in table~\cref{tab:IS_2-5,tab:IS_6}.

\begin{table}
  \begin{center}
    \caption{Smoothness indicators coefficients from $r=2$ to $r=5$}
    \label{tab:IS_2-5}
    \begin{tabular}{ccccccc}
      \toprule
      $r=2$  \\
      $j$  &  $l$  &  $k=0$ &  $k=1$ \\ \addlinespace
      $1$  &  $1$  &  $-2$  &  $-2$  \\ \addlinespace
           &  $0$  &  $ 1$  &  $ 1$  \\ \addlinespace
      $0$  &  $0$  &  $ 1$  &  $ 1$  \\ \addlinespace
      \midrule
      $r=3$  \\
      $j$  &  $l$  &  $k=0$            &  $k=1$            &  $k=2$            \\ \addlinespace
      $2$  &  $2$  &  $ \frac{11}{3}$  &  $ \frac{ 5}{3}$  &  $ \frac{11}{3}$  \\ \addlinespace
           &  $1$  &  $-\frac{31}{3}$  &  $-\frac{13}{3}$  &  $-\frac{19}{3}$  \\ \addlinespace
           &  $0$  &  $ \frac{10}{3}$  &  $ \frac{ 4}{3}$  &  $ \frac{ 4}{3}$  \\ \addlinespace
      $1$  &  $1$  &  $-\frac{19}{3}$  &  $-\frac{13}{3}$  &  $-\frac{31}{3}$  \\ \addlinespace
           &  $0$  &  $ \frac{25}{3}$  &  $ \frac{13}{3}$  &  $ \frac{25}{3}$  \\ \addlinespace
      $0$  &  $0$  &  $ \frac{ 4}{3}$  &  $ \frac{ 4}{3}$  &  $ \frac{10}{3}$  \\ \addlinespace
      \midrule
      $r=4$  \\
      $j$  &  $l$  &  $k=0$                  &  $k=1$                  &  $k=2$                  &  $k=3$                  \\ \addlinespace
      $3$  &  $3$  &  $-\frac{11389}{1440}$  &  $-\frac{ 2989}{1440}$  &  $-\frac{ 2989}{1440}$  &  $-\frac{11389}{1440}$  \\ \addlinespace
           &  $2$  &  $ \frac{14369}{ 480}$  &  $ \frac{ 1283}{ 160}$  &  $ \frac{ 3169}{ 480}$  &  $ \frac{ 9449}{ 480}$  \\ \addlinespace
           &  $1$  &  $-\frac{ 6383}{ 160}$  &  $-\frac{ 5069}{ 480}$  &  $-\frac{ 3229}{ 480}$  &  $-\frac{ 2623}{ 160}$  \\ \addlinespace
           &  $0$  &  $ \frac{25729}{2880}$  &  $ \frac{ 6649}{2880}$  &  $ \frac{ 3169}{2880}$  &  $ \frac{ 6649}{2880}$  \\ \addlinespace
      $2$  &  $2$  &  $ \frac{ 9449}{ 480}$  &  $ \frac{ 3169}{ 480}$  &  $ \frac{ 1283}{ 160}$  &  $ \frac{14369}{ 480}$  \\ \addlinespace
           &  $1$  &  $-\frac{35047}{ 480}$  &  $-\frac{11767}{ 480}$  &  $-\frac{11767}{ 480}$  &  $-\frac{35047}{ 480}$  \\ \addlinespace
           &  $0$  &  $ \frac{44747}{ 960}$  &  $ \frac{13667}{ 960}$  &  $ \frac{11147}{ 960}$  &  $ \frac{28547}{ 960}$  \\ \addlinespace
      $1$  &  $1$  &  $-\frac{ 2623}{ 160}$  &  $-\frac{ 3229}{ 480}$  &  $-\frac{ 5069}{ 480}$  &  $-\frac{ 6383}{ 160}$  \\ \addlinespace
           &  $0$  &  $ \frac{28547}{ 960}$  &  $ \frac{11147}{ 960}$  &  $ \frac{13667}{ 960}$  &  $ \frac{44747}{ 960}$  \\ \addlinespace
      $0$  &  $0$  &  $ \frac{ 6649}{2880}$  &  $ \frac{ 3169}{2880}$  &  $ \frac{ 6649}{2880}$  &  $ \frac{25729}{2880}$  \\ \addlinespace
      \midrule
      $r=5$  \\
      $j$  &  $l$  &  $k=0$                      &  $k=1$                     &  $k=2$                  &  $k=3$                        &  $k=4$                      \\ \addlinespace
      $4$  &  $4$  &  $ \frac{ 1076779}{60480}$  &  $ \frac{ 221869}{60480}$  &  $ \frac{  98179}{60480}$  &  $ \frac{ 221869}{60480}$  &  $ \frac{ 1076779}{60480}$  \\ \addlinespace
           &  $3$  &  $-\frac{ 5121853}{60480}$  &  $-\frac{1079563}{60480}$  &  $-\frac{ 461113}{60480}$  &  $-\frac{ 847303}{60480}$  &  $-\frac{ 3568693}{60480}$  \\ \addlinespace
           &  $2$  &  $ \frac{ 3141559}{20160}$  &  $ \frac{ 671329}{20160}$  &  $ \frac{ 266659}{20160}$  &  $ \frac{ 395389}{20160}$  &  $ \frac{ 1501039}{20160}$  \\ \addlinespace
           &  $1$  &  $-\frac{ 8055511}{60480}$  &  $-\frac{1714561}{60480}$  &  $-\frac{ 601771}{60480}$  &  $-\frac{ 725461}{60480}$  &  $-\frac{ 2569471}{60480}$  \\ \addlinespace
           &  $0$  &  $ \frac{  668977}{30240}$  &  $ \frac{ 139567}{30240}$  &  $ \frac{  20591}{15120}$  &  $ \frac{  20591}{15120}$  &  $ \frac{  139567}{30240}$  \\ \addlinespace
      $3$  &  $3$  &  $-\frac{ 3568693}{60480}$  &  $-\frac{ 847303}{60480}$  &  $-\frac{ 461113}{60480}$  &  $-\frac{1079563}{60480}$  &  $-\frac{ 5121853}{60480}$  \\ \addlinespace
           &  $2$  &  $ \frac{ 8405471}{30240}$  &  $ \frac{2027351}{30240}$  &  $ \frac{1050431}{30240}$  &  $ \frac{2027351}{30240}$  &  $ \frac{ 8405471}{30240}$  \\ \addlinespace
           &  $1$  &  $-\frac{ 2536843}{ 5040}$  &  $-\frac{ 306569}{ 2520}$  &  $-\frac{ 291313}{ 5040}$  &  $-\frac{  57821}{  630}$  &  $-\frac{ 1751863}{ 5040}$  \\ \addlinespace
           &  $0$  &  $ \frac{12627689}{60480}$  &  $ \frac{2932409}{60480}$  &  $ \frac{1228889}{60480}$  &  $ \frac{1650569}{60480}$  &  $ \frac{ 5951369}{60480}$  \\ \addlinespace
      $2$  &  $2$  &  $ \frac{ 1501039}{20160}$  &  $ \frac{ 395389}{20160}$  &  $ \frac{ 266659}{20160}$  &  $ \frac{ 671329}{20160}$  &  $ \frac{ 3141559}{20160}$  \\ \addlinespace
           &  $1$  &  $-\frac{ 1751863}{ 5040}$  &  $-\frac{  57821}{  630}$  &  $-\frac{ 291313}{ 5040}$  &  $-\frac{ 306569}{ 2520}$  &  $-\frac{ 2536843}{ 5040}$  \\ \addlinespace
           &  $0$  &  $ \frac{ 2085371}{ 6720}$  &  $ \frac{ 539351}{ 6720}$  &  $ \frac{ 299531}{ 6720}$  &  $ \frac{ 539351}{ 6720}$  &  $ \frac{ 2085371}{ 6720}$  \\ \addlinespace
      $1$  &  $1$  &  $-\frac{ 2569471}{60480}$  &  $-\frac{ 725461}{60480}$  &  $-\frac{ 601771}{60480}$  &  $-\frac{1714561}{60480}$  &  $-\frac{ 8055511}{60480}$  \\ \addlinespace
           &  $0$  &  $ \frac{ 5951369}{60480}$  &  $ \frac{1650569}{60480}$  &  $ \frac{1228889}{60480}$  &  $ \frac{2932409}{60480}$  &  $ \frac{12627689}{60480}$  \\ \addlinespace
      $0$  &  $0$  &  $ \frac{  139567}{30240}$  &  $ \frac{  20591}{15120}$  &  $ \frac{  20591}{15120}$  &  $ \frac{ 139567}{30240}$  &  $ \frac{  668977}{30240}$  \\ \addlinespace
      \bottomrule
    \end{tabular}
  \end{center}
\end{table}

\begin{table}
  \begin{center}
    \caption{Smoothness indicators coefficients for $r=6$}
    \label{tab:IS_6}
    \begin{tabular}{cccccccc}
      \toprule
      $r=6$  \\
      $j$  &  $l$  &  $k=0$                           &  $k=1$                           &  $k=2$                           &  $k=3$                           &  $k=4$                           &  $k=5$                           \\ \addlinespace
      $5$  &  $5$  &  $-\frac{ 131759526}{ 3224383}$  &  $-\frac{  24044484}{ 3193217}$  &  $-\frac{  28962993}{14228092}$  &  $-\frac{  28962993}{14228092}$  &  $-\frac{  24044484}{ 3193217}$  &  $-\frac{ 131759526}{ 3224383}$  \\ \addlinespace
           &  $4$  &  $ \frac{ 295095211}{ 1259192}$  &  $ \frac{ 195395281}{ 4459947}$  &  $ \frac{  79135747}{ 6577234}$  &  $ \frac{ 251883319}{23224320}$  &  $ \frac{  26449004}{  769961}$  &  $ \frac{ 112453613}{  657635}$  \\ \addlinespace
           &  $3$  &  $-\frac{ 427867945}{  780329}$  &  $-\frac{ 146902225}{ 1415767}$  &  $-\frac{  95644735}{ 3360137}$  &  $-\frac{  61673356}{ 2721737}$  &  $-\frac{ 347085621}{ 5587817}$  &  $-\frac{ 115324682}{  395671}$  \\ \addlinespace
           &  $2$  &  $ \frac{ 497902688}{  756325}$  &  $ \frac{ 356490569}{ 2842289}$  &  $ \frac{  99590409}{ 2965471}$  &  $ \frac{ 268747951}{11612160}$  &  $ \frac{ 315600562}{ 5645537}$  &  $ \frac{ 586668707}{ 2322432}$  \\ \addlinespace
           &  $1$  &  $-\frac{ 157371280}{  384113}$  &  $-\frac{ 338120165}{ 4351341}$  &  $-\frac{  87214523}{ 4439774}$  &  $-\frac{  74146214}{ 6413969}$  &  $-\frac{ 109600459}{ 4359925}$  &  $-\frac{ 504893127}{ 4547012}$  \\ \addlinespace
           &  $0$  &  $ \frac{ 373189088}{ 7027375}$  &  $ \frac{ 105552913}{10682745}$  &  $ \frac{  30913579}{13651507}$  &  $ \frac{  15418339}{13608685}$  &  $ \frac{  30913579}{13651507}$  &  $ \frac{ 105552913}{10682745}$  \\ \addlinespace
      $4$  &  $4$  &  $ \frac{ 112453613}{  657635}$  &  $ \frac{  26449004}{  769961}$  &  $ \frac{ 251883319}{23224320}$  &  $ \frac{  79135747}{ 6577234}$  &  $ \frac{ 195395281}{ 4459947}$  &  $ \frac{ 295095211}{ 1259192}$  \\ \addlinespace
           &  $3$  &  $-\frac{ 674462631}{  691651}$  &  $-\frac{ 270758311}{ 1365867}$  &  $-\frac{1512485867}{24006092}$  &  $-\frac{1512485867}{24006092}$  &  $-\frac{ 270758311}{ 1365867}$  &  $-\frac{ 674462631}{  691651}$  \\ \addlinespace
           &  $2$  &  $ \frac{1150428332}{  508385}$  &  $ \frac{ 771393469}{ 1663855}$  &  $ \frac{  87743770}{  602579}$  &  $ \frac{ 201365679}{ 1563055}$  &  $ \frac{ 840802608}{ 2367661}$  &  $ \frac{1328498639}{  803154}$  \\ \addlinespace
           &  $1$  &  $-\frac{ 497421494}{  185427}$  &  $-\frac{2984991531}{ 5434265}$  &  $-\frac{ 370146220}{ 2226351}$  &  $-\frac{ 723607356}{ 5654437}$  &  $-\frac{ 288641753}{  912148}$  &  $-\frac{2146148426}{ 1503065}$  \\ \addlinespace
           &  $0$  &  $ \frac{ 498196769}{  609968}$  &  $ \frac{ 169505788}{ 1035915}$  &  $ \frac{  24025059}{  519766}$  &  $ \frac{ 113243845}{ 3672222}$  &  $ \frac{ 142936745}{ 2029182}$  &  $ \frac{ 453375035}{ 1449454}$  \\ \addlinespace
      $3$  &  $3$  &  $-\frac{ 115324682}{  395671}$  &  $-\frac{ 347085621}{ 5587817}$  &  $-\frac{  61673356}{ 2721737}$  &  $-\frac{  95644735}{ 3360137}$  &  $-\frac{ 146902225}{ 1415767}$  &  $-\frac{ 427867945}{  780329}$  \\ \addlinespace
           &  $2$  &  $ \frac{1328498639}{  803154}$  &  $ \frac{ 840802608}{ 2367661}$  &  $ \frac{ 201365679}{ 1563055}$  &  $ \frac{  87743770}{  602579}$  &  $ \frac{ 771393469}{ 1663855}$  &  $ \frac{1150428332}{  508385}$  \\ \addlinespace
           &  $1$  &  $-\frac{ 378281867}{   99229}$  &  $-\frac{ 479783044}{  585775}$  &  $-\frac{ 274966489}{  950662}$  &  $-\frac{ 274966489}{  950662}$  &  $-\frac{ 479783044}{  585775}$  &  $-\frac{ 378281867}{   99229}$  \\ \addlinespace
           &  $0$  &  $ \frac{2292397033}{ 1024803}$  &  $ \frac{ 471933572}{  993629}$  &  $ \frac{ 200449727}{ 1269707}$  &  $ \frac{ 586743463}{ 4237706}$  &  $ \frac{1031953342}{ 2867575}$  &  $ \frac{1406067637}{  859229}$  \\ \addlinespace
      $2$  &  $2$  &  $ \frac{ 586668707}{ 2322432}$  &  $ \frac{ 315600562}{ 5645537}$  &  $ \frac{ 268747951}{11612160}$  &  $ \frac{  99590409}{ 2965471}$  &  $ \frac{ 356490569}{ 2842289}$  &  $ \frac{ 497902668}{  756325}$  \\ \addlinespace
           &  $1$  &  $-\frac{2146148426}{ 1503065}$  &  $-\frac{ 288641753}{  912148}$  &  $-\frac{ 723607356}{ 5654437}$  &  $-\frac{ 370146220}{ 2226351}$  &  $-\frac{2984991531}{ 5434265}$  &  $-\frac{ 497421494}{  185427}$  \\ \addlinespace
           &  $0$  &  $ \frac{1406067637}{  859229}$  &  $ \frac{1031953342}{ 2867575}$  &  $ \frac{ 586743463}{ 4237706}$  &  $ \frac{ 200449727}{ 1269707}$  &  $ \frac{ 471933572}{  993629}$  &  $ \frac{2292397033}{ 1024803}$  \\ \addlinespace
      $1$  &  $1$  &  $-\frac{ 504893127}{ 4547012}$  &  $-\frac{ 109600459}{ 4359925}$  &  $-\frac{  74146214}{ 6413969}$  &  $-\frac{  87214523}{ 4439774}$  &  $-\frac{ 338120165}{ 4351341}$  &  $-\frac{ 157371280}{  384113}$  \\ \addlinespace
           &  $0$  &  $ \frac{ 453375035}{ 1449454}$  &  $ \frac{ 142936745}{ 2029182}$  &  $ \frac{ 113243845}{ 3672222}$  &  $ \frac{  24025059}{  519766}$  &  $ \frac{ 169505788}{ 1035915}$  &  $ \frac{ 498196769}{  609968}$  \\ \addlinespace
      $0$  &  $0$  &  $ \frac{ 105552913}{10682745}$  &  $ \frac{  30913579}{13651507}$  &  $ \frac{  15418339}{13608685}$  &  $ \frac{  30913579}{13651507}$  &  $ \frac{ 105552913}{10682745}$  &  $ \frac{ 373189088}{ 7027375}$  \\ \addlinespace
      \midrule
      \bottomrule
    \end{tabular}
  \end{center}
\end{table}

For example, if $r=3$, \eqref{eq:Lagrange} can be applied to the leftmost stencil $S_0=\left\{ x_{i-2}, x_{i-1}, x_i \right\}$ to obtain the polynomial $p_0$:

\begin{equation}
  \label{eq:pol_0}
  p_0(x_{i+\frac{1}{2}}) = \frac{3}{8} u_{i-2} - \frac{5}{4} u_{i-1} + \frac{15}{8} u_i
\end{equation}

and this approximation is third order accurate if the function $u(x)$ is smooth in the stencil $S_0$. If we choose a different stencil $S_1=\left\{ x_{i-1}, x_{i}, x_{i+1} \right\}$ we obtain the polynomial $p_1$:

\begin{equation}
  \label{eq:pol_1}
  p_1(x_{i+\frac{1}{2}}) = -\frac{1}{8} u_{i-1} + \frac{3}{4} u_i + \frac{3}{8} u_{i+1}
\end{equation}

that is also third order accurate. The last stencil that can be used is the stencil $S_2=\left\{ x_{i}, x_{i+1}, x_{i+2} \right\}$ to obtain the third order accurate interpolating polynom $p_2$:

\begin{equation}
  \label{eq:pol_2}
  p_2(x_{i+\frac{1}{2}}) = \frac{3}{8} u_i + \frac{3}{4} u_{i+1} - \frac{1}{8} u_{i+2}
\end{equation}

When $r=3$, using \eqref{eq:Lagrange_big} on the stencil $S= S_0 \cup S_1 \cup S_2$, we obtain a fifth order accurate approximation of the function $u$ at the point $x_{i+\frac{1}{2}}$:

\begin{equation}
  \label{eq:pol_union}
  P(x_{i+\frac{1}{2}}) = \frac{3}{128} u_{i-2} - \frac{5}{32} u_{i-1} + \frac{45}{64} u_i + \frac{15}{32} u_{i+1} - \frac{5}{128} u_{i+2}
\end{equation}

For $r=3$, $\gamma_0 = \frac{1}{16}$, $\gamma_1 = \frac{5}{8}$, $\gamma_2 = \frac{5}{16}$.


\clearpage

%\input{src/API/API.tex}

\clearpage

%\input{src/tests/tests.tex}

\clearpage

%\input{src/parallel/parallel.tex}

\clearpage

%\input{src/conclusions/conclusions.tex}

\appendix

%\input{src/appendix/euler-1D-parallel-tests-API.tex}

\bibliographystyle{mycpc2}
\bibliography{Reference}

\end{document}
